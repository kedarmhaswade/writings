% use lualatex
%\documentclass[a4paper]{article}   % print quality
\documentclass[17pt]{extarticle}  % smartphone
\usepackage{polyglossia}
\usepackage[margin=20mm]{geometry}
\setdefaultlanguage{marathi}
\setotherlanguages{english}
%\textbf with "Sanskrit Text" produces the shape warning because the default Sanskrit Text font on Windows does not have a bold equivalent; try Yashomudra or Shobhika; now I like Tiro better
%\newfontfamily\devanagarifont[Script=Devanagari]{SanskritText}
%\newfontfamily\devanagarifont[Script=Devanagari]{Shobhika}
\newfontfamily\devanagarifont[Script=Devanagari]{Noto Sans Devanagari}
%\newfontfamily\devanagarifont[Script=Devanagari]{Gargi}
%\newfontfamily\devanagarifont[Script=Devanagari]{Kalimati}
%\setmainfont{ClearSans}
%\setmainfont{Atkinson Hyperlegible}
%\setmainfont{Chomsky}
\setmainfont{Baskervald ADF Std}
%\setmainfont{E B Garamond}
%\setmainfont{Engadget}



%\pagenumbering{gobble} % uncomment to suppress page numbers
\setlength{\parindent}{0pt}% Remove paragraph indent
\usepackage[skip=\medskipamount]{parskip}
\usepackage{fontawesome}
\usepackage[style=english]{csquotes}
\usepackage{verse}
\newcommand{\attrib}[1]{%
\nopagebreak{\raggedleft\footnotesize #1\par}}
\usepackage{lettrine}
\usepackage{xcolor}
\usepackage{hyperref}
\hypersetup{
    colorlinks=true,
    linkcolor=blue,
    filecolor=magenta,
    citecolor=blue,
    urlcolor=purple,
}

% control hyphenation: https://tex.stackexchange.com/a/177179/64425
\tolerance=1
\emergencystretch=\maxdimen
\hyphenpenalty=10000
\hbadness=10000
% control hyphenation: https://tex.stackexchange.com/a/177179/64425

\automatichyphenmode=1

\begin{document}

\title{हृषीकेशच्या मौजप्रश्नावलीची संक्षिप्त उत्तरे}
\author{केपी}
\date{पुणे, ०८ ऑगस्ट २०२५}
%\date{}
\maketitle
\vspace{5mm}
\hrule

\lettrine[lines=3]{का}{ही} दिवसांपूर्वी प्रिय मित्र हृषीकेश नेनेने हा एक स्तुत्य उपक्रम चालू केला: मित्रांना त्यांच्या वाढदिवशी अभीष्टचिंतन करून त्यांना काही प्रश्नांची उत्तरं द्यायची संधी द्यायची. ज्याला रस असेल त्याने उत्तरं लिहायची असा क्रम चालू झाला. 

खरं बघायला गेलं तर नित्यनूतन होणाऱ्या चैतन्यमय निसर्गाप्रमाणे आपल्यात आणि आपल्या मतांमध्ये सूक्ष्म बदल घडत असतो. आज आपल्याला जे एखाद्या गोष्टीबद्दल वाटतं तेच काही वर्षांपूर्वी वाटत होतं असं नाही. आणि बदल घडलाच तर तो प्रांजळपणे मान्य करणं  यात काही कमीपणा नाही.

काही मतं वज्रलेप असू शकतात, पण ती बदलायला आणखी काही काळ जावा लागेल असंही म्हणता येईल. 

एरीक् फ्रॉम् ने म्हटल्याप्रमाणे, 

\begin{english}
%\settowidth{\versewidth}{12345678901234567890123456789012345678901234567890}
%\begin{verse}[\versewidth]
\begin{displayquote}

\enquote{We are aware of the existence of a self, of a core in our personality which is unchangeable and which persists throughout our life in spite of varying circumstances, and regardless of certain changes in opinions and feelings. It is this core which is the reality behind the word `I', and on which our conviction of our own identity is based.}
%\end{verse}
\end{displayquote}
\attrib{\textit{Eric Fromm}, The Art of Loving}
\end{english}


तर असा हा काळाच्या ओघात बदलणारा \enquote*{सेल्फ्} कुठेतरी पकडता आला, \enquote{या या वेळी मला असं वाटत होतं} असं जर लिहून ठेवता आलं तर मजा येईल असं मला वाटलं आणि हृषीकेश ने दिलेल्या या संधीचा लाभ घ्यावा असं ठरवलं. दर काही वर्षांनी अशी नोंद करणे हा एक बरा विरंगुळा आहे\begin{english}---\end{english}सर्वांनाच आवडू शकेल असा, उत्तरं प्रसिद्ध नाही केलीत तरी चालेल. 

शिवाय मला कोण माझे विचार आणि आवडीनिवडी आत्मीयतेने विचारणार आहे?

म्हणूनच हा खटाटोप, इंग्लिश प्रश्नांना उत्तरे इंग्लिशमध्ये, तर मराठी प्रश्नांना मराठीत उत्तरे द्यायचा प्रयत्न केला आहे\dots

\begin{enumerate}
\item
\textbf{तू सध्या काय करतोस?}\\
अर्थार्जनासाठी थोडीफार (किंवा फार थोडी) गुंतवणूक करतो, पण पगारी नोकरी नाही. ज्ञानार्जन मात्र चालू असते: अभ्यास, वाचन, अध्ययन-अध्यापन वगैरे. अशा व्यक्तींना इंग्रजीत बिझी बीव्हर् म्हणतात बहुतेक. \enquote{पुढे काय?} या प्रश्नाला फारसं समाधानकारक उत्तर देता येत नाही बऱ्याचदा, पण सध्या काळ चांगला चालला आहे असं वाटतं. 

तसा मी \begin{english} \href{https://en.wikipedia.org/wiki/Hedonism}{present hedonist}\end{english} असल्याने \enquote{पुढचं पुढे} असा विचार करायला मला निसर्गाचीच फूस आहे.

\begin{english}

\item \textbf{Three Items in Bucket List?}\\
\begin{enumerate}
\item Visit MCG, the Melbourne Cricket Ground, and watch a boxing-day test match for a day.
\item Visit Antarctica.
\item Learn French.
\item Ride a spacecraft.
\item Die like a hedonist, before the infirmity of the old age defeats me: Live fast, die young, and leave a good-looking corpse behind.
\end{enumerate}

\item 
\textbf{Your Role Model(s)?}\\
\begin{enumerate}
\item Donald Knuth.
\item Manfred Riem.
\item Martin Gardner.
\item B. V. Joshi (my father-in-law).
\item Kurt G\"odel. I don't think I want to be like G\"odel (which is what role models beckon you to do), but I am extremely interested in his Incompleteness Theorems.
\item Alexander Grothendieck. It will be more appropriate to say I am curious about his life; he's not a \enquote{role model}.
\item I am also intrigued by the lives of many great, but freethinking yet unknown authors, mathematicians, scientists, artists, athletes, open-source computer programmers.
\end{enumerate}

\item
\textbf{Favourite movie?}\\
\begin{marathi}
फार सिनेमे बघायला जमत नाही मला सध्या, पण एखादी शिफारस आली तर बघणं होतं. काही आवडलेले:
\end{marathi}
\begin{enumerate}
\item 
\begin{english}
\href{https://en.wikipedia.org/wiki/The_Bear_(1988_film)}{The Bear} (This, in my opinion, is the best animal movie ever made). The spirit of Curwood's immortal quote, \enquote{The greatest thrill is not to kill but to let live.} is brilliantly captured in the wildlife.
\item I have liked parts of \href{https://en.wikipedia.org/wiki/Into_the_Wild_(film)}{Into the Wild}.
\end{english}

\item \begin{marathi} बदला (अमिताभ, तापसी पण्णू).\end{marathi}
\item \begin{marathi} शामची आई.\end{marathi}

\item I like several Studio Ghibli movies like Grave of the Fireflies, Spirited Away, and My Neighbor Totoro.
\end{enumerate}

\item 
\textbf{Favourite Heroine/Hero?}\\
\begin{enumerate}
\item I used to like Madhuri Dixit when we were (both ;-)) younger. They used to call me MKD (Madhuri Ka Deewaanaa) in IIT Madras. I am not sure I'll like her now (haven't watched her recently).
\item I grew up with Amitabh Bacchan and his aura of a \enquote*{hero} hasn't left me yet.
\end{enumerate}

\item
\textbf{Favourite song(s)?}\\
Too many to list here, of course. A few:
\begin{enumerate}
\item \begin{marathi}\href{https://youtu.be/qc-Nj-39BJ4?si=f-JNw4ILa6B5vHVd&t=60}{पराधीन आहे जगती\footnote{१९ तारखेला हे लिहिले; पहिल्यावेळी कसा विसरलो काय माहित!}}\end{marathi}
\item \begin{marathi}\href{https://www.youtube.com/watch?v=fRlPQHFlfyc&list=RDfRlPQHFlfyc&start_radio=1&rv=7rebbrL8RO8}{कहाॅं तक ये मन को अंधेरे छलेंगे\footnote{१२ सप्टेंबर २०२५ ला हे लिहिले; पहिल्यावेळी कसा विसरलो काय माहित!}}\end{marathi}
\item \begin{marathi}\href{https://youtu.be/I9Ehun8Ahk4?si=mRLlBz0B9LF-880y}{श्रावणात घननीळा}\end{marathi}
\item\begin{marathi}\href{https://youtu.be/QjCdcO4t1qI?si=HYx-DluK-yfos_GU}{ये जवळी घे जवळी}\end{marathi}
\item \begin{marathi}\href{https://youtu.be/4CwFFWleNNA?si=ZL-2KpfQ4d7v4GRL}{फूलों के रंग से}\end{marathi}
\item \begin{marathi}\href{https://youtu.be/I2RefAyeVRA?si=RrVtDNg5ekkHvkSH}{गुजरा हुआ जमाना, आता नहीं दुबारा} \end{marathi} I listened to it again while writing this.
\item \href{https://youtu.be/M11SvDtPBhA?si=2zDFraxrFgbzlw7y}{Party in the USA}.
\end{enumerate}

\item \textbf{Favourite singer(s)?}\\
\begin{enumerate}
\item Lata Mangeshkar.
\item Kishore Kumar.
\item Asha Bhosale.
\end{enumerate}

\item \textbf{Favourite Achievement(s)?}\\
\begin{enumerate}
\item Being surrounded by loving people. Being lucky in this regard is an achievement.
\item A much lesser one: The score of 99.74 percentile in the GATE 1995 exam (on a second attempt).
\end{enumerate}
\item \textbf{Favourite book(s)?}\\
I can't even choose the top ten or top 100. Just for the mention, I will pick one.
\begin{enumerate}
\item (Fiction) 1984 by George Orwell.
\end{enumerate}

\item \textbf{Favourite quote(s)?}\\
Even harder choice. My love of pithy quotes is well-known among my friends. I am sorry, but I will pass this one.

\item \textbf{Favourite place/destination(s)?}\\

\begin{enumerate}
\item Various restaurants in Pune.
\item Stanford University.
\end{enumerate}

\item\textbf{Favourite food(s)?}\\
\begin{enumerate}
\item \begin{marathi}पुरणपोळी (मी करतो ती, \href{https://github.com/kedarmhaswade/writings/blob/main/marathi/puraNa-poLii/saakharechii-puraNapoLii.pdf}{साखरेची} :-))\end{marathi}
\end{enumerate}

\item\textbf{Favourite drink(s)?}\\
\begin{enumerate}
\item \begin{marathi}चहा\end{marathi}
\end{enumerate}

\item\textbf{Favourite ice cream flavour(s)?}\\
None.

\item\textbf{Favourite hobbies/pass time(s)?}\\
\begin{enumerate}
\item Reading. This one is enough.
\item Computer programming, especially conceiving problems and solving them with succinct and readable computer programs.
\item Chess.
\item Typesetting documents using \LaTeX.
\item Geography, for example, exploring Google Maps.
\item Cooking (especially making \begin{marathi}पुरणपोळी\end{marathi}).
\end{enumerate}

\item\textbf{Favourite sport(s)?}\\
\begin{enumerate}
\item (To watch) Cricket.
\item (To watch) Parallel Bars (Women's gymnastics), Roman Rings (Men's gymnastics).
\item (To watch) American Football.
\item (To play) Cricket.
\item (To play) Badminton.
\item (To play and watch) Chess (yes, Chess can be a sport ;-)).
\item I used to like \begin{marathi}अप्पारप्पी\end{marathi} in high school.
\item (To play?) Bicycling.
\item (To watch) Tennis.
\end{enumerate}

\item\textbf{Favourite sportspeople?}\\
\begin{enumerate}
\item Sachin Tendulkar.
\item Virat Kohli.
\item Tom Brady (American Football).
\item Carlos Alcaraz.
\end{enumerate}


\item\textbf{Your strength/talent/secret weapons?}\\

That's a trick question. Secret weapons are not to be divulged like this! 

Ever since I read Peter Drucker's seminal essay, Managing Yourself, I have been trying hard to identify my true strengths and weaknesses. I have been largely unsuccessful. But perhaps one strength is patience. Another strength is I am fond of personalities.

Like Einstein, perhaps, I am passionately curious. But I am painfully aware of the fact that being curious is a tough position.

\item\textbf{Any birthday resolution(s)?}\\

None, specifically, but usually I want to invest in myself.

\item\textbf{And, Anything else you wish we had asked?}\\

Do you have any regrets?

-- Another good friend, Rahul (RaKu) Kulkarni asked me to answer my own question. Here's my response\footnote{I responded on the group on that day, but updated it here on 19 August 2025.}: 

\begin{marathi}रिग्रेट च्या बाबतीत सांगायचं तर पुन्हा ऑस्कर वाईल्ड (च्या कोट) चा आधार घेतला पाहिजे!\end{marathi} He once said, ``Regretting your experience is denying your existence.'' Since, contrapositively, I don't deny my existence since 1972, I must not regret my experiences! So, no, absolutely no regrets.
\end{english}
\end{enumerate}
\vspace{5mm}
\hrule
\end{document}
