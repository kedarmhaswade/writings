% use xelatex 
%\documentclass[a4]{article}   % print quality 
\documentclass[17pt]{extarticle}  % smartphone 
\usepackage{polyglossia} 
\usepackage{enumitem}
\usepackage[margin=20mm]{geometry} 
\setdefaultlanguage{marathi} 
\setotherlanguages{english} 
\newfontfamily\devanagarifont[Script=Devanagari, Mapping=devanagarinumerals, StylisticSet=1]{Tiro Devanagari Marathi}
%\newfontfamily\devanagarifont[Script=Devanagari, Mapping=devanagarinumerals, StylisticSet=1]{Shobhika}
\newfontfamily\englishfont{Baskervald ADF Std} 
 
 	
\pagenumbering{gobble} % uncomment to suppress page numbers 
\setlength{\parindent}{0pt}% Remove paragraph indent 
\usepackage[skip=\medskipamount]{parskip} 

\usepackage{lettrine} 
\usepackage{xcolor} 
\usepackage[normalem]{ulem} % strikethrough 

\usepackage[style=english]{csquotes} % quotes

% verses
\usepackage{verse}
% verses

\usepackage{xcolor}

\usepackage{float} % table placement [H]

% tikz 
\usepackage{tikz} 
\usetikzlibrary{arrows.meta} 
\usetikzlibrary{calc} 
% tikz 

\usepackage{graphicx}

\usepackage{caption}
\captionsetup[figure]{
    position=below,
}

\usepackage{lettrine} % multiline letter to begin the article
\usepackage{hyperref}
\hypersetup{
    colorlinks=true,
    linkcolor=blue,
    filecolor=magenta,
    citecolor=blue,
    urlcolor=purple,
}

% control hyphenation: https://tex.stackexchange.com/a/177179/64425
\tolerance=1
\emergencystretch=\maxdimen
\hyphenpenalty=10000
\hbadness=10000
% control hyphenation: https://tex.stackexchange.com/a/177179/64425

\newcommand{\eng}[1]{\begin{english}#1\end{english}}

\begin{document}
\title{शारदा शिंत्रे: एक चिंतन}
\author{केदार म्हसवडे}
\maketitle
\lettrine[lines=2]{दो}{न} ऑक्टोबरला मयूरचा व्हॉट्सॅप् वर अचानक फोन आला. मी कामात होतो; फोन न घेता, \enquote{फोन करतो ५ मिनिटांत} असा मेसेज् त्याला लिहून, मग फोन केला. तो म्हणाला, \enquote{अरे, सुयशची आई\textemdash शिंत्रे काकू\textemdash गेली. दुपारी काकांशी बोलून त्या जरा पडल्या होत्या आणि झोपेतच गेल्या; साधारण ५-५.३० ला. मी आत्ता जाऊन पाहतो, मग कळवतो.}

सुन्न झालो. फोन यंत्रवत ठेवला. बातमीवर विश्वास बसतच नव्हता. त्यांचं वय काही फार नव्हतं. मला त्यांच्या व्यक्तिमत्वाचा होत असलेला परिचय अल्पकाळ टिकेल असं वाटलं नव्हतं. पण कोणीतरी म्हटलेलं \eng{\enquote{All good things come to an end!}} आठवलं आणि आलेला दुर्धर प्रसंग मूकपणे स्वीकारायचा हे मला  

मला त्यांच्या घरी नुकताच घडलेला प्रसंग आठवला. पुण्यात असेन तेव्हा त्यांच्या \enquote{१०१ वेणूनाद} मधल्या घरी जाऊन यायचे हा माझा शिरस्ता असे. परवा मी त्यांच्याकडे बरेच भोवरे घेऊन गेलो होतो. 
\end{document}
