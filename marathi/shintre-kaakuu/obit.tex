% use xelatex 
%\documentclass[a4]{article}   % print quality 
\documentclass[17pt]{extarticle}  % smartphone 
\usepackage{polyglossia} 
\usepackage{enumitem}
\usepackage[margin=20mm]{geometry} 
\setdefaultlanguage{marathi} 
\setotherlanguages{english} 
\newfontfamily\devanagarifont[Script=Devanagari, Mapping=devanagarinumerals, StylisticSet=1]{Tiro Devanagari Marathi}
%\newfontfamily\devanagarifont[Script=Devanagari, Mapping=devanagarinumerals, StylisticSet=1]{Shobhika}
\newfontfamily\englishfont{Shobhika} 
 
 	
\pagenumbering{gobble} % uncomment to suppress page numbers 
\setlength{\parindent}{0pt}% Remove paragraph indent 
\usepackage[skip=\medskipamount]{parskip} 

\usepackage{lettrine} 

\usepackage{epigraph}
% Customize epigraph style
\setlength{\epigraphwidth}{0.8\textwidth} % Set width to 80% of text width
\setlength{\epigraphrule}{2pt} % Remove the rule
\renewcommand{\epigraphflush}{center} % Center the epigraph


\usepackage{xcolor} 
\usepackage[normalem]{ulem} % strikethrough 

\usepackage[style=english]{csquotes} % quotes

% verses
\usepackage{verse}
% verses

\usepackage{xcolor}

\usepackage{float} % table placement [H]

% tikz 
\usepackage{tikz} 
\usetikzlibrary{arrows.meta} 
\usetikzlibrary{calc} 
% tikz 

\usepackage{graphicx}

\usepackage{caption}
\captionsetup[figure]{
    position=below,
}

\usepackage{lettrine} % multiline letter to begin the article
\usepackage{hyperref}
\hypersetup{
    colorlinks=true,
    linkcolor=blue,
    filecolor=magenta,
    citecolor=blue,
    urlcolor=purple,
}

% control hyphenation: https://tex.stackexchange.com/a/177179/64425
\tolerance=1
\emergencystretch=\maxdimen
\hyphenpenalty=10000
\hbadness=10000
% control hyphenation: https://tex.stackexchange.com/a/177179/64425

\newcommand{\eng}[1]{\begin{english}#1\end{english}}

\begin{document}
\title{शारदा शिंत्रे: एक चिंतन}
\author{केदार म्हसवडे}
\maketitle
\lettrine[lines=2]{दो}{न} ऑक्टोबरला मयूरचा व्हॉट्सॅप् वर अचानक फोन आला. मी कामात होतो; फोन न घेता, \enquote{फोन करतो ५ मिनिटांत} असा मेसेज् त्याला लिहून, मग फोन केला. तो म्हणाला, \enquote{अरे, सुयशची आई\textemdash शिंत्रे काकू\textemdash गेली. दुपारी काकांशी बोलून त्या जरा पडल्या होत्या आणि झोपेतच गेल्या; साधारण ५-५.३० ला. मी आत्ता जाऊन पाहतो, मग कळवतो.}

सुन्न झालो. फोन यंत्रवत ठेवला. बातमीवर विश्वास बसतच नव्हता. त्यांचं वय काही फार नव्हतं. मला त्यांच्या व्यक्तिमत्वाचा होत असलेला परिचय असा अल्पकाळ टिकेल असं वाटलं नव्हतं. पण कोणीतरी म्हटलेलं \eng{\enquote{All good things come to an end!}} आठवलं आणि आलेला दुर्धर प्रसंग मूकपणे स्वीकारायचा असं ठरवलं.

मला त्यांच्या घरी नुकताच घडलेला प्रसंग आठवला. पुण्यात असेन तेव्हा त्यांच्या घरी जाऊन यायचं हा माझा शिरस्ता असे. परवा मी त्यांच्याकडे बरेच भोवरे घेऊन गेलो होतो. बरेचसे त्यांना जमिनीवर फिरवून, हवेतल्या हवेत फिरवून तळहातावर घेऊन  दाखवले. मग त्यांनी त्यांची पिशवी काढली. त्यात एक \enquote*{न्यूटनचा पाळणा} म्हणून प्रसिद्ध असलेलं, पण त्यापेक्षा छान खेळणं होतं. दोन पोकळ जाडसर प्लॅस्टिक् चे बॉल आणि त्यांना जोडलेला नायलॉन चा बारीक दोरा अशीच त्याची साधी योजना होती. तो दोरा वरखाली कौशल्याने केला तर कितीतरी वेळ त्या दोन बॉल्स् चे आपटणे चालू राहते. त्या लीलया खेळत होत्या. मला जरा वेळ लागला, पण नंतर जमलं. 

काकू नेहमी म्हणायच्या, \enquote{पुण्यात असलास की १५ मिनिटं येऊन जात जा, फार काही नको, पण कसं काय चाललंय वगैरे सांगून जात जा}. मग मी त्यांच्याकडे जमेल तसा जाऊन येत असे. मे २०२४ मध्ये एक दिवस म्हणाल्या, \enquote{तू आता इथे असशील काही दिवस तर माझं तुझ्याकडे एक काम आहे. उद्यापासून ५ दिवस माझ्याकडे तासभर यायचं. आपण जमेल तशी वेळ ठरवू या.} मी \enquote*{बरं} म्हटलं आणि ठरलेल्या वेळी त्यांच्याकडे पोहोचलो. एक टेबल, त्यावर असंख्य गोष्टी (पसारा म्हणा हवं तर, पण \enquote*{पसारा} या शब्दाकडे सकारात्मकपणे बघण्याची फॅशन् नाहीये सध्या),  दोन  खुर्च्या अशी साधीशीच मांडणी होती. 

त्या म्हणाल्या, \enquote{मी तुला आज मराठी वाचायला शिकवणार आहे}. मी जरा आश्चर्याने त्यांच्याकडे पाहिलं, पण त्यांनी \enquote{नंतर त्याचा खुलासा करीन} असं खुणवून सुरुवात केली. प्रसंगावधान बाळगून मी माझ्या फोनवर \enquote{ध्वनी-मुद्रण करतो} असं म्हणत रेकॉर्ड् चं बटण दाबलं. हळूहळू मला त्यांनी त्यांची शैली, साधी-सोपी अक्षरशः असंख्य साधने\textemdash तक्ते, कागद, वह्या, पेन्सिली, खेळणी, हस्तकलेच्या गोष्टी, \enquote*{शिकवणी}चा आभास होऊ न देता साधलेला संवाद, अफाट शब्दसंपदा, बुद्धीला चालना देणारं पण आढ्यतेचा स्पर्शही नसलेलं संभाषण, मराठीवरचं प्रभुत्व, आणि मराठी \enquote*{वाचायला} कसं शिकवावं यावरचं संशोधन इत्यादींनी इतकं मंत्रमुग्ध करून टाकलं की तास कसा गेला ते कळलं देखील नाही. 

मग काय, ५ दिवस ५ तास आम्ही मराठी वाचायला भेटलो. त्या शिकवत होत्या आणि मी, वय होऊन गेल्यावरही, मराठी वाचायला (आणि ते कसं शिकवावं हेही) सहजपणे शिकत होतो. माझ्याकडे काही शिकवण्यातली पदवी नाही, पण \enquote{रम्य ते बघूनिया मज वेड लागे} असा माझा कल असल्याने, ही पद्धत आणि त्यामागचा व्यासंग, ध्यास, आणि तळमळ माझ्या लक्षात यायला काही वेळ लागला नाही. त्या हे सगळं कसं करतात याचा एक व्हिडिओ काढावा असं मला वाटलं, पण दुर्दैवाने तेवढा वेळ नव्हता. 

ही पद्धत त्यांनी १९९२ च्या \enquote*{रोहिणी} अंकात \cite{२५तास} काहीशा विस्ताराने लिहिली आहे. अनेकांनी \href{https://github.com/kedarmhaswade/writings/blob/main/marathi/shintre-kaakuu/maraathii-shikavuu-yaa-godiine.pdf}{ती} पहायला हवी, आपल्या मुलामुलींना \href{https://github.com/kedarmhaswade/writings/blob/main/marathi/shintre-kaakuu/maraathii-shikavuu-yaa-godiine.pdf}{त्याप्रमाणे} वाचायला शिकवावं असं मला फार वाटलं.

मी संस्कृतभारतीचा एकेकाळी कार्यकर्ता होतो. तेव्हा थोडंफार संस्कृत बोलायला शिकवायचा प्रयत्न करायचो. माझ्या तेव्हाच्या शिक्षिका\textemdash सौ. चारुहासिनी भावे\textemdash (त्याही अशाच ग्रेट आहेत!) म्हणायच्या, \enquote{संस्कृतस्य चतुस्सोपानाः श्रवणं सम्भाषणं पठनं लेखनम् इति।} तद्वत् शिंत्रे काकू म्हणायच्या:

\epigraph{\itshape एकदा वाचायचं वेड लागलं, किंवा झिंग चढली, की माणूस मुक्त होतो. काय वाचू आणि काय नको असं त्याला होतं. अशा \enquote*{वाचनाची} गोडी लहान मुलांमध्ये निर्माण करण्याचं काम फार मोठं नसलं तरी महत्त्वाचं आहे, पण उपेक्षित आहे. कुसुमाग्रजादि दिग्गजांनी \enquote*{वर} काम खूप केलंय, खूप वाचनीय लिखाण मराठीत आहे. पण ते वाचायला \enquote*{खाली} करावं लागणारं काम झालं नाहीये, होत नाहीये.}
{\textit{मराठी शिकवू या गोडीने-वेगाने}\\ \textsc{शारदा शिंत्रे}}

कालांतराने कळलं की काकूंनी १९६७-६८ सालच्या पुणे विद्यापीठाच्या मराठी एम्.ए. च्या परीक्षेत सुवर्णपदक मिळवलं होतं (हे त्यांनी मला कधीच सांगितलं नाही, जाणवू दिलं नाही). तेव्हापासून मराठीसाठी असं काही करावं असा विचार काहीसा बंडखोरपणे त्यांनी अंगी बाळगला होता. त्याप्रमाणे अनेकांना (स्वतः च्या मुलांना आणि इतर मुलांना, इतर शिक्षकांना) शिकवलं होतं.

त्यांनी २०२४ मध्ये \enquote{माझं तुझ्याकडे काम आहे} असं सांगून मला \textit{मराठी वाचायला कसं शिकवावं} याचे धडे का दिले ते ठाऊक नाही. पण तसं त्यांनी केलं हे माझ्या दृष्टीने वरदान ठरेलही. आता पुन्हा \enquote*{१०१, वेणूनाद} मध्ये जाईन तेव्हा त्यांनी मला शिकवलेले मराठी-वाचन आठवेल. आता पाणावताहेत तशा डोळ्यांच्या कडा पाणावतील. त्यांचा वेणू अबोल झाला असला तरी टेरी प्रॅचेट् ने म्हटल्याप्रमाणे मी माझ्या मनाची समजूत घालेन, जमलं तर त्यांच्या पावलावर पाऊल ठेवत (त्यांच्याइतक्या समर्थपणे नाही होणार) कदाचित इतर लहानग्यांना मराठी-वाचनाचे धडे देईन. हीच त्या शारदेला माझी श्रद्धांजली \dots

\epigraph{\itshape "Do you not know that a man is not dead while his name is still spoken?"}{\textit{Going Postal}\\ \textsc{Terry Pratchett}}

\begin{thebibliography}{99}
\bibitem{२५तास} मराठी शिकवू या गोडीने-वेगाने. शारदा शिंत्रे. रोहिणी बालक-पालक विशेषांक, सप्टेंबर १९९२. \href{https://github.com/kedarmhaswade/writings/blob/main/marathi/shintre-kaakuu/maraathii-shikavuu-yaa-godiine.pdf}{(पी.डी.एफ्)}
\end{thebibliography}
\end{document}
