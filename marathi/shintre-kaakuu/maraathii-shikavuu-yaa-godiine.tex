% use lualatex 
%\documentclass[a4paper]{article}   % print quality 
\documentclass[17pt]{extarticle}  % smartphone 
\usepackage{polyglossia} 
\usepackage[margin=20mm]{geometry} 
\setdefaultlanguage{marathi} 
\setotherlanguages{english} 
%\textbf with "Sanskrit Text" produces the shape warning because the default Sanskrit Text font on Windows does not have a bold equivalent; try Yashomudra or Shobhika; now I like Tiro better 
%\newfontfamily\devanagarifont[Script=Devanagari]{SanskritText} 
\newfontfamily\devanagarifont[Script=Devanagari]{Shobhika} 
%\newfontfamily\devanagarifont[Script=Devanagari]{Noto Sans Devanagari} 
%\newfontfamily\devanagarifont[Script=Devanagari]{Gargi} 
%\newfontfamily\devanagarifont[Script=Devanagari]{Kalimati} 
%\setmainfont{ClearSans} 
%\setmainfont{Atkinson Hyperlegible} 
%\setmainfont{Chomsky} 
\setmainfont{Baskervald ADF Std} 
%\setmainfont{E B Garamond} 
%\setmainfont{Engadget} 
 
 
 
%\pagenumbering{gobble} % uncomment to suppress page numbers 
\setlength{\parindent}{0pt}% Remove paragraph indent 
\usepackage[skip=\medskipamount]{parskip} 
\newcommand{\attrib}[1]{% 
\nopagebreak{\raggedleft\footnotesize #1\par}} 
\usepackage{lettrine} 
\usepackage{xcolor} 
\usepackage[normalem]{ulem} % strikethrough 
\usepackage[style=english]{csquotes}
\usepackage{verse}
\usepackage{xcolor}
% tikz 
\usepackage{tikz} 
\usetikzlibrary{arrows.meta} 
\usetikzlibrary{calc} 
% tikz 

\usepackage{graphicx}

\usepackage{caption}
\captionsetup[figure]{
    position=below,
}

\usepackage{hyperref}
\hypersetup{
    colorlinks=true,
    linkcolor=blue,
    filecolor=magenta,
    citecolor=blue,
    urlcolor=purple,
}

% control hyphenation: https://tex.stackexchange.com/a/177179/64425
\tolerance=1
\emergencystretch=\maxdimen
\hyphenpenalty=10000
\hbadness=10000
% control hyphenation: https://tex.stackexchange.com/a/177179/64425

\newcommand{\eng}[1]{\begin{english}#1\end{english}}
\begin{document}

\title{मराठी शिकवू या गोडीने-वेगाने\\ \large{(२५ तासांत मराठी)}}
\author{लेखिका: कै.\footnote{शिंत्रेकाकूंचे ०२ ऑक्टोबर २०२५ रोजी दुःखद निधन झाले.} $\;$ सौ. शारदा शिंत्रे\\ \tiny{(टंकलेखक: केदार म्हसवडे)}}
\date{रोहिणी बालक-पालक विशेषांक, सप्टेंबर १९९२}
%\date{}
\maketitle
\vspace{5mm}

\hrule
{\large एखादा विषय येत नाही म्हटलं की मुले त्या विषयाकडे पाठ फिरवतात. प्रयत्न केला तर जमेल हे मात्र त्यांना वाटत नाही. अशावेळी पालकांनी काही वेगळे प्रयोग करायला हवेत. या लेखात दिलेला मराठी वाचनाचा (आणि मराठीचा) अभ्यास हा एक प्रयोगच आहे, लेखिकेने स्वतः केलेला\textendash एक पालक, एक शिक्षक म्हणून आणि, या पद्धतीचा समाजाला उपयोग व्हावा ह्या हेतूने शब्दबद्ध केलेला\textendash एक पत्रकार म्हणून.}

\hrule

\lettrine[lines=3]{प्र}{त्येक}	 व्यक्तीच्या जडणघडणीमध्ये \enquote*{भाषा} हा फार महत्त्वाचा घटक असतो. डॉ. पु. ग. सहस्रबुद्धे तर भाषाशिक्षणाला फार महत्त्वाचे स्थान देत! \enquote{एखादा विषय शिकतो तेव्हा आपण तोच एक विषय शिकतो. तेव्हा एक जीवन पद्धती आत्मसात करत असतो.} असा ते आवर्जून उल्लेख करीत. एरवी हे सारे आपल्या कळत नकळत घडत असते. परंतु माझ्या मुलाचे शिक्षण इंग्रजी माध्यमातून जेव्हा सुरू झाले, तेव्हा अनेक गोष्टी पालक म्हणून मला, व शिकताना त्यालाही त्रासदायक ठरू लागल्या. तेव्हा विचारचक्राला गती मिळत गेली. या आधी मुलीचे शिक्षण मराठी माध्यमातून झाले होते. तेव्हाही अनेक बाबी मला तितक्याशा पसंत नव्हत्या. विचार करू लागल्यावर त्या नापसंत चालीही पुन्हा आठवल्या. एक पालक म्हणून अस्वस्थ होण्यापलीकडे मी काहीच करू शकत नव्हते. १९७९ साली बालवाडी शिक्षिका होण्याचा योग आला तेव्हा डोक्यातले अनेक प्रयोग स्वतः करून पाहून, तावून सुलाखून घेण्याची संधी मिळाली. त्या प्रयोगामधून १९८१ पासून \enquote{मराठी शिकवू या गोडीने-वेगाने} हा प्रकल्प अमलात आणता आला.
\section*{मराठी शिकणाऱ्या कोणालाही उपयोगी}
मराठी बोलू व समजू शकणाऱ्या आपल्या महाराष्ट्रीय लहान मुलाप्रमाणे प्रौढांनाही ही पद्धत उपयोगी पडू शकते. मराठीचा गंधही नसणाऱ्या एका छोट्या ७-८  वर्षांच्या मुलीने\textendash आलेशाने\textendash ७-८ दिवसांत एकूण ३-४ तासांच्या शिक्षणात १५-२० वाक्ये लिहिण्या-वाचण्याइतकी प्रगती केली होती. रोहिणीच्या या शिक्षण विशेषांकात या थोड्या वेगळ्या, धाडसी प्रयोगाची थोडक्यात सूत्ररूपाने माहिती देत आहे. अलिकडे शैक्षणिक प्रयोग अनेक जाणकार मंडळी यशस्वीपणे करत आहेत. अर्थात, प्रत्येकाच्या पद्धतीत थोडाफार फरक आहे. मला व्यक्तिशः सर्व पद्धतींची समग्र माहिती नाही. तरीही एक पालक म्हणून मला या कल्पना सुचल्या व मुलांना त्या आवडल्या, त्यांच्या पालकांना उपयोगी वाटल्या. त्यांची ही तोंडओळख. यावरच्या सर्व प्रकारच्या प्रतिक्रियांचे मनापासून स्वागत!	
\section*{चित्र व अक्षर यांची फारकत}
कोणत्याही पालकांना किंवा बालवाडीच्या ताईला विचारले तर पहिले उत्तर येते: \enquote*{क} कमळाचा, \enquote*{ब} बदकाचा, \enquote*{ग} गणपतीचा किंवा गवताचा! साधारणतः घरोघरीही \enquote*{अ} पासून \enquote*{ज्ञ} पर्यंत अक्षरे आणि त्या अद्याक्षरांनी सुरू होणाऱ्या शब्दांची चित्रे असणारा तक्ता पालक हौसेने आणतात. प्रत्यक्षात \enquote*{ब} या अक्षराच्या आकारापेक्षा बदकाचे चित्र जास्त आकर्षक असते. मूल तोंडाने \enquote{ब ब बदकाचा} म्हणत असले तरी, त्याचे लक्ष (असले  तर) ते बदक, त्याचा रंग, चोच, पाणी इत्यादींकडे असते. अनेकवेळा \enquote*{ब} हे अक्षर समोर असूनही त्याने पाहिलेलेसुद्धा नसते. अशा पद्धतीमुळे मुळात \enquote*{ब} हा जो उच्चार होतो त्याची खूण म्हणून असं अर्धवर्तुळ, त्यात तिरकी रेघ, व शेजारी चिकटून उभी रेघ असे \enquote*{ब} हे अक्षर येते हा महत्त्वाचा भागच निसटून जात असतो. \textbf{विशिष्ट उच्चारासाठी विशिष्ट आकाराचे चित्र} ही जी योजना भाषा शिक्षणात आहे ती लक्षात येणे महत्त्वाचे आहे. तशी एकदा  आणून दिली की मुले त्यादृष्टीने विचार करून अक्षरशः ढीगभर शब्द मागू लागतात. 

त्यात मग क बाक मध्ये येतो, वाकडी काकडी मध्ये येतो. चॉकलेट्, केक, टाक, ठीक अशा अनेक शब्दांत येतो. 

\begin{figure}
    \centering
    \includegraphics[scale=0.25]{images/a3}
    \caption{अभ्यास आणि व्यक्तिमत्त्व विकास ही दोन्ही \enquote*{पारडी} तुल्यबळ हवीत.}
\end{figure}

इथे चाकोरीतील झापडे लावण्याआधीच काढायची प्रक्रिया सुरू होते. घोकंपट्टीचा भाग जाऊन विचारप्रक्रियेला आणि कल्पनाशक्तीला वाव मिळतो. शिक्षणाची एक व्याख्या \textbf{शिकायला शिकणे} अशीही केली जाते. त्यादृष्टीने मूल मुळातच आपणहून उच्चार व त्या उच्चाराची खूण यांची सांगड घालते. आपल्या मराठीत \enquote*{च} आणि \enquote*{ज} हे महत्त्वाचे अपवाद सोडले तर जे म्हणतो तेच आपण लिहितो. उदा. \textbf{म रा ठी} ! (इंग्लिशमध्ये \textbf{बी ए टी} लिहितो आणि \textbf{बॅट} म्हणतो, वाचतो). मराठीच्या या उच्चारानुसारी देवनागरी लिपीमुळे अगदी सुरुवातीला थोडे अवघड वाटले तरी एकदा येऊ लागल्यावर मराठी सोपी वाटू शकते.

\section*{त्रिकोणी कार्डे--डोंगरासारखी}

साधारणतः २ इंच$\times$२ इंच पुठ्ठयाच्या चौरसाचे कर्णावर कापल्यास २ त्रिकोणी पुठ्ठयाचे तुकडे मिळतात. त्यातली लांब बाजू पाया व दोन सम बाजू बाजूला असा डोंगर मिळतो. 
\tikz \draw[line width=2pt,color=red] (0,0) -- (2.828, 0) -- (1.414, 1.414) -- cycle; 

या डोंगरासारख्या कार्डामुळे मूल अक्षरे सुलटच वाचते. शिवाय एका कार्डावर दोन्ही बाजूंना एकच अक्षर लिहायचे. अशी \enquote*{व} \enquote*{ब} \enquote*{क} यांची तीन कार्डे मूल बिनचूक वाचू लागले की त्याला \enquote*{वा}\enquote*{बा}\enquote*{का} चे शब्द उच्चारातील आकार चांगला लांबवून सांगायाचे व विचारायचेही: वाऽ गं वाऽ च! काका, बाक, वाक, रवा, खवा, टाका, वाका, तावा, बाबा, इ. इ. 

यावेळी \enquote*{काना} हा शब्द सांगणे मात्र आपण विसरून जायचे. \enquote{(कार्डाक्षरे दाखवत \dots) हा ... वा, हा ... बा, हा ... ब आणि हा ... बा, हा ... क आणि हा ... का} अशा रितीने मुलांना \enquote*{वा}, \enquote*{बा}, \enquote*{का} वाचता यायला लागते. तेही बिनचूक येईपर्यंत \enquote{बाबा}, \enquote{काका}, \enquote{बाक}, \enquote{वाक} असे शब्द वाचून घ्यायचे. या शब्दांना अर्थ असतो, त्यामुळे अर्थपूर्ण वाचनाचा आनंद मिळून मुलांना शिकण्याचा हुरूप वाटतो असा माझा अनुभव आहे.

\section*{२५ तासांचा हिशोब}
काहीतरी नवीन गोष्ट शिकताना त्यात ज्ञाताकडून अज्ञाताकडे टप्प्याटप्प्याने जाणे अभिप्रेत असते. लहान मुलांच्या बाबतीत तर त्यांचा  --- कालावधी दहा-एक मिनिटांचा असू शकतो. त्यामुळे एकेक नवीन धडा शिकवताना तो १५ मिनिटांपेक्षा जास्त असणार नाही हे मी पाहिले आहे. त्यामुळे घड्याळाच्या ६० मिनिटांच्या तासामध्ये मराठीचे ४ तास होतात. जेव्हा मराठी वाचायला शिकण्याच्या १०० बैठका होतात तेव्हा मिनिटांच्या हिशोबाने फक्त २५ तासच झालेले असतात. म्हणून २५ तासांत मूल मराठी शिकू शकते ही तशी अतिशयोक्ती म्हणता येणार नाही. अगदी १ दिवसाआड शिकवले तरी २०० दिवसांत म्हणजे जुलै ते डिसेंबर्-जानेवारीपर्यंत मूल वाक्ये, सोप्या गोष्टी उत्तम वाचू शकते. ४ वर्षे पूर्ण वयाची मोठ्या शिशुवर्गातली मुले ५ पूर्ण होऊन पहिलीत जातात. या वेळात बहुतेक शाळांमधून  \enquote*{अ} ते \enquote*{ज्ञ} पर्यंत मुळाक्षरे व काही कान्याचे शब्द शिकवले जातात. त्याऐवजी ही पद्धत\textendash शिंत्रे पद्धत म्हणूया\textendash वापरली तर मुले अधिक वेगाने व गोडीने वाचू लागलेली दिसतील (दिसतातही).

\section*{धडा तिसरा\textendash वेलांटी चा}
साधारणतः २ ते ३ दिवसांमध्ये मुले \enquote{व-वा}, \enquote{ब-बा}, \enquote*{क-का} ही अक्षरे बिनचूक वाचू शकतात. यानंतर वरीलप्रमाणे, \enquote*{ट}, \enquote*{ठ} ही दोन नवीन अक्षरे त्यांना शिकवता येतात. त्यात \enquote*{ट} हा \enquote{वाट}, \enquote{वीट}, \enquote{तीट}, \enquote{बोट}, \enquote{बावळट} यांतला तर \enquote*{ठ} हा \enquote{माठ}, \enquote{मीठ}, \enquote{ठक ठक}, अशा शब्दांतला असतो. 

त्याचवेळी मुले \enquote*{टा}, \enquote*{ठा} वाचू-ओळखू शकतात हे पाहून त्यांना \enquote*{टी}, \enquote*{ठी} सांगायची आहे. इथेही आपण \enquote{ठ, ठ ठशाचा}, \enquote{ठ ला वेलांटी ठी ठी} असे काही म्हणणार नाही. \enquote{हा ठ, हा ठा, ही ठी} असे (अक्षरे दाखवत) सांगणार त्यांचे शब्द सांगणार, विचारणार. 

त्याचवेळी, \enquote*{वी}, \enquote*{बी}, \enquote*{की} \textbf{बेमालूमपणे लिहून} (त्यांना \enquote*{व} \enquote*{ब} \enquote*{क} ठाऊक आहेच ना एव्हाना!) विचारणार, \enquote{हा ठ, ही ठी (वी दाखवत) तर ही?} \enquote*{वी} हे उत्तर त्यांच्याकडून येणे अपेक्षित आहे. तसे येण्यासाठी प्रयत्न करताना त्यांना काही सूचके (क्लूज) द्यावीत. 

आता पहा, शिकायला, शिकवायला सुरुवात केल्यापासून ५ व्या दिवशी आपल्याकडे  

\begin{table}[h!]
  \begin{center}
    \label{tab:दिवस-५}
    \begin{tabular}{ccc} % <-- Alignments: 1st column left, 2nd middle and 3rd right, with vertical lines in between
      व & वा & वी\\
      ब & बा & बी\\
      क & का & की\\
      ट & टा & टी\\
      ठ & ठा & ठी\\
    \end{tabular}
    %\caption{केवळ पाचव्या दिवसापर्यंतची प्रगती}
  \end{center}
\end{table}

इतकी अक्षरे वाचायची क्षमता येऊ शकते. त्यानंतर \enquote*{न}, \enquote*{म} ही अक्षरे. अर्थात त्याचवेळी \enquote{न ना नी} \enquote{म मा मी} येतेच. एकीकडे अशी एक वा दोन नवीन अक्षरे शिकत असताना येत असलेल्या अक्षरांच्या मदतीने \enquote*{काट}, \enquote*{वाट}, \enquote*{नीट}, \enquote*{वीट}, \enquote*{बीट}, \enquote*{ठाक}, \enquote*{ठीक} असे कित्येक शब्द 
\vspace{5mm}
\hrule
\end{document}
