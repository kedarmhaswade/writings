% Thanks to: https://tex.stackexchange.com/q/565173/64425
%\documentclass[17pt]{extarticle}
\documentclass[a6paper]{article}
\usepackage{geometry}
\geometry{
    bottom=30mm,
    left=0mm,
    right=5mm,
}
\usepackage{longtable}
\usepackage{polyglossia}
\setdefaultlanguage{marathi}
\setotherlanguages{marathi,english}
% Main font for Sanskrit: Sanskrit2003
\setmainfont[Mapping=velthuis-sanskrit,Script=Devanagari,Language=Default]{Sanskrit2003}
% Other font (for Marathi): Shobhika
\newfontfamily\marathifont[Mapping=velthuis-sanskrit,Script=Devanagari,Language=Default]{Sanskrit2003}
% Other well-designed fonts for devanagari: Sanskrit2003, Mukta, Yashomudra, Mortel

\begin{document}
\edef~{\string~}

    \begin{verse}
        ga.naadhii"sa jo ii"sa sarvaa.m gu.naa.mcaa|
        muLaara.mbha aara.mbha to nirgu.naacaa|\\

        namuu.m "saaradaa muuLa catvaara vaacaa|
        gamuu.m pa.mtha aana.mta yaa raaghavaacaa || 1 ||\\
        (la-gaa-gaa | la-gaa-gaa | la-gaa-gaa | la-gaa-gaa |)
    \end{verse}

    \begin{verse}
        manaa sajjanaa bhaktipa.mtheci jaave.m|
        tarii "sriiharii paavijeto svabhaave.m|\\

        janii.m ni.mdya te.m sarva so.dUni dyaave.m|
        janii.m va.mdya te.m sarva bhaave.m karaave.m || 2 ||\\
        (la-gaa-gaa | la-gaa-gaa | la-gaa-gaa | la-gaa-gaa |)
    \end{verse}

    \begin{verse}
        prabhaate manii.m raama ci.mtiita jaavaa|
        pu.dhe.m vaikharii.m raama aadhii.m vadaavaa|\\

        sadaacaara haa thora saa.m.duu.m naye to|
        janii.m toci to maanavii.m dhanya hoto || 3 ||\\
        (la-gaa-gaa | la-gaa-gaa | la-gaa-gaa | la-gaa-gaa |)
    \end{verse}

    \begin{verse}
        manaa vaasanaa du.s.ta kaamaa naye re|
        manaa sarvathaa paapabuddhii nako re|\\

        manaa dharmataa niiti so.duu.m nako ho|
        manaa a.mtarii.m saara viicaara raaho|| 4 ||\\
        (la-gaa-gaa | la-gaa-gaa | la-gaa-gaa | la-gaa-gaa |)
    \end{verse}
    \begin{verse}
        manaa paapas.mkalpa so.doni dyaavaa|
        manaa satyasa.mkalpa jiivii.m dharaavaa|\\

        manaa kalpanaa te nako vii.sayaa.mcii|
        vikaare.m gha.de ho janii.m sarva cii cii|| 5 ||\\
        (la-gaa-gaa | la-gaa-gaa | la-gaa-gaa | la-gaa-gaa |)
    \end{verse}
    \begin{verse}
        nako re manaa krodha haa khedakaarii|
        nako re manaa kaama naanaavikaarii|\\

        nako re \underline{madaa} sarvadaa a.mgikaaruu|
        nako re manaa matsaruu da.mbhabhaaruu|| 6 ||\\
        (la-gaa-gaa | la-gaa-gaa | la-gaa-gaa | la-gaa-gaa |)

        da.mbha = .dho.mga, madaa = garvaalaa
    \end{verse}

    \begin{verse}
        manaa "sre.s.Ta dhaari.s.ta jiivii.m dharaave|
        manaa bola.ne.m niica sosiita jaave.m|\\

        svaye.m sarvadaa namra vaace vadaave.m|
        manaa sarva lokaa.msi re niivavaave.m|| 7 ||\\
        (la-gaa-gaa | la-gaa-gaa | la-gaa-gaa | la-gaa-gaa |)

        niivavaave.m = samaadhaana dyaave
    \end{verse}

    \begin{verse}
        \underline{tanuu tyaa}gitaa.m kiirti maage.m uraavii|
        manaa sajjanaa heci kriiyaa dharaavii|\\

        manaa ca.mdanaace parii tvaa.m jhijaave.m|
        parii a.mtarii sajjanaa niivavaave.m|| 8 ||\\
        (la-gaa-gaa | la-gaa-gaa | la-gaa-gaa | la-gaa-gaa |)
    \end{verse}

    \vspace{2mm}
    \parbox{8cm}
    {
        \small
        {
            (
                mULa "slokaata ``\underline{dehe tyaa}gitaa.m kiirti maage.m uraavii'' as.m aahe.
                pa.na maga pahilii tiina ak.sara.m: de-he-tyaa --- gaa-gaa-gaa 
                a"sii hotaata aa.ni v.rttaata basata naahiita. mha.nuuna 
                (samarthaa.mcii k.samaa maaguuna) ``dehe'' 
                aivajii mii ``tanuu'' haa "sabda vaaparalaa aahe. aataa 
                ta-nuu-tyaa --- la-gaa-gaa hii tiina ak.sara.m v.rttaata basataata.
            )
        }
    }
    \begin{verse}
        nako re manaa dravya te puu.Dilaa.mce|
        atii svaartha buddhii na re paapa saa.mce|\\

        gha.de bhoga.ne.m paapa te.m karma kho.te.m|
        na hotaa.m manaasaarikhe.m du.hkha mo.Te.m|| 9 ||\\
        (la-gaa-gaa | la-gaa-gaa | la-gaa-gaa | la-gaa-gaa |)
    \end{verse}

    \begin{verse}
        sadaa sarvadaa priiti raamii.m dharaavii.m|
        dukhaacii svaye.m saa.m.di jiivii.m karaavii|\\

        tanuudu.hkha te.m suukha maaniita jaave.m|
        viveke.m sadaa sasvarupii.m bharaave.m|| 10 ||\\
        (la-gaa-gaa | la-gaa-gaa | la-gaa-gaa | la-gaa-gaa |)
    \end{verse}

    \begin{verse}
        janii.m sarvasuukhii asaa ko.na aahe?
        vicaare.m manaa tuu.mci "sodhuuni paahe|\\

        manaa tvaa.m ci re puurvasa.mciita kele.m|
        tayaasaarikhe.m bhoga.ne.m praapta jhaale.m|| 11 ||\\
        (la-gaa-gaa | la-gaa-gaa | la-gaa-gaa | la-gaa-gaa |)
    \end{verse}
    \begin{verse}
        manaa maanasii.m du.hkha aa.nuu.m nako re|
        manaa sarvathaa "soka ci.mtaa nako re|\\

        viveke.m tanuubuddhi so.duuna dyaavii|
        videhiipa.ne.m mukti bhogiita jaavii|| 11 ||\\
        (la-gaa-gaa | la-gaa-gaa | la-gaa-gaa | la-gaa-gaa |)
    \end{verse}
    \begin{verse}
        manaa saa.mga paa.m raava.naa kaaya jaale.m|
        akasmaata te.m raajya sarvai bu.daale.m|\\

        mha.nonii ku.dii vaasanaa saa.m.di vegii.m|
        baLe.m laagalaa kaaLa haa paa.thilaagii.m||12||\\
        (la-gaa-gaa | la-gaa-gaa | la-gaa-gaa | la-gaa-gaa |)

        paa.m = v.rttaata basa.nyaasaa.thii vaaparalelaa ekaak.sarii "sabda -- tyaalaa tasaa kaahii artha naahii.\\
        ku.dii = "sariiraatuuna
    \end{verse}
    \begin{verse}
        jivaa karmayoge.m janii.m janma jaalaa|
        parii seva.tii.m kaaLamuukhii.m nimaalaa|\\

        mahaa.m thora te m.rtyapanthe.mci gele|
        kitiiyeka te janmale aa.ni mele ||12||\\
        (la-gaa-gaa | la-gaa-gaa | la-gaa-gaa | la-gaa-gaa |)

        nimaalaa = melaa \\
    \end{verse}

\begin{tabular}{|l|l|}
\hline
krama & "sabda\\
\hline
1 & ga.naa \\
\hline
2 & bhaktii \\
\hline
3 & prabhaate \\
\hline
4 & manaa-4 \\
\hline
5 & s.mkalpa \\
\hline
6 & krodha-kaama-mada-matsara \\
\hline
7 & "sre.s.Ta-dhaari.s.ta \\
\hline
8 & tanuu ca.mdana\\
\hline
9 & dravya-svaartha-bhoga.ne.m-du.hkha\\
\hline
10 & raamii.m-saa.m.di-tanuudu.hkha-viveke.m\\
\hline
11 & sarvasuukhii-vicaare-puurvasa.mciita\\
\hline
12 & maanasii-"soka-tanuubuddhi-mukti\\
\hline
13 & saa.mgapaa.m-akasmaata-mha.nonii-baLe.m\\
\hline
14 & jivaa-parii-mahaa.m-kitiiyeka\\
\hline
\end{tabular}
        
\end{document}
