\documentclass[17pt]{extarticle}  % print quality
\usepackage{polyglossia}
\usepackage[margin=20mm]{geometry}
\setdefaultlanguage{marathi}
\setotherlanguages{english}
\usepackage{fourier}
%\textbf with "Sanskrit Text" produces the shape warning because the default Sanskrit Text font on Windows does not have a bold equivalent; try Yashomudra or Shobhika; now I like Tiro better
\newfontfamily\devanagarifont[Script=Devanagari]{Sanskrit Text}
%\newfontfamily\devanagarifont[Script=Devanagari]{Tiro Devanagari Marathi}

%\pagenumbering{gobble}
\setlength{\parindent}{0pt}% Remove paragraph indent
\usepackage[skip=\medskipamount]{parskip}
\usepackage[style=english]{csquotes}
\usepackage{verse}
\usepackage{xcolor}
\usepackage{hyperref}
\hypersetup{
    colorlinks=true,
    linkcolor=blue,
    filecolor=magenta,
    citecolor=blue,
    urlcolor=purple,
}

% control hyphenation: https://tex.stackexchange.com/a/177179/64425
\tolerance=1
\emergencystretch=\maxdimen
\hyphenpenalty=10000
\hbadness=10000
% control hyphenation: https://tex.stackexchange.com/a/177179/64425

\begin{document}

\title{युगंधरांचे पालुपद}
\author{कवी यशवंत}
%\textenglish {

%This document is a sample document to test font families and font typefaces.
%}

\date{}
\maketitle
\hrule
\vspace{5mm}

घडी बसाया हवी जगाची -- \\
नवीन सुंदर शांत निरामय\\
सिंधु आटले किती शाईचे\\
बोरु मेरु मांदारांचे\\
तासुनि तासुनि गेले चिंबुनि\\
फेसही आला ओठांभवती\\
परंतु त्यांचे एक पालुपद जमेचना, घडेचना, बनेचना, पटेचना!\\

म्हणून निघाले जाहिरातनामे --\\
पायावरती जे उभे राहती\\
कवायतीला ते यापुढती\\
बांधा ... उभे-आडवे\\
छातीभवती पिंडऱ्यांभवती खांद्यांभवती!\\
कवायतीतील परि बुटांच्या एकमुखाने वदती टाचा\\
त्यांची वाचा ऐका ऐका लिहा नि वाचा!\\
अहो बंदुका रोखून धरणें\\
हुकमासरशी चाप ओढणें\\
किंवा वरूनि आग ओतणें\\
ही भ्याडपणाची लपवालपवी!\\
स्वतः पराजित दुजास शिकवी\\
शस्त्रे म्हणजे वीरभूषणे\\
रणांत मरणें म्हणजे जगणें\\
शुद्ध ढोंग हे\\
एकच यामधे अर्थ खरा पण\\
शूर व्हायचे जमेचना, घडेचना, बनेचना, पटेचना!\\

अहो निघाला शिष्टाईला\\
अवतारी परमात्मा साक्षात!\\
फलश्रुति पण अगा सुइच्या अग्रावरती\\
राहणे जितुके शक्य धरित्री\\
मिळेल तितुकीही नापरती\\
याही चुकिच्या चला शक्यतर मागे जाऊ\\
चुका परंतु भुतासारख्या 

                       
\vspace{5mm}
\hrule

\end{document}
