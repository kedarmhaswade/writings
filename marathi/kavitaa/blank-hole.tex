\documentclass[17pt]{extarticle}  % print quality
\usepackage{polyglossia}
\usepackage[margin=20mm]{geometry}
\setdefaultlanguage{marathi}
\setotherlanguages{english}
\usepackage{fourier}
%\textbf with "Sanskrit Text" produces the shape warning because the default Sanskrit Text font on Windows does not have a bold equivalent; try Yashomudra or Shobhika; now I like Tiro better
\newfontfamily\devanagarifont[Script=Devanagari]{Sanskrit Text}
%\newfontfamily\devanagarifont[Script=Devanagari]{Tiro Devanagari Marathi}

%\pagenumbering{gobble}
\setlength{\parindent}{0pt}% Remove paragraph indent
\usepackage[skip=\medskipamount]{parskip}
\usepackage{fontawesome}
\usepackage[style=english]{csquotes}
\usepackage{verse}
\usepackage{xcolor}
\usepackage{hyperref}
\hypersetup{
    colorlinks=true,
    linkcolor=blue,
    filecolor=magenta,
    citecolor=blue,
    urlcolor=purple,
}

% control hyphenation: https://tex.stackexchange.com/a/177179/64425
\tolerance=1
\emergencystretch=\maxdimen
\hyphenpenalty=10000
\hbadness=10000
% control hyphenation: https://tex.stackexchange.com/a/177179/64425

\begin{document}

\title{ब्लॅंक होल}
\author{निरंजन पेडणेकर}
\date{}
\maketitle
\hrule
\vspace{5mm}

%\textenglish {

%This document is a sample document to test font families and font typefaces.
%}

कुट्ट पडद्याआड कोणी थाटला संसार आहे\\
का अनंताच्या तळाशी साचला अंधार आहे?

काळजाच्या कोपऱ्याची ही छबी विस्तारली का\\
काळ ज्याच्या उंबऱ्याशी थांबला ते दार आहे?

का प्रकाशाच्या अघोरी कृष्णकृत्यांच्या भयाने\\
स्वाभिमानी तारकांनी मांडला जोहार आहे?

का अकर्त्याच्या अभागी आंधळ्या डोळ्यात गोळा\\
आर्त काळ्या आसवांनी छेडला मल्हार आहे?

षड्ज-पंचम विलय-उत्पत्तीतले झंकारता का\\
बंद ओठांशी स्वयंभू दाटला गंधार आहे? 

गुंतणारे ह्यात सारे आपुले हरवून जाती\\
हा तुझ्या बेबंद केसांचा खुला संभार आहे!

\vspace{5mm}
\hrule
\end{document}
