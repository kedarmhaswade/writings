\documentclass[a6paper]{article}
\usepackage{polyglossia}
\usepackage[margin=10mm]{geometry}
\setdefaultlanguage{sanskrit}
\setotherlanguages{english, marathi}
\newfontfamily\devanagarifont[Script=Devanagari]{Sanskrit Text}
\newfontfamily\englishfont{Gentium Book Basic}
\setmainfont{Sanskrit Text}
%\newfontfamily\sanskritfont[Mapping=velthuis-sanskrit,Script=Devanagari,Language=Sanskrit]{Noto Serif Devanagari}
% make sure ~ as non-breaking space doesn't interfere with velthuis-sanskrit mapping
\edef~{\string~}
\pagenumbering{gobble}
\setlength{\parindent}{0pt}% Remove paragraph indent
\usepackage[skip=\medskipamount]{parskip}
\usepackage{xcolor}
\usepackage{hyperref}
\hypersetup{
    colorlinks=true,
    linkcolor=blue,
    filecolor=magenta,
    citecolor=blue,
    urlcolor=purple,
}
\newcommand \eng[1]{
    \textenglish{#1}
}
\begin{document}

{\Large बाभळी}

%TC:ignore
%TC:endignore

...

कुसर कलाकृती अशी बाभळी तिला न ठावी नागररिती\\
दूर कुठेतरी बांधावरती झुकून जराशी उभी एकटी\\
अंगावरती खेळवी राघू लाघट शेळ्या पायाजवळी\\ 
बाळगुराखी होउनिया मन रमते तेेेथे सांजसकाळी\\
येते परतून नवेच होऊन लेवून हिरवे नाजुक लेणे\\
अंगावरती माखुन अवघ्या धुंद सुवासिक पिवळे उटणे!\\
...


{\small -- इंदिरा संत}
%TC:ignore
%TC:endignore
\end{document}
