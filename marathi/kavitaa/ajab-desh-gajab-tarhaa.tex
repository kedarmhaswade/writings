\documentclass[17pt]{extarticle}  % print quality
\usepackage{polyglossia}
\usepackage[margin=20mm]{geometry}
\setdefaultlanguage{marathi}
\setotherlanguages{english}
\usepackage{fourier}
%\textbf with "Sanskrit Text" produces the shape warning because the default Sanskrit Text font on Windows does not have a bold equivalent; try Yashomudra or Shobhika; now I like Tiro better
\newfontfamily\devanagarifont[Script=Devanagari]{Sanskrit Text}
%\newfontfamily\devanagarifont[Script=Devanagari]{Tiro Devanagari Marathi}

%\pagenumbering{gobble}
\setlength{\parindent}{0pt}% Remove paragraph indent
\usepackage[skip=\medskipamount]{parskip}
\usepackage{fontawesome}
\usepackage[style=english]{csquotes}
\usepackage{verse}
\usepackage{xcolor}
\usepackage{hyperref}
\hypersetup{
    colorlinks=true,
    linkcolor=blue,
    filecolor=magenta,
    citecolor=blue,
    urlcolor=purple,
}

% control hyphenation: https://tex.stackexchange.com/a/177179/64425
\tolerance=1
\emergencystretch=\maxdimen
\hyphenpenalty=10000
\hbadness=10000
% control hyphenation: https://tex.stackexchange.com/a/177179/64425

\begin{document}

\title{अजब देश, गजब तऱ्हा}
\author{विंदा करंदीकर}
\date{}
\maketitle
\hrule
\vspace{5mm}

%\textenglish {

%This document is a sample document to test font families and font typefaces.
%}

अलाण्याच्या ब्रशावरती, फलाण्याचं दंतमंजन\\
अलबत्याचं अंडरवेअर, गलबत्याचं थोबाडरंजन 

लिलीसारख्या त्वचेसाठी, लिलीब्रॅंड साबणजेली\\
चरबी हटवा, वजन घटवा, हजारोंनी खात्री केली

केंटुकीच्या कोंबडीवरती, फेंटुकीचं मोहरीचाटण\\
डबलडेकर सँडविचमध्ये, बबलछाप मटणघाटण

पिझ्झाहट चा पिझ्झा खा, मेपलज्यूस, अॅप्पलपाय\\
अमुक डोनट, तमुक पीनट, अजून खाल्ले नाहीत काय?

अ ब क ड इ फ ग  . . .  तयार घरे, सात टाइप्स\\
रेडिमेड खिडक्या-दारे, भिंती-छपरे, गटरपाइप्स 

हॉटेल, मोटेल, खेटर, मोटर, देशभर करा प्रवास\\
एकच रूप, एकच रंग, एकच रुचि, एकच वास 

आकाशवाणी घटवत असते, दूरदर्शन पटवत असते\\
जहांबाज जाहिरातबाजी गिऱ्हाइकांना गटवत असते 

अजब देश! गजब तऱ्हा! व्यक्तीवरती सक्ती नाही\\
सहस्रशीर्ष मनुजा, तुझी गणवेशातून मुक्ती नाही!

\vspace{5mm}
\hrule
\end{document}
