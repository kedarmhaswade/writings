\documentclass[17pt]{extarticle}  % print quality
\usepackage{polyglossia}
\usepackage[margin=20mm]{geometry}
\setdefaultlanguage{marathi}
\setotherlanguages{english}
\usepackage{fourier}
%\textbf with "Sanskrit Text" produces the shape warning because the default Sanskrit Text font on Windows does not have a bold equivalent; try Yashomudra or Shobhika; now I like Tiro better
\newfontfamily\devanagarifont[Script=Devanagari]{Sanskrit Text}
%\newfontfamily\devanagarifont[Script=Devanagari]{Tiro Devanagari Marathi}

%\pagenumbering{gobble}
\setlength{\parindent}{0pt}% Remove paragraph indent
\usepackage[skip=\medskipamount]{parskip}
\usepackage[style=english]{csquotes}
\usepackage{verse}
\usepackage{xcolor}
\usepackage{hyperref}
\hypersetup{
    colorlinks=true,
    linkcolor=blue,
    filecolor=magenta,
    citecolor=blue,
    urlcolor=purple,
}

% control hyphenation: https://tex.stackexchange.com/a/177179/64425
\tolerance=1
\emergencystretch=\maxdimen
\hyphenpenalty=10000
\hbadness=10000
% control hyphenation: https://tex.stackexchange.com/a/177179/64425

\begin{document}

\title{दोन वाटा}
\author{कवी अनिल}
%\textenglish {

%This document is a sample document to test font families and font typefaces.
%}

\date{}
\maketitle
\hrule
\vspace{5mm}

वाटचुकार वासराशिवाय फिरकलेले कोणी नसते\\
अशा जागी गेली वाट सरत सरत आली असते\\
तिथे पुन्हा दोन वाटा फुटलेल्याच्या खुणा दिसतात\\
एक जरा सरावलेली दुसरीवरती ठसे नसतात\\
तिने चाहूल नाही म्हणून पाऊल पाऊल ओढून नेले\\
सोडून दिल्या वाटेकडे फिरून फिरून पाहू दिले\\
किती वेळ दिसत होती धरली नाही तीच वाट\\
तिची वळणे हेलकावे तिचे उतार तिचे घाट\\
तिनेच गेलो असतो तर ही कदाचित मिटली असती\\
तसे व्हावयाचे नव्हते - हीच माझी वाट होती!\\
                       
\vspace{5mm}
\hrule

\end{document}
