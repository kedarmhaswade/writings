\documentclass[17pt]{extarticle}  % print quality
\usepackage{polyglossia}
\usepackage[margin=20mm]{geometry}
\setdefaultlanguage{marathi}
\setotherlanguages{english}
\usepackage{fourier}
%\textbf with "Sanskrit Text" produces the shape warning because the default Sanskrit Text font on Windows does not have a bold equivalent; try Yashomudra or Shobhika; now I like Tiro better
\newfontfamily\devanagarifont[Script=Devanagari]{Sanskrit Text}
%\newfontfamily\devanagarifont[Script=Devanagari]{Tiro Devanagari Marathi}

%\pagenumbering{gobble}
\setlength{\parindent}{0pt}% Remove paragraph indent
\usepackage[skip=\medskipamount]{parskip}
\usepackage{fontawesome}
\usepackage[style=english]{csquotes}
\usepackage{verse}
\usepackage{xcolor}
\usepackage{hyperref}
\hypersetup{
    colorlinks=true,
    linkcolor=blue,
    filecolor=magenta,
    citecolor=blue,
    urlcolor=purple,
}

% control hyphenation: https://tex.stackexchange.com/a/177179/64425
\tolerance=1
\emergencystretch=\maxdimen
\hyphenpenalty=10000
\hbadness=10000
% control hyphenation: https://tex.stackexchange.com/a/177179/64425

\begin{document}

\title{जाईन दूर गावा}
\author{आरती प्रभू, नक्षत्रांचे देणे}
\date{}
\maketitle
\hrule
\vspace{5mm}

%\textenglish {

%This document is a sample document to test font families and font typefaces.
%}

तेव्हा मलाच माझा, वाटेल फक्त हेवा,\\
घरदार टाकुनी मी, जाईन दूर गावा.  

पाण्यांत ओंजळीच्या, आला चुकून मीन, \\
चमकून हो तसाच, गाण्यांत अर्थ जावा. 

शिंपीत पावसाळी, सर्वत्र या लकेरी,\\
यावा अनाम पक्षी, स्पर्शू मलाच यावा. 

देतां कुणी दुरून, नक्षत्रसे इशारे\\
साराच आसमंत, घननीळ होत जावा. 

पेरून जात वाळा, अंगावरी कुणी जो\\
शेल्यापरी कुसुंबी, वाऱ्यावरी वहावा. 

तारांवरी पडावा, केव्हा चुकून हात :\\
विस्तीर्ण पोकळीचा, गंधार सापडावा.  

तेव्हा मलाच माझा, वाटेल फक्त हेवा,\\
घरदार टाकुनी मी, जाईन दूर गावा.  

\vspace{5mm}
\hrule
\end{document}
