% $Header$

\documentclass{beamer}

% This file is a solution template for:

% - Giving a talk on some subject.
% - The talk is between 15min and 45min long.
% - Style is ornate.



% Copyright 2004 by Till Tantau <tantau@users.sourceforge.net>.
%
% In principle, this file can be redistributed and/or modified under
% the terms of the GNU Public License, version 2.
%
% However, this file is supposed to be a template to be modified
% for your own needs. For this reason, if you use this file as a
% template and not specifically distribute it as part of a another
% package/program, I grant the extra permission to freely copy and
% modify this file as you see fit and even to delete this copyright
% notice. 


\mode<presentation>
{
  %\usetheme{Warsaw}
  %\usetheme{Berlin}
  %\usetheme{metropolis}
  \usetheme{PaloAlto}
  %\usetheme{CambridgeUS}
  %\usetheme{Berkeley}
  
  % font themes
  \usefonttheme{serif}
  %\usefonttheme{structurebold}

  % color themes
  %\usecolortheme{albatross}
  %\usecolortheme{beetle}
  %\usecolortheme{crane}
  %\usecolortheme{dove}
  %\usecolortheme{fly}
  %\usecolortheme{monarca}
  %\usecolortheme{seagull}
  %\usecolortheme{beaver}
  % color themes
  % or ...

  %\setbeamercovered{transparent}
  \setbeamercovered{transparent=5}
  % or whatever (possibly just delete it)
}


\usepackage[english]{babel}
% Place figures exactly where you mean to
%https://tex.stackexchange.com/a/8633/64425
\usepackage{float}
% Place figures exactly where you mean to
% or whatever

\usepackage[utf8]{inputenc}
% or whatever

\usepackage[T1]{fontenc}
% Or whatever. Note that the encoding and the font should match. If T1
% does not look nice, try deleting the line with the fontenc.

% math
\usepackage{amsmath}
\usepackage{amssymb}
\usepackage{amsthm}
% math

% tikz
\usepackage{tikz}
\usetikzlibrary{arrows.meta}
\usetikzlibrary{calc}
% tikz
\usepackage{caption}
\usepackage{hyperref}
\hypersetup{
    colorlinks,
    linkcolor={magenta!50!black},
    citecolor={blue!50!black},
    urlcolor={blue!80!black}
}
% large commented sections
\usepackage{comment}
% large commented sections

% some beautiful fonts
\usepackage{yfonts}
% some beautiful fonts

% canceling in equations
\usepackage{cancel}
% canceling in equations

\title[The Great Gatsby] % (optional, use only with long paper titles)
{The Great Gatsby}

\subtitle
{On a Famous American Novel from the Jazz Age} % (optional)

\author[FSF,KM] % (optional, use only with lots of authors)
{F. Scott Fitzgerald\inst{1}}
% - Use the \inst{?} command only if the authors have different
%   affiliation.

\institute[Unknown] % (optional, but mostly needed)
{
  \inst{1}%
  Original Author
}
% - Use the \inst command only if there are several affiliations.
% - Keep it simple, no one is interested in your street address.

\date[September 2025] % (optional)
{September 2025 / Free Learner's School Conversations}

\subject{Fun Conversations at Home School}
% This is only inserted into the PDF information catalog. Can be left
% out. 



% If you have a file called "university-logo-filename.xxx", where xxx
% is a graphic format that can be processed by latex or pdflatex,
% resp., then you can add a logo as follows:

% \pgfdeclareimage[height=0.5cm]{university-logo}{university-logo-filename}
% \logo{\pgfuseimage{university-logo}}



% Delete this, if you do not want the table of contents to pop up at
% the beginning of each subsection:
\AtBeginSubsection[]
{
  \begin{frame}<beamer>{Outline}
    \tableofcontents[currentsection,currentsubsection]
  \end{frame}
}


% If you wish to uncover everything in a step-wise fashion, uncomment
% the following command: 

%\beamerdefaultoverlayspecification{<+->}


\begin{document}

\begin{frame}
  \titlepage
\end{frame}

\begin{frame}{Outline}
  \tableofcontents
  % You might wish to add the option [pausesections]
\end{frame}


% Since this a solution template for a generic talk, very little can
% be said about how it should be structured. However, the talk length
% of between 15min and 45min and the theme suggest that you stick to
% the following rules:  

% - Exactly two or three sections (other than the summary).
% - At *most* three subsections per section.
% - Talk about 30s to 2min per frame. So there should be between about
%   15 and 30 frames, all told.

% [Kedar] I am keeping this structure, but abusing it to serve my purpose.
% [Kedar] I expect this `presentation' to have many hundred slides, if I end up doing it right.
% [Kedar] There are three sections per presentation, no subsections, but each section may have many, many slides. Let's see. I am just getting started with LaTeX and Beamer.
% [Kedar] I may roughly make a chapter in his book a section in this presentation.
\section{Introduction}
\begin{frame}
\frametitle{Introductory Calculus--Infinitesimal Approach}
\framesubtitle{Introducing Infinitesimals}
\label{slide:intro-01}
\begin{itemize}
\item These Are My Notes from H. Jerome Keisler's Book by The Same Name.
\pause\item While Teaching Elementary Calculus to My Daughter, I Realized That I Better Learn The Infinitesimal Approach Better.
\end{itemize}
\end{frame}

\section{Chapter 1}
\begin{frame}
\frametitle{Chapter 01: 01}
\framesubtitle{Introductions}
\label{slide:chapter-01-01}
\begin{itemize}
\item ``Whenever You Feel Like Criticizing Any One'', He Told Me, ``Just Remember That All The People in This World Haven't Had The Advantages That You've Had.''
\begin{itemize}
\pause
\item Nick Carraway Starts The Novel as A First-person Narrative. \alert{Gatsby} Is Introduced Soon. 
\end{itemize}
\pause
\item \dots In Consequence, I'm Inclined to Reserve All Judgments, \dots
\pause
\item \dots Reserving Judgments Is A Matter of \alert{Infinite Hope}. I Am Still A Little Afraid of Missing Something If I Forget That, \dots
\pause
\item And, After Boasting \alert{This Way of My Tolerance}, I Came to The Admission That It Has A Limit.
\end{itemize}
\end{frame}

\begin{frame}
\frametitle{Chapter 01: 02}
\framesubtitle{Introductions}
\label{slide:chapter-01-02}
\begin{itemize}
\item 
If Personality Is An Unbroken Series of Successful Gestures, Then There Was Something Gorgeous about Him [Gatsby], Some Heightened Sensitivity to The Promises of Life, as If He Were Related to One of Those Machines That Registered Earthquakes Ten Thousand Miles Away \dots
\begin{itemize}
\pause \item Unusual Way to Introduce a Protagonist! He Was \alert {That Sensitive}!
\end{itemize}
\pause
\item \dots The Actual Founder of My Line Was My Grandfather's Brother, Who Came Here in [18]Fifty-one, Sent A Substitute to The Civil War \dots
\begin{itemize}
\pause \item During The American Civil War, A Provision Known as \alert{Substitution Allowed Men Drafted into Military Service to Pay Someone Else to Serve in Their Place}.
\pause \item Nick Looked Like This Great-Uncle of His.
\end{itemize}
\end{itemize}
\end{frame}

\begin{frame}
\frametitle{Chapter 01: 03}
\framesubtitle{Introductions}
\label{slide:chapter-01-03}
\begin{itemize}
\item I Participated in That \alert{Delayed Teutonic Migration}.
\begin{itemize}
\pause
\item
\dots a Sarcastic Allusion Referring to The \alert{Great War (World War I)} as a Belated and More Peaceful Version of Ancient German (Teutonic) Migrations into Europe.
\end{itemize}
\pause
\item A \alert{Weatherbeaten Cardboard Bungalow} at Eighty A Month \dots
\begin{itemize}
\pause \item Shows A Sharp Economic Contrast w His Rich Neighbors \dots
\end{itemize}
\item \dots And A Finnish Woman, Who Made My Bed And Cooked Breakfast And \alert{Muttered Finnish Wisdom} to Herself Over The Electric Stove.
\pause \item I Lived at \alert{West Egg}, The--Well, The Less Fashionable of The Two, \dots
\pause \item So I Had A View of The Water, \dots And \alert{The Consoling Proximity of Millionaires}--All For Eighty Dollars A Month.
\end{itemize}
\end{frame}

\begin{frame}
\frametitle{Chapter 01: 04}
\framesubtitle{Introductions}
\label{slide:chapter-01-04}
\begin{itemize}
\item On [Tom Buchanan]\dots Who Reach Such An \alert{Acute Limited Excellence} That \textit{Everything After} Savors of \alert{Anticlimax}.
\begin{itemize}
\pause
\item This Is A Satirical Remark. Tom Clearly Represents The [Less Intellectual, And] Traditionally Wealthy Family Member (\alert{The East Egg}) from The Mid-west Who Ventured to The East, A Land of The Neo-Rich (\alert{The West Egg}).
\end{itemize}
\pause \item She Was Extended Full Length at Her End of The Divan \dots
\pause \item The Other Girl, Daisy, \dots \alert{``I'm P-paralyzed with Happiness''}.
\begin{itemize}
\pause \item Tom, Daisy, Miss Baker--The \alert{East Eggs}.
\end{itemize}
\pause\item The Object She [Baker] Was Balancing Had Obviously Tottered{\footnote{Move, Walk Unsteadily.}}\dots Almost \alert{Any Exhibition of Complete Self-sufficiency Draws a Stunned Tribute from Me}.
\end{itemize}
\end{frame}

\begin{frame}
\frametitle{Chapter 01: 05}
\framesubtitle{Things Began to Heat up in A Family Party of Strangers}
\label{slide:chapter-01-05}
\begin{itemize}
\item Well, [Daisy, While Talking to Nick,] She [Daisy-Tom's Daughter] Was Less Than An Hour Old And Tom Was God Knows Where. [After Knowing She Delivered A Baby Girl] \dots -- That's The Best Thing a Girl Can Be in This World, \alert{A Beautiful Little Fool.}
\begin{itemize}
\pause \item Daisy Seems to Think That The World Is Brutal for Women!
\end{itemize}
\pause \item (Tom and Miss Baker Are Together) The Lamp-light, Bright on His Boots And Dull on The Autumn-Leaf Yellow of Her Hair, Glinted Along The Paper as She Turned A Page With a Flutter\footnote{Quick Movement, Vibration.} of Slender Muscles in Her Arms. 
\pause \item (Nick Is About to Leave after A \textit{Strange} Party) As I Started My Motor Daisy Peremptorily\footnote{Decisively, Imperatively.} Called: ``Wait!''
\end{itemize}
\end{frame}

\begin{frame}
\frametitle{Chapter 01: 06}
\framesubtitle{In The \alert{Unquiet} Darkness}
\label{slide:chapter-01-05}
\begin{itemize}
\pause \item When I looked Once More for Gatsby He Had Vanished, And I was Alone Again in \alert{The Unquiet Darkness}.
\begin{itemize}
\pause \item This is Fitzgerald's \alert{Evocative Writing Style}. Though This Atmosphere is Alien to Me And I Am Ambivalent about Huge Mansions and Wealthy Lifestyles, I Feel Like Reading On.
\pause \item We Are All a Little Strange, But These Characters, Tom, Daisy, Jordan Feel Even Stranger. One (Who's Not Wealthy And Isn't Attracted to Wealth) Cannot Escape a Feeling of Contempt for the Ultra-rich.
\end{itemize}
\end{itemize}
\end{frame}

\section{Chapter 2}
\begin{frame}
\frametitle{Chapter 02: 01}
\framesubtitle{Strange Proceedings}
\label{slide:chapter-02-01}
\begin{itemize}
\pause \item (Things Are Taking A Sudden Turn. Tom Forces Nick to See \alert{His Girl in New York}!)
\begin{itemize}
\pause \item What Followed Was a Boisterous Party (That Turned Bloody) in A Small Apartment. Nick, Tom, Myrtle (Tom's Mistress), Mr. George Wilson (Myrtle's Apparently Lame Husband), Catherine (Myrtle's Sister), Mr. \& Mrs. McKee Were All Part of This Absolutely Strange Meeting. 
\end{itemize}
\pause\item \dots Then I Was Lying Half Asleep in The Cold Lower Level of The Pennsylvania Station, Staring at The Morning \textit{Tribune}, And Waiting for The Four O'clock Train.
\end{itemize}
\end{frame}

\section{Chapter 3}
\begin{frame}
\frametitle{Chapter 03: 01}
\framesubtitle{Gatsby's Great Party}
\label{slide:chapter-03-01}
\begin{itemize}
\item THERE Was Music from My Neighbor's House through The Summer Nights. In His Gardens \alert{Men And Girls Came And Went Like Moths} among The Whisperings And The Champagne And The Stars.
\pause\item By Seven O'clock The Orchestra Has Arrived, No Thin Five-Piece Affair, But Whole Pitful (?) of Oboes And Trombones And Saxophones And Viols And Cornets And Piccolos, And Low And High Drums.
\pause\item On a Chance We Tried \alert{an Important-looking Door}, And Walked into A High Gothic Library,\dots
\pause\item \dots He Had Just Bought A Hydroplane\footnote{A Plane That Lands on/Lifts off Water Surface.} \dots
\pause\item ``This Is An Unusual Party for Me. I Haven't Even Seen The Host. \dots ''
\pause\item ``\alert{I'm Gatsby},'' He Said Suddenly.
\end{itemize}
\end{frame}

\begin{frame}
\frametitle{Chapter 03: 02}
\framesubtitle{Gatsby's Great Party}
\label{slide:chapter-03-02}
\begin{itemize}
\item ``Anyhow, He Gives Large Parties,'' Said Jordan, \alert{Changing The Subject with An Urban Distaste for The Concrete}. ``And I Like Large Parties. They're So Intimate. At Small Parties There Isn't Any Privacy.''
\pause\item \dots Girls Were Putting Their Heads on Men's Shoulders in A Puppyish, Convivial\footnote{Occupied with The Pleasures of Good Company.} Way, Girls Were Swooning\footnote{Pass Out.} Backward Playfully into Men's Arms, \dots
\pause\item \dots \alert{No French Bob Touched Gatsby's Shoulder} \dots
\begin{itemize}
\pause\item ``French Bob'' Signifies Physical Connection; Gatsby Was Drawn from The Crowd.
\end{itemize}
\pause\item \dots There Was a Jauntiness\footnote{Stylishness, Dapperness.} about Her Movements \dots
\end{itemize}
\end{frame}

\begin{frame}
\frametitle{Chapter 03: 03}
\framesubtitle{Gatsby's Great Party Starts Becoming Unbearable in Parts \dots}
\label{slide:chapter-03-03}
\begin{itemize}
\item I Looked Around. Most of The Remaining Women Were Now Having Fights with Men Said to Be Their Husbands. \dots Resorted to Flank Attacks\footnote{Attack from Side, Rather Than Front.}--At Intervals She Appeared Suddenly \dots, And Hissed: ``You Promised!'' into His Ear.
\begin{itemize}
\pause\item How Sad And Absurd All This Feels \dots
\pause\item Amid This, Gatsby Gets \alert{Mysteriously Interested in Jordan} And Asks to See Her Alone.
\end{itemize}
\pause\item (And Then There's An Accident as Everyone Leaves The Party \dots) ``But I Wasn't Even Trying [to Drive],'' He Explained Indignantly\footnote{Angered by Something Wrong or Unjust.}, ``I Wasn't Even Trying.''
\end{itemize}
\end{frame}

\section{Summary}

\end{document}


