\documentclass[a6paper]{article}
\usepackage{polyglossia}
\usepackage[margin=5mm]{geometry}
\setdefaultlanguage{english}
\setmainfont{Noto Serif}
\begin{document}
\section{Introduction}
This collection is inspired by Dmitri Borgmann's entertaining book, \emph{Language on Vacation} \cite{lov}. We are all intrigued by the effects of a language, our most astonishing creation \cite{wap}. This is an attempt to capture such effects. The author of this article has called such effects \emph{languames}. A \emph{languame} is an entertaining game, with flexible rules, based on simple yet surprising effects of a spoken language.

\begin{thebibliography}{00}
    \bibitem{lov} Dmitri A. Borgmann. Language on Vacation: An Olio of Orthographical Oddities. New York: Charles Scribner's Sons.
    \bibitem{wap} Giles Lytton Strachey and George Rylands. Words and Poetry. New York: Payson and Clarke. Introduction, Page xii.
\end{thebibliography}
\end{document}
