\documentclass[a6paper]{article}
\usepackage{polyglossia}
\usepackage[skip=\medskipamount]{parskip}
\usepackage{lettrine}
\usepackage{fontawesome}
\usepackage{epigraph, varwidth}
\setlength{\epigraphwidth}{0.7\textwidth}
\usepackage{hyperref}
\hypersetup{
    backref=true,
    citecolor=magenta,
    colorlinks=true,
    linkcolor=blue,
    filecolor=magenta,
    urlcolor=cyan,
}
\urlstyle{same}
\usepackage{geometry}
\geometry {
    bottom=15mm,
    left=10mm,
    right=10mm,
}
\setlength{\parindent}{0pt}% Remove paragraph indent
\setdefaultlanguage{english}
\setotherlanguages{sanskrit, marathi}
%\setmainfont[Scale=0.9]{Noto Serif}
%\setmainfont[Language=English]{Gentium Book Basic}
%\setmainfont[Language=English,Scale=0.90]{Lato}
\setmainfont[Language=English,Scale=0.90]{IndUni-H-Regular}
\newfontfamily\sanskritfont[Mapping=velthuis-sanskrit,Script=Devanagari,Language=Sanskrit]{Noto Serif Devanagari}
\newfontfamily\marathifont[Mapping=velthuis-sanskrit,Script=Devanagari]{Noto Serif Devanagari}
% make sure ~ as non-breaking space doesn't interfere with velthuis-sanskrit mapping
\edef~{\string~}
\pagenumbering{gobble}


\begin{document}

\title{On My Pandemic Position}
\author{Kedar Mhaswade}
\date{July 2021\footnote{It has undergone a few revisions since its first draft.}}
\maketitle

\lettrine[lines=3]{T}{he pandemic} has hurt us all. At a personal level, and perhaps as one of its less severe effects, it has deeply challenged my rationality. I have struggled with many behavioral and some philosophical questions arising out of my pandemic life. Perhaps I am overthinking and my reasoning is faulty, but the problem is real. This results in an unfortunate misunderstanding, a loss of precious time, and perhaps raises the mental stress levels even when a stressed state of mind is not apparent. In such cases, an easier way out is to choose a set of axioms and stick to them like a robot would. This is a tough position to take for a caring human being and I am more likely to be misunderstood (or worse, branded snob or pseudo-intellectual) by even my near and dear ones, but I will have to take that risk. 

The axioms of choice may be faulty (this happens even in mathematics, a glaring example of which is our belief in the universality of the Euclidean Geometry for two millennia!) but they are not faulty given our ``level of awareness'' at the time we made or chose the axioms. We should, of course, change or ``upgrade'' our axioms as and when we feel a need. Typically the axioms are modified because we find solid evidence mandating it. In this delicate (because it involves both objective and subjective factors) case such evidence would comprise increased vaccination, drastically reduced new cases, no new virus variants etc. 

My current axioms are:

\begin{enumerate}

    \item Even if I am fully vaccinated, avoid any non-essential physical social interaction in closed spaces like restaurants, gathering halls, or malls where the crowding is usually not monitored actively. Restaurants are to be especially avoided because we must remove masks there and meeting that way is very uncomfortable anyway -- it is not fun at all. Meeting on a playground or a hilltop is okay. This is an axiom only because it can be so \emph{in my case}. I feel for people who have to take the risk of going out only to make a living. 

\item Meet people more often using technology. Sincerely call them, don't just text on WhatsApp even if I am showing signs of the so-called ``Zoom Fatigue'' and meeting this way is nothing compared to meeting in person. Here's a simple trick to dodge the dilemma: Pretend that they are in another town! Set up regular time on my calendar rather than doing this offhandedly. This is as important as 1). We often need a method to take care of things that are dear to us.

\item If physical separation suggested in 1) is not possible (e.g. I need to take my ailing mother to a doctor or a hospital) exercise all the care suggested by the WHO or the ICMR or both.

\item Do 1) and 2) till the percentage of fully vaccinated people in Pune reaches $\approx 60$. Do not challenge the ``out-of-thin-air'' number $\approx 60$. Consider it to be an axiom.

\item There is no fifth axiom \faSmileO.

\end{enumerate}

Believe me, since I am an extrovert person who naturally likes to physically meet people and talk to them, taking such a position more or less consistently has been psychologically hard for me. But I am going to stick to this position to avoid further turmoil and I need your help in not misunderstanding me (even if you strongly believe that I am wrong right now). If this were a barter, in return, I offer to have no misunderstanding about your pandemic position.

I hope I have written it down as well as it is clear to me. It also saves time because I have taken a lot of time in writing it down and as long as my position is the same, I can forward this essay to everyone! Being \emph{explicit} and \emph{assertive} (even if our position turns out to be wrong) is something that we should all get better at. Even after writing this down some people may think that this is a ``convenient position'' so that I can do what I want anyway and make this \emph{excuse} when confronted with the things that I do not want to do. To them, I can only offer the following quote:

\epigraph
{
    Do what you want and say what you feel, for, those who matter won't mind and those who mind don't matter.
}
{
    \textit{Dr. Seuss}
}

Thank you!

\end{document}
