\documentclass[20pt,a4paper]{article}
\usepackage{euler}
\usepackage{lettrine}
\usepackage{extsizes}
\usepackage{epigraph, varwidth}
\usepackage{polyglossia}
\usepackage{beton}
\usepackage[OT1]{fontenc}
\begin{document}
\setmainfont[Language=English]{Gentium Book Basic}
\renewcommand{\epigraphsize}{\small}
\setlength{\epigraphwidth}{1.0\textwidth}

\epigraph
{
    Vigorous writing is \emph{concise}. A sentence should contain no unnecessary words, a paragraph no unnecessary sentences, for the same reason that a drawing should have no unnecessary lines and a machine no unnecessary parts. This requires \emph{not} that the writer make all his sentences short, or that he avoid all detail and treat his subject only in outline, but that every word tell.
}
{
    ---\textit{E. B. White}
}
\epigraph
{
    Many matters, however, are less easily brought to the test of experience. If, like most of mankind, you have passionate convictions on many such matters, there are ways in which you can make yourself aware of your own bias. \emph{If an opinion contrary to your own makes you angry, that is a sign that you are subconsciously aware of having no good reason for thinking as you do}. If some one maintains that two and two are five, or that Iceland is on the equator, you feel pity rather than anger, unless you know so little of arithmetic or geography that his opinion shakes your own contrary conviction. The most savage controversies are those about matters as to which there is no good evidence either way. Persecution is used in theology, not in arithmetic, because in arithmetic there is knowledge, but in theology there is only opinion. So whenever you find yourself getting angry about a difference of opinion, be on your guard; you will probably find, on examination, that your belief is going beyond what the evidence warrants.
}
{
    ---\textit{Bertrand Russell, Unpopular Essays}
}
\end{document}
