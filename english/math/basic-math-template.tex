\documentclass[10pt]{article}         %% What type of document you're writing.

%%%%% Preamble

%% Packages to use

\usepackage{amsmath,amsfonts,amssymb}   %% AMS mathematics macros

%% Title Information.

\title{Basic \LaTeX{} Template}
\author{William J. Turner}
%% \date{2 July 2004}           %% By default, LaTeX uses the current date

%%%%% The Document

\begin{document}

\maketitle

\begin{abstract}
This paper computes the distance between two points and fits both linear and
exponential functions through the two points.
\end{abstract}

\section{Introduction}
Consider the two points $(-1,16)$ and $(3,1)$.  Section~\ref{sec: distance}
computes the distance between these two points.  Section~\ref{sec: linear fit}
computes a linear equation $y = m x + b$ through the two points, and
Section~\ref{sec: exponential fit} fits a exponential equation $y = A e^{k x}$
through the two points.

\section{Distance}
\label{sec: distance}
We can use the distance formula
\begin{equation}
\label{eqn: distance}
	d = \sqrt{(x_2 - x_1)^2 + (y_2 - y_1)^2}
\end{equation}
to determine the distance between any two points $(x_1, y_1)$ and $(x_2, y_2)$
in $\mathbb{R}^2$.  For our example, $(x_1, y_1) = (-1, 16)$ and $(x_2, y_2) =
(3, 1)$, so plugging these values into the distance formula~\eqref{eqn:
distance} tell us the distance between the two points is
$$
	d 
	= \sqrt{(3 - (-1))^2 + (1 - 16)^2}
	= \sqrt{4^2 + (-15)^2}
	= \sqrt{241}
	.
$$

\section{Linear Fit}
\label{sec: linear fit}
Consider a linear equation $y = m x + b$ through the two points.  We will
first determine the slope $m$ of the line in Section~\ref{sec: slope}, and we
will then determine the $y$-intercept $b$ of the line in Section~\ref{sec:
intercept}.

\subsection{Slope}
\label{sec: slope}

The slope of the line passing through the two points is given by the forumula
$$
	m 
	= \frac{\Delta y}{\Delta x} 
	= \frac{y_2 - y_1}{x_2 - x_1}
	.
$$
Plugging in our two points, we find the slope of the line between them is
\begin{equation}
\label{eqn: slope}
	m 
	= \frac{1 - 16}{3 - (-1)}
	= - \frac{15}{4}
	.
\end{equation}

\subsection{Intercept}
\label{sec: intercept}

To find the $y$-intercept of the line, we start with the point-slope form of
the line of slope $m$ through the point $(x_0, y_0)$:
$$
	y - y_0 = m (x - x_0)
	.
$$
We plug in the point $(x_0, y_0) = (-1, 16)$ and the slope we found
previously~\eqref{eqn: slope} to obtain the equation
$$
	y - 16 = - \frac{15}{4} (x + 1)
	.
$$
Solving for $y$, we find the slope-intercept form of the line:
\begin{align*}
	y 
	&= - \frac{15}{4} x - \frac{15}{4} + 16 \\
	&= - \frac{15}{4} x + \frac{49}{4}
	.
\end{align*}
Therefore, the $y$-intercept is $b = 49/4$, and the equation 
$y = - \frac{15}{4} x + \frac{49}{4}$ describes the line through the two
points.

\section{Exponential Fit}
\label{sec: exponential fit}

Let us consider the exponential function $y = A e^{k x}$.  For this function
to pass through both points, we must find constants $A$ and $k$ that satisfy
both equations $16 = A e^{-k}$ and $1 = A e^{3 k}$.  To solve these two
simultaneous equations, we first take the ratio of the two equations, which
gives us a single equation involving only $k$:
$$
	16
	= \frac{A e^{-k}}{A e^{3 k}}
	= e^{-4 k}
	.
$$
We can take the natural logarithm of this equation to solve for $k$:
$$
	-4k = \ln(16) = 4 \ln (2)
	,
$$
which means $k = - \ln(2)$.

We can then use this value of $k$, along with either of the two points to
solve for $A$.  Let us consider the point $(-1, 16)$:
$$
	16 = A e^{(-\ln(2))(-1)} = A e^{\ln{2}} = 2 A
	.
$$
Solving for $A$, we find $A = 8$, and the exponential equation through both
points is
$$
	y
	= 8 e^{-\ln(2) x}
	= 8 2^{-x}
	= 8 \left( \frac{1}{2} \right)^x
	.
$$

Here are examples of piecewise functions:

\begin{equation}
\chi_{\mathbb{Q}}(x)=
    \begin{cases}
        1 & \text{if } x \in \mathbb{Q}\\
        0 & \text{if } x \in \mathbb{R}\setminus\mathbb{Q}
    \end{cases}
\end{equation}

\begin{equation}
    C_{k} =
    \begin{cases}
        1 & \text{if } k = 1\\
        1 & \text{if } k = 2\\
        C_{k-1} + C_{k-2} & \text{otherwise}
    \end{cases}
\end{equation}
\end{document}

