\begin{frame}
\frametitle{Introductory Calculus--Infinitesimal Approach}
\framesubtitle{1.5 The Extension Principle: Creating Hyperreals, Dealing with Real, Hyperreal Functions}
\label{slide:1.5-05}
\begin{itemize}
\item We Start with \alert{The Extension Principle}, Which \alert{Gives Us Hyperreal Numbers}, And \alert{Extends All Real Functions to Them}.
\pause
\item $f:\mathbb{R}\rightarrow\mathbb{R}$ Defines a \alert{Real-Valued Function}. When We Say $f(a)=b$, $f$ \textit{relates} Some $a\in\mathbb{R}$ to Some $b\in\mathbb{R}$. $f$ May Fail to Relate some $a\in\mathbb{R}$ with Any Real Number. Then We Say $f(a)$ is \textit{Undefined}.
\pause\item Similarly, $F$, \alert{A Hyperreal Function}, \textit{relates} A Hyperreal Number $H$ with Another Hyperreal Number $K$, Or Is \textit{Undefined}. Symbolically, $F:\mathbb{R}^*\rightarrow\mathbb{R}^*$. We say $K=F(H)$.
\pause\item We Can, Like Real-Valued Functions, Easily Define Hyperreal Functions of More Than 1 Hyperreal Number.
\end{itemize}
\end{frame}
