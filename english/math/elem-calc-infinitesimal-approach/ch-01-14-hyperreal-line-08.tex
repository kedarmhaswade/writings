\begin{frame}
\frametitle{Introductory Calculus--Infinitesimal Approach}
\framesubtitle{1.4 Startling Observation about \alert{The Physical Space}}
\label{slide:1.4-08}
\begin{block}{The Nature of Physical Space}
Euclid Struggled to Define \alert{a `Point': Something That Has a Position But No Magnitude\footnote{Isn't This `Definition' (\textit{Accepted} for Centuries) \textit{Meaningless}?}}.

We Have No Way of Knowing What a Line in Physical Space Is Really Like (``What Is It Composed of?''). It Might Be Like the Hyperreal Line (with Infinitesimals Surrounding Every `Real Point'), The Real Line (without Any Infinitesimals), Or Neither. \alert{However, in Applications of The Calculus It Is Helpful to Imagine a Line in Physical Space as A Hyperreal (Rather Than Real) Line. The Hyperreal Line Is, Like The Real Line, A Useful \underline{Mathematical Model}\footnote{And, Here's A Stark Reminder: All Models Are Wrong; Some Are Useful!} for A Line in Physical Space.}
\end{block}
\end{frame}
