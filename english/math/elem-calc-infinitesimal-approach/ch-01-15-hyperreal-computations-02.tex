\begin{frame}
\frametitle{Introductory Calculus--Infinitesimal Approach}
\framesubtitle{1.5 Computations with Hyperreal Numbers}
\label{slide:1.5-02}
\begin{itemize}
\item In Our Quest, We Must Satisfactorily (Firmly, Rigorously) Answer Questions Such as:
\begin{itemize}
\pause\item $\Delta x$ Denotes An Infinitesimal. Do Familiar Functions (e.g. Polynomial, Rational, \dots) Map Infinitesimals on to Infinitesimals? Thus, If $\Delta x$ Infinitesimal, Is A Polynomial Function of Infinitesimals Alone, e.g. $(\Delta x)^2 -(\Delta x)$, Infinitesimal Too?
\pause\item How Do We Think of Functions That \alert{\textit{Combine} Infinitesimals with Reals}? For Example, \alert{Given That $\Delta x$ Is Infinitesimal, Is $\underline{2x\Delta x}+(\Delta x)^2$ Infinitesimal Too}?
\end{itemize}
\end{itemize}
\pause
Shouldn't We Get The Taste of The Pain That Robinson et.al. Had While Conceiving A New Kind of Number (Hyperreal) That Must Live Amicably with The Kind of Number (Real) That Already Existed And Was Widely Understood? 
\end{frame}
