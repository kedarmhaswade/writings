\begin{frame}
\frametitle{Introductory Calculus--Infinitesimal Approach}
\framesubtitle{1.4 Properties of Hyperreals And Infinitesimals}
\label{slide:1.4-06}
\begin{itemize}
\item $\mathbb{R}^*$: The Set of All Hyperreal Numbers.
\pause\item $x\in\mathbb{R}\implies x\in\mathbb{R}^*\implies\mathbb{R}\subseteq\mathbb{R}^*$. However, $\mathbb{R}^*$ Has Other Elements Too, \alert{$\therefore \mathbb{R}\subset\mathbb{R}^*$}.
\pause\item Infinitesimals in $\mathbb{R}^*$ Are of \alert{Three Kinds: Positive, Negative, And \underline{The Real Number $0$}}.
\pause\item $x,x_0,x_1,y,\dots$ Denote Reals. $\Delta x, \Delta y, \varepsilon\text{ (epsilon)},\delta\text{ (delta)},\dots$ Denote Infinitesimals.
\pause\item If $a,b\in\mathbb{R}^*$ And $a-b \text{ Is Infinitesimal}$, Then We Say \alert{$a$ Is ``Infinitely Close'' to $b$}.
\begin{itemize}
\pause\item If $\Delta x=(x_0+\Delta x)-(x_0)$ Is Infinitesimal, Then \alert{$x_0+\Delta x$ And $x_0$ Are ``Infinitely Close'' to Each Other}.
\end{itemize}
\pause \item If \alert{$\varepsilon$ Is Positive Infinitesimal}, Then 
\begin{itemize}
\pause\item\alert{$-\varepsilon$ Is Negative Infinitesimal}.
\pause\item\alert{$\frac{1}{\varepsilon}$ Is Infinite Positive, > All Real Numbers}.
\pause\item\alert{$\frac{-1}{\varepsilon}$ Is Infinite Negative, < All Real Numbers}.
\end{itemize}
\end{itemize}
\end{frame}
