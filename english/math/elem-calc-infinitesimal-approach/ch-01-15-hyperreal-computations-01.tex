\begin{frame}
\frametitle{Introductory Calculus--Infinitesimal Approach}
\framesubtitle{1.5 Computations with Hyperreal Numbers}
\label{slide:1.5-01}
\begin{itemize}
\item Our Choice of The Mathematical Model of A Line in The Physical Space, \alert{The Hyperreal Line}, Contains Points Representing Real And Hyperreal Numbers.
\item Surrounding Each Real Number $r\in\mathbb{R}$, There Are \alert{Hyperreal Numbers Infinitely Close to $r$}.
\item Hyperreal Numbers Infinitely Close to 0 Are Called Infinitesimals.
\item $0$ Is The Only \textit{Real} Infinitesimal Number. There Are Many Nonzero Hyperreal Infinitesimals.
\pause\item In This Section, \alert{We Describe Hyperreal Numbers More Precisely And Develop a Facility for Computation with Them}.
\pause\item We Must Undertake This Effort, Because, After All, We Took \alert{The Bold Step of Conceiving \textit{Hyperreal Number: A New Kind of Number}}; We Must Follow Through!
\end{itemize}
\end{frame}
