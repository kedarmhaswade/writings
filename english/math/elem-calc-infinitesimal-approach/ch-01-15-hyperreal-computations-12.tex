\begin{frame}
\frametitle{Introductory Calculus--Infinitesimal Approach}
\framesubtitle{1.5 Using Extension And Transfer Principles for Hyperreal Computations}
\label{slide:1.5-12}
We Now Logically Proceed to Doing Computations Involving Hyperreal Numbers.
\begin{itemize}
\item The Extension Principle Tells Us That At Least One Positive Infinitesimal, $\varepsilon$, Exists. By Definition, $\varepsilon$ Is \textit{Infinitely Close to 0}.
\pause\item Readily, The Transfer Principle Suggests $0<\varepsilon^2<\varepsilon$ (Follows from the `Real Statement': $0<r^2<r\;\forall r\in\mathbb{R}^+\mid 0<r<1$).
\pause\item It Follows That We Can Now Construct Infinitely Many Infinitesimals. Here Are Some Examples Listed In An Order of Increasing Magnitude: $\varepsilon^3,\varepsilon^2,\frac{\varepsilon}{100},\varepsilon,75\varepsilon,\sqrt{\varepsilon},\varepsilon+\sqrt{\varepsilon}$.
\end{itemize}
\end{frame}
