\begin{frame}
\frametitle{Introductory Calculus--Infinitesimal Approach}
\framesubtitle{1.5 Why `Resurrect' Infinitesimals?}
\label{slide:1.5-03}
What might Have inspired Robinson to take the effort?

Humans look at the historical context of developments. We rejoice in history. We respect those who came before us. After Karl Weierstrass `banished' infinitesimals with his rigorous reformulation of calculus (he also must have felt the burden of creation while conceiving ``this baby'' in the 1870s; he defied Newton, Euler, \dots), mathematical community thought that calculus was `solved'.

Why would Abraham Robinson, a young mathematician in the 1940s and 50s, ``bring infinitesimals back''? Did he want to become famous? Could he not banish infinitesimals convincingly? Why conceive a new kind of number and put it on a rigorous pedestal? Why solve a ``solved problem''?

These questions on ``Foundations of Mathematics'' are deep.
\end{frame}
