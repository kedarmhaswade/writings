\begin{frame}
\frametitle{Introductory Calculus--Infinitesimal Approach}
\framesubtitle{1.5 A Parallel Definition}
\label{slide:1.5-13}
We Defined \alert{Infinitesimal} in Definition [\ref{def:infinitesimal}].

We Have a Corresponding Definition Pertaining The Other Hyperreal Numbers.
\begin{definition}[Finite And Infinite Hyperreal Number]
A Hyperreal Number $b$ Is
\begin{enumerate}
\item \alert{\textit{Finite}}, If $b$ Is between Two Real Numbers.
\item \alert{\textit{Positive Infinite}} If $b$ Is Greater Than Every Real Number.
\item \alert{\textit{Negative Infinite}} If $b$ Is Less Than Every Real Number.
\end{enumerate}
\label{def:finite-infinite}
\end{definition}
Since Every \alert{Infinitesimal} Is between $0$ And, Say, $1$ (Which Are Both Real Numbers), It Is \alert{Finite}. Clearly, \alert{Not Every Finite Is Infinitesimal}.
\end{frame}
