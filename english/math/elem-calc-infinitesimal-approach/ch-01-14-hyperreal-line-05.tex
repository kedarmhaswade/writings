\begin{frame}
\frametitle{Introductory Calculus--Infinitesimal Approach}
\framesubtitle{1.4 Properties of Infinitesimals}
\label{slide:1.4-05}
\begin{definition}[Infinitely Small Or Infinitesimal Number]
A \textit{Number} $\varepsilon$ Is Said to Be \textit{Infinitely Small, Or Infinitesimal,} If $-a<\varepsilon<a \quad\forall a\in\mathbb{R}^+$.
\end{definition}
\begin{itemize}
\item Note: All Infinitesimal Numbers Exist Between \textit{Every} Positive Real Number And Its Negative\footnote{M. Gardner Calls It The \textit{Number Neverland.}}.
\pause\item The \alert{Only \underline{Real} Infinitesimal Number is $0$}.
\pause\item We Introduce A New Number System, \alert{The Hyperreal Numbers}, Which \alert{Contains All The Real Numbers And \underline{Infinitesimals That Are Not Zero}}.
\pause\item \alert{Integers Create Rationals. Rationals Create Reals. Reals Create Hyperreals.}
\pause\item Right Now, \alert{We Study Properties of Hyperreals Needed for The Calculus}. We'll Study Their Creation Later.
\end{itemize}
\end{frame}
