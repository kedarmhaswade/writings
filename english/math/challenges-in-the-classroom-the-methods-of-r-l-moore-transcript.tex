\documentclass[a6paper]{article}
\usepackage{polyglossia}
\usepackage{amsmath,amsfonts,amssymb}   %% AMS mathematics macros
\usepackage[margin=5mm]{geometry}
\usepackage{fontspec}
\usepackage{lettrine}
\usepackage{epigraph, varwidth}
\usepackage{xcolor}
\usepackage{graphicx}
\usepackage{siunitx}
\usepackage{mhchem}
\usepackage{hyperref}
\hypersetup{
    backref,
    colorlinks=true,
    filecolor=magenta,
    linkcolor=blue,
    urlcolor=violet,
    citecolor=blue,
}
\usepackage[skip=\medskipamount]{parskip}
\urlstyle{same}
\setlength{\epigraphwidth}{0.8\textwidth}
\setdefaultlanguage{english}
\setotherlanguages{marathi}
\setmainfont[Scale=1.0]{Noto Serif}
\newfontfamily\marathifont[Mapping=velthuis-sanskrit,Script=Devanagari,Language=Marathi]{Noto Serif Devanagari}

% commands
\newcommand{\perscom}[1]{
    {\footnotesize Personal commentary:  #1}
}
\newcommand{\vect}[1]{\pmb{\vec{#1}}}
\newcommand{\uvec}[1]{\pmb{\hat{#1}}}
\begin{document}
\title{Transcript of the MAA Video Classic: Challenges in the Classroom -- the Methods of R. L. Moore}
\date{August 2021}
\author{Kedar Mhaswade}
\maketitle

After the initial series of still pictures, Prof. R. L. Moore (Prof. Moore from now on) appears on the screen.

Prof. Moore (holding a chalk and talking in front of a few students seated on chairs; there are two blackboards behind him): What do you think? If you have ... I have a somebody who think they can prove this ... how would you know?
(need to get this right).

Narrator: A class in Austin, Texas, 1964. (The teacher sits in his chair). A very unusual class in many ways. The teacher (now in focus)? Prof. R. L. Moore. (The professor calls someone's name who gets up. The professor asks the young man to go at the board [and explain how to do it] and not to erase anything.)

Prof. Moore (appears against a blackboard, all by himself (students aren't there)): Years ago, when I gave a course in advanced calculus, the course was so different from courses ordinarily given by others with the same title that I thought it'd be better to change the title and the number. About 1941, the title was changed to ``Introduction to the Foundations of Analysis'' and the number was changed to 24 and, later on, automatically to \textbf{624}. (With a pause) Ordinarily, in this course, some time near the beginning of the course, I raised a question, ``Whether or not, there exists on the X-axis, a closed and bounded point-set, $M$, such that each point of $M$ is a limit-point of $M$, but $M$ contains no interval.''  

One year, a student said that he had proved this theorem. He went to the board and before he had uttered ... more than ... 3 sentences, I ... asked him whether he'd been reading anything on this subject. He said ``No'', ... but [that] he had talked to someone about it. I said, ``Well, that's enough; you've spoiled it for this class.'' And he sat down. 

After class, in the hall, another student said he certainly did follow(?) it. After he said that it is easy to see what the answer was.

Now I suppose some people will say, ``Why do you make so much on that simple thing? Why you ... tag so much importance to whether or not a person can show that there exists such a point-set [...?] mathematics know the answer?'' 

Well, I'd like for each person who says that to ask himself how he found out what the answer was. Did he make it out for himself? Or was he not told the answer? (Further in close-up) And how does he know what difficulties he may have run into if he'd tried to settle the question without any hint as to what the answer was?

In this connection (looking at the ground as if he is trying to recall something), I've ... often thought of two statements. One of them is a statement by \href{https://en.wikipedia.org/wiki/Anatole_France}{Anatole France} that ``to know is nothing, '' (he is looking at you penetratingly now) ``to imagine is everything.''

(Looking at the ground again, and lightly scratching his nose) The other statement is one that I've read somewhere in a biographical sketch, I think, of Enrico Fermi. I am not sure who wrote it, I wish I did. But it went something like this, ``It is only the clearest of mind, (repeats) only the clearest of mind, that, are the first to think of something, which when once thought of, is clear to everybody.'' And I wonder if this isn't a good example of just that thing. [...?] thought of this. A lot of other people read about it and it seems very simple to them. But could they have thought of this themselves in the first place? 

(A fountain in [what looks like] a university campus appears on the screen.)

Narrator: R. L. Moore has been teaching others to think for themselves for more than 50 years!  
\end{document}
