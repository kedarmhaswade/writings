% $Header$

\documentclass{beamer}

% This file is a solution template for:

% - Giving a talk on some subject.
% - The talk is between 15min and 45min long.
% - Style is ornate.



% Copyright 2004 by Till Tantau <tantau@users.sourceforge.net>.
%
% In principle, this file can be redistributed and/or modified under
% the terms of the GNU Public License, version 2.
%
% However, this file is supposed to be a template to be modified
% for your own needs. For this reason, if you use this file as a
% template and not specifically distribute it as part of a another
% package/program, I grant the extra permission to freely copy and
% modify this file as you see fit and even to delete this copyright
% notice. 


\mode<presentation>
{
  %\usetheme{Warsaw}
  %\usetheme{Berlin}
  %\usetheme{metropolis}
  \usetheme{PaloAlto}
  %\usetheme{CambridgeUS}
  %\usetheme{Berkeley}
  
  % font themes
  \usefonttheme{serif}
  %\usefonttheme{structurebold}

  % color themes
  %\usecolortheme{albatross}
  %\usecolortheme{beetle}
  %\usecolortheme{crane}
  %\usecolortheme{dove}
  %\usecolortheme{fly}
  %\usecolortheme{monarca}
  %\usecolortheme{seagull}
  %\usecolortheme{beaver}
  % color themes
  % or ...

  %\setbeamercovered{transparent}
  \setbeamercovered{transparent=5}
  % or whatever (possibly just delete it)
}


\usepackage[english]{babel}
% Place figures exactly where you mean to
%https://tex.stackexchange.com/a/8633/64425
\usepackage{float}
% Place figures exactly where you mean to
% or whatever

\usepackage[utf8]{inputenc}
% or whatever

\usepackage[T1]{fontenc}
% Or whatever. Note that the encoding and the font should match. If T1
% does not look nice, try deleting the line with the fontenc.

% math
\usepackage{amsmath}
\usepackage{amssymb}
\usepackage{amsthm}
% math

% tikz
\usepackage{tikz}
% tikz
\usepackage{caption}
\usepackage{hyperref}
\hypersetup{
    colorlinks,
    linkcolor={magenta!50!black},
    citecolor={blue!50!black},
    urlcolor={blue!80!black}
}
% large commented sections
\usepackage{comment}
% large commented sections



\title[Calculus Made Easy] % (optional, use only with long paper titles)
{Calculus Made Easy}

\subtitle
{Beautiful Ideas Presented Simply} % (optional)

\author[SPT, MG, KM] % (optional, use only with lots of authors)
{Silvanus P. Thompson\inst{1} \and Martin Gardner\inst{2}}
% - Use the \inst{?} command only if the authors have different
%   affiliation.

\institute[Royal Society] % (optional, but mostly needed)
{
  \inst{1}%
  Original Author
  \and
  \inst{2}%
  Annotator
  }
% - Use the \inst command only if there are several affiliations.
% - Keep it simple, no one is interested in your street address.

\date[August 2025] % (optional)
{Aug 2025 / Free Learner's School Conversations}

\subject{Fun Conversations at Home School}
% This is only inserted into the PDF information catalog. Can be left
% out. 



% If you have a file called "university-logo-filename.xxx", where xxx
% is a graphic format that can be processed by latex or pdflatex,
% resp., then you can add a logo as follows:

% \pgfdeclareimage[height=0.5cm]{university-logo}{university-logo-filename}
% \logo{\pgfuseimage{university-logo}}



% Delete this, if you do not want the table of contents to pop up at
% the beginning of each subsection:
\AtBeginSubsection[]
{
  \begin{frame}<beamer>{Outline}
    \tableofcontents[currentsection,currentsubsection]
  \end{frame}
}


% If you wish to uncover everything in a step-wise fashion, uncomment
% the following command: 

%\beamerdefaultoverlayspecification{<+->}


\begin{document}

\begin{frame}
  \titlepage
\end{frame}

\begin{frame}{Outline}
  \tableofcontents
  % You might wish to add the option [pausesections]
\end{frame}


% Since this a solution template for a generic talk, very little can
% be said about how it should be structured. However, the talk length
% of between 15min and 45min and the theme suggest that you stick to
% the following rules:  

% - Exactly two or three sections (other than the summary).
% - At *most* three subsections per section.
% - Talk about 30s to 2min per frame. So there should be between about
%   15 and 30 frames, all told.

% [Kedar] I am keeping this structure, but abusing it to serve my purpose.
% [Kedar] I expect this `presentation' to have many hundred slides, if I end up doing it right.
% [Kedar] There are three sections per presentation, no subsections, but each section may have many, many slides. Let's see. I am just getting started with LaTeX and Beamer.
% [Kedar] I may roughly make a chapter in his book a section in this presentation.
\section{Introduction}

\begin{frame}
\frametitle{Nonstandard Analysis}
\framesubtitle{Common Words, Uncommon Meanings, Yet Again!}
\label{slide:nonstd-anal}
\begin{itemize}
\pause
\item This Is A Review of A Davis-Hersh\footnote{Both Were Mathematicians with a Gift of Writing Extraordinary Expository Articles.} Article in Scientific American, June 1972.
\begin{itemize}
\pause
\item You Perhaps Haven't Even Heard \alert{``Nonstandard Analysis''} in The Context of Calculus-1.
\pause
\item \alert{Analysis} Refers to \alert{Foundations of Calculus}.
\item \alert{Nonstandard} Only Because The Word \alert{`Standard' Was Taken} to Mean Formalization of Calculus Using Some Other (Beautiful) Way!
\end{itemize}
\pause
\item These Slides Are A Gentle Introduction to Nonstandard Analysis, Are for Personal Benefit, Only Indirectly for Actual Presentation.
\pause
\item There's No Need to Get Intimidated by \alert{Big Words}.
\pause
\item Let's Be Fearless in Guided Imagination, and Make \alert{Inquiry} A Habit.
\end{itemize}
\end{frame}

\begin{frame}
\frametitle{Assumptions}
\framesubtitle{Because A Slide-deck Is Not as Fun as A Fairy Tale!}
\label{slide:assumptions}
\begin{itemize}
\pause
\item We Wish Mathematics Were Fun to Thinking People Who Enjoy Fairy Tales, Rap Music, or Memes.
\begin{itemize}
\pause
\item Sadly, It Isn't!
\end{itemize}
\pause
\item However, This Slide-deck \alert{Isn't So Great to Make It So}!
\pause
\item We Therefore Expect That:
\begin{itemize}
\pause
\item You Want to \alert{Explore Calculus of Your Own Accord}.
\pause
\item You Can \alert{Follow Logical {\tiny (Deductive, Inductive)} Arguments}.
\pause
\item You've Heard of \alert{Natural, Rational, Irrational, Real, Complex Numbers}.
\begin{itemize}
\pause
\item You Understand How a Number Such as \alert{$\sqrt{2}$ Differs from} a Number Such as \alert{$1.737465$, 7, or $2+3i$}.
\end{itemize}
\pause
\item You \alert{Understand The Language of School Algebra}, And \alert{Can Solve Some `Equations'}. Learning \alert{Calculus before Algebra Isn't Helpful}!
\pause 
\item You Have Heard of \alert{`Functions'}, And Know Some of Them (e.g. \alert{Polynomial, Rational, Logarithmic, Exponential, and Trigonometric}).
\end{itemize}
\end{itemize}
\end{frame}

\begin{frame}{}
    \centering
    Most Importantly,\pause

    \Huge \alert{Have Fun!}
\end{frame}


\begin{comment}
\begin{frame}
\frametitle{There Is No Largest Prime Number}
\framesubtitle{The proof uses \textit{reductio ad absurdum}.}
\begin{theorem}
There is no largest prime number.
\end{theorem}
\begin{proof}
\begin{enumerate}
\item<1-> Suppose $p$ were the largest prime number.
\item<2-> Let $q$ be the product of the first $p$ numbers.
\item<3-> Then $q + 1$ is not divisible by any of them.
\item<1-> But $q + 1$ is greater than $1$, thus divisible by some prime
number not in the first $p$ numbers.\qedhere
\end{enumerate}
\end{proof}
\uncover<4->{The proof used \textit{reductio ad absurdum}.}
\end{frame}

\begin{frame}
\frametitle{What’s Still To Do?}
\begin{columns}[t]
\column{.5\textwidth}
\begin{block}{Answered Questions}
How many primes are there?
\end{block}
\pause
\column{.5\textwidth}
\begin{block}{Open Questions}
Is every even number the sum of two primes?
\cite{Goldbach1742}
\end{block}
\end{columns}
\end{frame}

\begin{thebibliography}{10}
\bibitem{Goldbach1742}[Goldbach, 1742]
Christian Goldbach.
\newblock A problem we should try to solve before the ISPN ’43 deadline,
\newblock \emph{Letter to Leonhard Euler}, 1742.
\end{thebibliography}

\end{comment}

%\begin{frame}{Learn from Masters} %{Subtitles are optional.}
%  % - A title should summarize the slide in an understandable fashion
%  %   for anyone how does not follow everything on the slide itself.
%
%  \begin{itemize}
%  \item
%    Use \texttt{itemize} a lot.
%  \item
%    Use very short sentences or short phrases.
%  \end{itemize}
%\end{frame}
%
%\begin{frame}{Make Titles Informative.}
%
%  You can create overlays\dots
%  \begin{itemize}
%  \item using the \texttt{pause} command:
%    \begin{itemize}
%    \item
%      First item.
%      \pause
%    \item    
%      Second item.
%    \end{itemize}
%  \item
%    using overlay specifications:
%    \begin{itemize}
%    \item<3->
%      First item.
%    \item<4->
%      Second item.
%    \end{itemize}
%  \item
%    using the general \texttt{uncover} command:
%    \begin{itemize}
%      \uncover<5->{\item
%        First item.}
%      \uncover<6->{\item
%        Second item.}
%    \end{itemize}
%  \end{itemize}
%\end{frame}
%
%
%\subsection{Second Subsection}
%
%\begin{frame}{Make Titles Informative.}
%\end{frame}
%
%\begin{frame}{Make Titles Informative.}
%\end{frame}
%
%
% [Kedar] Only one section in this presentation.
% [Kedar] No subsections, sections have slides.


\begin{comment}

\section*{Summary}

\begin{frame}{Summary}

  % Keep the summary *very short*.
  \begin{itemize}
  \item
    The \alert{first main message} of your talk in one or two lines.
  \item
    The \alert{second main message} of your talk in one or two lines.
  \item
    Perhaps a \alert{third message}, but not more than that.
  \end{itemize}
  
  % The following outlook is optional.
  \vskip0pt plus.5fill
  \begin{itemize}
  \item
    Outlook
    \begin{itemize}
    \item
      Something you haven't solved.
    \item
      Something else you haven't solved.
    \end{itemize}
  \end{itemize}
\end{frame}

\end{comment}

\end{document}


