\begin{frame}
\frametitle{Infinitesimals in Dynamics}
\framesubtitle{From Geometry to Dynamics}
\label{slide:infinitesimals-in-dynamics}
Since Antiquity, Geometry Played A Crucial Role in Providing Analysis Problems.

Dynamics Was a Relatively Recent Experience. All Kinds of \alert{Moving Bodies} Started to Appear After Newton. Tools of Calculus Help Tremendously to Analyze \alert{Motion}.

Motion Was, After All, Difficult to Ignore. Since The Time of Zeno\footnote{Around 450 BC in Ancient Greece.} of Elea And, His Teacher, Parmenides, We Were Confused about Motion. Zeno Asserted That The \alert{Universe Was Static And All Motion Was Illusion}. He Postulated Several Paradoxes to Confound (Illuminate?) People.

But After Newton, We Had to Study Motion Systematically, Which Would \alert{Remove Zeno's Paradoxes about The Ubiquitous Motion} (If Not Erase His Philosophy) \dots

Let's Analyze An Everyday Motion by The Two Approaches (\alert{Standard: Weierstrass, Nonstandard: Robinson}).
\end{frame}
