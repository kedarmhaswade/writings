\begin{frame}
\frametitle{Implications of Formalization by Limits}
\framesubtitle{Calculus = Arithmetic of Real Numbers?}
\label{slide:implications}
\begin{itemize}
\pause
\item We End the First Part of This Presentation Here. Here's The Blurb That I Don't Fully Understand \dots
\end{itemize}
\begin{block}{Implications \dots}
The reconstruction of the calculus on the basis of the limit concept and its epsilon-delta definition amounted to a reduction of the calculus to the \textbf{arithmetic of real numbers}. The momentum gathered by these foundational clarifications \textbf{led naturally to an assault on the logical foundations of the real-number system itself}. This was a return after two and a half millenniums to \textbf{the problem of irrational numbers, which the Greeks had abandoned as hopeless after Pythagoras}. One of the tools in these efforts was the newly developing field of mathematical, or symbolic, logic.
\end{block}
\end{frame}
