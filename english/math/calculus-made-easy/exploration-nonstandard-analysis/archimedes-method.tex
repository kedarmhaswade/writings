\begin{frame}
\frametitle{Archimedes's Work: Lost And Found}
\framesubtitle{Riddle of The Quadrature of The Parabola}
\label{slide:archimedes-method}
\begin{itemize}
\pause
\item \alert{Archimedes's Work Came in Two Streams of Tradition}: 1) Continuous, 2) After A Gap of 1000 Years.
\pause
\item Of Present Interest Is His \alert{Method of Exhaustion}.
\begin{itemize}
\pause
\item Discovered in Constantinople in 1906!
\pause
\item Relies on \alert{An ``Indirect Argument''} And \alert{Purely Finite Constructions}.
\pause
\item Finds The Volumes of \alert{Surfaces of Revolution} (e.g., A Paraboloid).
\pause
\item Since \alert{Infinitesimals Don't Exist\footnote{Or Do They?}}, It Gives A Logically Acceptable, Rigorous Proof of His Results!
\begin{enumerate}
\pause
\item No References Are Made to \alert{Infinitesimals} in It.
\pause
\item Ironically, You May Find Its Logic Akin to \alert{Weierstrass's} Formulation.
\end{enumerate}
\end{itemize}
\end{itemize}
\end{frame}
