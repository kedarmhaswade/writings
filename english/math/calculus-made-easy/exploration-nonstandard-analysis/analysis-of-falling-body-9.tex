\begin{frame}
\frametitle{Analysis of The Motion of \alert{A Falling Body}}
\framesubtitle{9: An Illustration of Weierstrass's Limit}
\label{slide:analysis-of-falling-body-9}
\begin{itemize}
\pause
\item We Challenge Weierstrass to Prove That $(\frac{\Delta s}{\Delta t}-9.8)=4.9\Delta t$ Is Less Than Some Positive Number $\epsilon$ We Pick, for Every Positive Number $\Delta t$ That Is Less Than Some Positive Number $\delta$ He Picks.
\pause
\item Let's Give Him $\epsilon=0.00049$.
\pause
\item Weierstrass Readily Picks $\delta=\frac{\epsilon}{4.9}=0.0001$.
\pause
\item Clearly, for Every Positive $\Delta t$ Less Than $\delta=\frac{\epsilon}{4.9}$ (i.e., $0<\Delta t<\frac{\epsilon}{4.9}$), $0<4.9\Delta t<\epsilon$.
\pause
\item Weierstrass Succeeds!
\pause
\item Therefore, \alert{as $\Delta t$ Approaches 0, $(\frac{\Delta s}{\Delta t}-9.8)$ Also Approaches 0}.
\pause
\item Therefore, \alert{as $\Delta t$ Approaches 0, $\frac{\Delta s}{\Delta t}$ Approaches \textit{Exactly} 9.8}. The Instantaneous Velocity is $9.8 m/s$ at $t=1s$.
\pause
\item Weierstrass Could Seamlessly Find the Exact Limits of Many Other Functions.
\end{itemize}
\end{frame}
