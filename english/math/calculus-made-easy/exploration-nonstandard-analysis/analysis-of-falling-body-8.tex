\begin{frame}
\frametitle{Analysis of The Motion of \alert{A Falling Body}}
\framesubtitle{8: \textit{Limits} Achieve Limitless Rigor!} 
\label{slide:analysis-of-falling-body-8}
\begin{itemize}
\pause
\item Weierstrass Illustrates The Idea of A \textit{Limit} Before Defining It Precisely \dots
\pause
\item He Tries to Show That However Close to Zero $\Delta t$ Goes, $\frac{\Delta s}{\Delta t}=9.8+4.9(\Delta t)$ May Go Closer to A Real Number. If He Succeeds, His Rigor Holds Strong!
\pause
\item First, We Give Weierstrass A \alert{Positive Real Number}, However Small, $\epsilon$. Thus, \alert{$\epsilon>0$}.
\pause
\item Weierstrass Is Free To \alert{Choose Another Positive Real Number, $\delta$}. In This Case, He Chooses $\delta=\frac{\epsilon}{4.9}; \delta>0$.
\pause
\item Then, Weierstrass Asserts That \alert{For Any Positive Value of $\Delta t$ That Is Less Than $\delta$}, $(\frac{\Delta s}{\Delta t}-9.8)$ Which Equals $4.9\Delta t$ \alert{Must Be Less Than $\cancel{4.9}\frac{\epsilon}{\cancel{4.9}}$, That Is, $\epsilon$}.
\pause
\item IOW, He Asserts That as $\delta$ \textit{Approaches} 0, $(\frac{\Delta s}{\Delta t}-9.8)$ Also Approaches 0, i.e., \alert{$\frac{\Delta s}{\Delta t}$ \textit{Approaches} No Other Number But 9.8}: $\displaystyle{\lim_{\Delta t\to0}\frac{\Delta s}{\Delta t}}=9.8$.
\end{itemize}
\end{frame}
