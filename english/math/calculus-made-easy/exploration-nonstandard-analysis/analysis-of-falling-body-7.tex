\begin{frame}
\frametitle{Analysis of The Motion of \alert{A Falling Body}}
\framesubtitle{7: Weierstrass's Rigorous Formalization. Look Ma! No Infinitesimals!}
\label{slide:analysis-of-falling-body-7}
\begin{itemize}
\pause
\item Like Before, The Instantaneous Position of A Falling Body (in Meters) is Given by A Function of Time: $s=4.9t^2$ (Where Time $t$ Is Measured in Seconds).
\pause
\item Therefore, at $t=1$, $s=4.9$.
\pause
\item A \alert{Finite Time Interval\footnote{\tiny Only Familiar Real Numbers; No $dt,ds$ Business.} $\Delta t$ Later}, Its Instantaneous Position, $s^\prime=4.9(1+\Delta t)^2$.
\pause
\item Correspondingly, The Change in Instantaneous Position, $\Delta s=s^\prime-s=4.9((1+\Delta t)^2-1)$.
\pause
\item $\therefore \Delta s=9.8\Delta t+4.9(\Delta t)^2\implies\frac{\Delta s}{\Delta t}=9.8+4.9\Delta t$.
\pause
\item In a Stark Contrast with Newton, We Define Instantaneous Velocity, Not as A Ratio of Distance and Time ($\frac{\Delta s}{\Delta t}$), But as A \alert{\textit{Limit}} Reached by It.  
\end{itemize}
\end{frame}
