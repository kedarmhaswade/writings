\begin{frame}
\frametitle{The Area And Circumference of The Unit Circle}
\framesubtitle{Argument 2: A \textit{Logical, But Pedantic,} Argument Avoiding Infinitesimals\alert{?}}
\label{slide:arg-2-archimedes-1}
\begin{proof}[Area of Unit Circle = $\frac{1}{2}$ Its Circumference ({\tiny Flawless Logic?})]
\let\qed\relax % Temporarily removes the QED symbol
Let \alert{$S$} Be The Proposition\footnote{Therefore, $S$ Must Be Either \texttt{true} Or \texttt{false}.} That \alert{The Area of A Unit Circle ($\mathbb{A}_C$) = Its Half-circumference ($\mathbb{H}_C$)}.

\alert{If} $S$ Is \texttt{false}, \alert{Then} 
\begin{equation}
\label{eq:ac-hc}
\text{Either}\quad\mathbb{A}_C>\mathbb{H}_C\quad\text{Or}\quad\mathbb{H}_C>\mathbb{A}_C
\end{equation}

Let $\mathbb{D}$ Be The \textit{Positive Difference} between $\mathbb{A}_C$ And $\mathbb{H}_C$.

Therefore, If $\mathbb{H}_C>\mathbb{A}_C$, Then

\begin{equation}
\label{eq:d=hc-ac}
\mathbb{D}=\mathbb{H}_C-\mathbb{A}_C
\end{equation}

Otherwise,
\begin{equation}
\label{eq:d=ac-hc}
\mathbb{D}=\mathbb{A}_C-\mathbb{H}_C
\end{equation}

\end{proof}
\end{frame}
