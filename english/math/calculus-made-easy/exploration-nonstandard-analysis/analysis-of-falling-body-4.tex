\begin{frame}
\frametitle{Analysis of The Motion of \alert{A Falling Body}}
\framesubtitle{4: Newton's Insight, Leibniz's Notation, And Robinson's Formalization}
\label{slide:analysis-of-falling-body-4}
\begin{itemize}
\pause
\item The Instantaneous Position of A Falling Body (in Meters) is Given by A Function of Time: $s=4.9t^2$ (Where Time $t$ Is Measured in Seconds).
\pause
\item Therefore, at $t=1$, $s=4.9$.
\pause
\item An Infinitesimal Time Interval $dt$ Later, Its Instantaneous Position, $s^\prime=4.9(1+dt)^2$.
\pause
\item Correspondingly, The Change in Instantaneous Position, $ds=s^\prime-s=4.9((1+dt)^2-1)$.
\pause
\item $\therefore ds=9.8dt+4.9dt^2$.
\pause
\item The Instantaneous Velocity,$v_1$, at Time $t=1$ Is $\frac{ds}{dt}=\frac{9.8dt+4.9dt^2}{dt}=9.8+4.9dt$.
\pause
\item Since \alert{$dt$ Is Infinitesimal, So Is $4.9dt$}. We Only Entertain Real Quantities, So \textbf{Drop The Infinitesimal}!
\pause
\item Therefore, Instantaneous Velocity, $v_1$, \alert{Is The Real Part of $\frac{ds}{dt}=9.8$}.
\end{itemize}
\end{frame}
