\begin{frame}
\frametitle{Analysis of The Motion of \alert{A Falling Body}}
\framesubtitle{3: Introducing \alert{The Infinitesimal Change}}
\label{slide:analysis-of-falling-body-3}
\begin{itemize}
\pause
\item To Determine The Instantaneous Velocity of A Body That Ever Moves Faster, Newton Made \alert{A Fair Assumption}.
\begin{itemize}
\pause
\item The Velocity Remains Constant During A Tiny (Infinitesimal) Period of Time.
\pause
\item The Continuous Change in Velocity Actually Comes in The Form of Tiny, Discrete Jumps.
\pause
\item Robinson Formalized This Change as \alert{A Number That Behaves Differently from Real Numbers}.
\end{itemize}
\pause
\item We'll First \alert{Study What Newton Proposed}. Then We'll \alert{Go through Bishop Berkeley's Objections And Weierstrass's `Limit'ed Remedy Resulting in Standard Analysis}. Finally, We Will Encounter \alert{Robinson's Revival of Infinitesimals Resulting in Nonstandard Analysis}!
\begin{itemize}
\pause
\item This Will \alert{Compare And Contrast Robinson's Nonstandard Analysis} with \alert{Weierstrass's Standard One}.
\end{itemize}
\end{itemize}
\end{frame}
