\begin{frame}
\frametitle{The Area And Circumference of The Unit Circle}
\framesubtitle{Argument 2: Continued \dots}
\label{slide:arg-2-archimedes-2}
\begin{proof}[Area of Unit Circle = $\frac{1}{2}$ Its Circumference ({\tiny Flawless Logic?})]
\let\qed\relax % Temporarily removes the QED symbol
\dots

Now We Use \alert{Proof by Contradiction} to Show that Both Eq. [\ref{eq:d=hc-ac}] and Eq. [\ref{eq:d=ac-hc}] Lead to Contradiction with Eq. [\ref{eq:ac-hc}]. 

Let's start with Eq. [\ref{eq:d=hc-ac}] first.

We Can \alert{Circumscribe @ The Circle A Regular Polygon with as Many Sides as We Wish}.

Since The Polygon Is Composed of Finite Number of Finite Triangles with Altitude = 1, \alert{Area of The Polygon equals Its Half-perimeter}.
\begin{equation}
\label{eq:ap=hp}
\mathbb{A}_P=\mathbb{H}_P
\end{equation}

As The Number of Sides of The Polygon Circumscribing The Circle Increases, The Difference in Their Areas, $\mathbb{A}_P-\mathbb{A}_C$, Reduces.
%We Can Increase The Number of Sides of The Polygon Such That $\mathbb{A}_P-\mathbb{A}_C<\frac{\mathbb{D}}{2}$
\end{proof}
\end{frame}
