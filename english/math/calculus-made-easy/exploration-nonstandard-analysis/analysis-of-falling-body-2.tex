\begin{frame}
\frametitle{Analysis of The Motion of \alert{A Falling Body}}
\framesubtitle{2: The Concept of \textit{Instantaneous} Velocity}
\label{slide:analysis-of-falling-body-2}
\begin{itemize}
\pause
\item We Take This Routine Experience of A Falling Body for Granted. However, Its Motion is Complicated\dots
\begin{itemize}
\pause
\item First, We Must Be Able to Tell (Somehow) When An Exact Time Interval (e.g., 1 Second) Elapses\footnote{\tiny Exact `Time-keeping' Is As Fascinating As It Is Challenging.}.
\pause
\item We Can Then Think of Measuring The Precise Distance Traveled by A Body Falling Freely (Under Gravity).
\pause
\item Then We Can Tell \textit{Where in Air} The Body Was When $t=1s,2s,\dots$, i.e., We Can Describe Its \alert{Instantaneous Position as A \textit{Function} of Time}.
\pause
\item We Understand The Idea of The \alert{Average Speed of A Moving/Moved Body over The Entire Duration of Its Travel} as $\frac{\text{Total Distance Traveled}}{{\text{Total Time Elapsed since The Beginning}}}$.
\pause
\item But Do We Really Comprehend \alert{How Fast A Falling Body Was Moving \textit{at $t=1s$, or $t=2s$}}?
\pause
\item Determining This Velocity Is Challenging When We Realize that \alert{The Body \underline{Constantly} Moves Faster and Faster \dots}
\end{itemize}
\end{itemize}
\end{frame}
