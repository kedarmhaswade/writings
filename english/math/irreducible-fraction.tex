\documentclass[a4paper]{article}
\usepackage[margin=20mm]{geometry}
\usepackage{amsmath,amssymb,amsthm,amsfonts}   %% AMS mathematics macros
\renewcommand{\epsilon}{\varepsilon}
\renewcommand{\theta}{\vartheta}
\renewcommand{\kappa}{\varkappa}
\renewcommand{\rho}{\varrho} % remember my teacher and friend Adalberto!
\renewcommand{\phi}{\varphi}

\begin{document}
\textbf{Problem: 1959 IMO -- Romania}: For every integer $n$ prove that the fraction 
$$
\frac{21n + 4}{14n + 3}
$$
cannot be reduced further.
\begin{proof}
    $\frac{21n+4}{14n+3}$ cannot be reduced further if its numerator and denominator are \emph{relatively prime}, that is, they only have 1 as the common factor. Thus, we need to prove that $21n+4$ and $14n+3$ are relatively prime for all $n \in \mathbb{N}$.
    
    When $n$ is even, $21n+4$ (since it is even) and $14n+3$ (since it is odd) are relatively prime.

    Let $n=2k+1$ which makes $n$ odd for every $k \in \mathbb{N}$. Thus, we need to prove that $42k+25$ and $28k+17$ are relatively prime $\forall k \in \mathbb{N}$.

\end{proof}
\end{document}

% latex ignores this :-)
    Instead, let us assume that $21n+4$ and $14n+3$ have the highest common factor $q > 1$ for some $n$. Then,
\begin{equation}
    \label{num}
    21n + 4 = q\cdot x 
\end{equation}
and
\begin{equation}
    \label{den}
    14n + 3 = q\cdot y 
\end{equation}
    where $n, q, x, y \in \mathbb{N}$

Note that $x$ and $y$ are relatively prime because if they were not, then they would have some other factor common between them and then $q$ would not be the \emph{highest} common factor of $21n+4$ and $14n+3$.

    Adding and subtracting (\ref{num}) and (\ref{den}) we get:
    \begin{equation}
        \label{add}
        q\cdot (x + y) = 35n + 7
    \end{equation}
    \begin{equation}
        \label{sub}
        q\cdot (x - y) = 7n + 1
    \end{equation}
