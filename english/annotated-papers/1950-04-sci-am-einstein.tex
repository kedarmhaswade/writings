\documentclass[a4]{article}   % smartphone
%\documentclass[17pt]{extarticle}  % print quality
\usepackage{polyglossia}
\usepackage[margin=20mm]{geometry}
\setdefaultlanguage{english}
\setotherlanguages{marathi}
%\textbf with "Sanskrit Text" produces the shape warning because the default Sanskrit Text font on Windows does not have a bold equivalent;
% try Yashomudra or Shobhika; now I like Tiro better
%\newfontfamily\devanagarifont[Script=Devanagari]{Tiro Devanagari Marathi}
\newfontfamily\devanagarifont[Script=Devanagari]{Sanskrit Text}
\usepackage{libertine,libertinust1math}

%\pagenumbering{gobble}
\setlength{\parindent}{0pt}% Remove paragraph indent
\usepackage[skip=\medskipamount]{parskip}
\usepackage{fontawesome5}
\usepackage[style=english]{csquotes}
\usepackage{verse}
\usepackage{xcolor}
\usepackage{hyperref}
\hypersetup{
    colorlinks=true,
    linkcolor=blue,
    filecolor=magenta,
    citecolor=blue,
    urlcolor=purple,
}

% control hyphenation: https://tex.stackexchange.com/a/177179/64425
\tolerance=1
\emergencystretch=\maxdimen
\hyphenpenalty=10000
\hbadness=10000
% control hyphenation: https://tex.stackexchange.com/a/177179/64425

\newcommand \mar[1]{
    \textmarathi{#1}
}
\begin{document}

\title{On the Generalized Theory of Gravitation (Annotated)\\
\large An account of the newly published extension
of the general theory of relativity against
its historical and philosophical background}
\author{Albert Einstein \thanks{%
    Annotator: Kedar Mhaswade}}
\date{}
%\date{}
\maketitle
\hrule
\vspace{5mm}
\textbf{Abstract.} This article appeared in the April 1950 issue of the \textit{Scientific American} magazine (Volume 182, No. 4). Reading it has been an indescribable joy. I decided to annotate it for my own benefit. Annotating is a form of note-taking for better understanding.
\vspace{5mm}
\hrule
\end{document}
