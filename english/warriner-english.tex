\documentclass[12pt]{article}         %% What type of document you're writing.
%%%%% Preamble
%% Packages to use
\usepackage[utf8]{inputenc}
\usepackage[english]{babel}
\usepackage[T1]{fontenc}
\usepackage{epigraph, varwidth}
\usepackage{hyperref}
\hypersetup{
    colorlinks=true,
    linkcolor=blue,
    filecolor=magenta,
    urlcolor=cyan,
}
\urlstyle{same}
\usepackage{sectsty}% http://ctan.org/pkg/sectsty
\usepackage{titlecaps}% http://ctan.org/pkg/titlecaps

%% Title Information.

\title{Notes: Warriner's English Grammar and Composition}
\author{Kedar Mhaswade}
%% \date{2 July 2004}           %% By default, LaTeX uses the current date

%%%%% The Document

\begin{document}
\tableofcontents

\maketitle

\begin{abstract}
    This article consists of my notes from John Warriner's introductory course on English grammar and composition. One may borrow this book from the archive.org library at: \url{https://archive.org/details/englishgrammarco00holt}.
\end{abstract}
\renewcommand{\epigraphsize}{\small}
\setlength{\epigraphwidth}{0.8\textwidth}
\epigraph
{
    Vigorous writing is concise. A sentence should contain no unnecessary words, a paragraph no unnecessary sentences, for the same reason that a drawing should have no unnecessary lines and a machine no unnecessary parts. This requires \textit{not} that the writer make all sentences short or avoid all detail and treat subjects only in outline, but \textit{that every word tell.}
}
{\textit{Will Strunk}}
\sectionfont{\MakeUppercase}
\section{Parts of Speech}
\subsectionfont{\MakeUppercase}
\subsection{The Adverb}
\label{sec: adverbs}
\begin{itemize}
\item 
    A \textit{modifier} makes a word more exact. An adjective modifies a noun or a pronoun. An \textit{adverb} modifies a verb, an adjective, or another adverb.
\item 
    Adjectives answer these questions: \textit{``What kind?'', ``Which one(s)?'', ``How many?''}, and \textit{``How much?''}. Adverbs answer these questions: \textit {``When?'', ``Where?'', ``How?'', ``How often?'',} and \textit{``To what extent?''}. One should ask these questions to the word that adverb modifies. Perhaps hardest of these is the question ``Where?'' because it may erroenously identify a preposition as an adverb.
\item
    Examples (adverbs are italicized and the words they modify are underlined):
        \begin{enumerate}
            \item Our paper boy \underline{delivers} the paper \textit{very} \textit{\underline{early}}. (Here, \textit{very} modifies the adverb \textit{early}, which modifies the verb \textit{delivers}).
            \item He was \textit{too} \underline{tired} to watch the TV. (Here, \textit{too} modifies the verb \textit{tired}; it answers the question, ``To what extent?'')
            \item \textit{Not} is almost always an adverb: He \underline{could} \textit{not} \underline{swim} very well.
        \end{enumerate}
    \item An adverb may come before or after the word it modifies. Some adverbs \textit{interrupt} the words they modify:
        \begin{enumerate}
            \item Martha \textit{usually} \underline{finishes} her homework in the class.
            \item \underline{Live} \textit{simply}.
            \item They \underline{could} \textit{barely} \underline{hear} the speaker.
        \end{enumerate}
    \item Adverbs can help make your writing clear by sharpening the focus. Use adverbs effectively.
\end{itemize}

\section{The Preposition}
\label{sec: preposition}
\begin{itemize}
    \item A preposition shows the \textit{relationship} between a noun or a pronoun and some other word in the sentence:
        \begin{enumerate}
            \item Hank hit the ball \textit{over} the net.
            \item Drake hit the ball \textit{into} the net.
        \end{enumerate}
    \item The following words are often used as prepositions:
        \begin{center}
        \begin{tabular}{ c c c c c }
            aboard & about & above & across & after \\
            at & before & behind & below & between \\
            by & down & during & except & for \\
            from & in & into & like & of \\
            off & on & over & past & since \\
            through & throughout & to & toward & under \\
            until & up & upon & via & with \\
            within & without 
        \end{tabular}
        \end{center}
    \item The Prepositional Phrase: \textbf{A preposition is always used with a noun or a pronoun}. It is called the \textit{object} of the preposition. Sometimes the object has words that modify or describe it:
        \begin{enumerate}
            \item Monica dumped more \underline{leaves} \textit{on} \underline{the huge pile}. Here, the preposition \textit{on} describes the relationship between `leaves' and `the huge pile' which is its \textit{object}.
        \end{enumerate}
\end{itemize}
\end{document}
