\documentclass[12pt]{article}
\usepackage{chemfig}
\usepackage{carbohydrates}
\usepackage{graphicx}
\usepackage{gensymb}
\usepackage[version=4]{mhchem}
\usepackage{polyglossia}
\usepackage{fontspec}
\usepackage{titlesec}
\usepackage{lettrine}
\usepackage{xcolor}
\usepackage{epigraph, varwidth}
\setlength{\epigraphwidth}{0.8\textwidth}
% sectioning used with \paragraph{at subsubsubsection}
\setcounter{secnumdepth}{4}
\titleformat{\paragraph}
{\normalfont\normalsize\bfseries}{\theparagraph}{1em}{}
\titlespacing*{\paragraph}
{0pt}{3.25ex plus 1ex minus .2ex}{1.5ex plus .2ex}
% sectioning
\setdefaultlanguage{english}
\setotherlanguages{marathi}
\newfontfamily\marathifont[Mapping=velthuis-sanskrit,Script=Devanagari,Language=Marathi]{Shobhika}
\usepackage{soul}
\usepackage{hyperref}
\hypersetup{
    backref,
    colorlinks=true,
    filecolor=magenta,
    linkcolor=blue,
    urlcolor=cyan,
}
\urlstyle{same}
% new commands
\newcommand{\ctext}[3]{
    \colorbox{#2}{\parbox{0.9\textwidth}{\textcolor{#1}{#3}}}
}
% new commands
\begin{document}
\title{Self-study Notes on OpenStax AP Biology Books and Other Resources}
\author{Kedar Mhaswade}
\date{2020-21}
\maketitle
\def\dev{\edef~{\string~}\textmarathi}
\section{Unit 1: The Chemistry of Life}
\subsection{The Study of Life}
\subsection{The Chemical Foundation of Life}
\subsection{Biological Macromolecules}
\subsubsection{Introduction}
\lettrine[lines=2]{F}{ood provides} the body with the nutrients it needs to survive. Many of these nutrients are ``macro'' or large molecules necessary for and built by living things. They are \emph{necessary} because humans have evolved to use them for survival: 
\begin{itemize}
    \item amino acids in protein for healthy bone and muscle, 
    \item fats to build new cells, store energy, and digest properly, 
    \item carbohydrates to drive all living processes, and
    \item nucleic acids to carry genetic information.
\end{itemize}

Even so, an \emph{imbalance} in our consumption of these macromolecules may lead to health problems. My personal view here is that since all humans are different (however similar they may be) and there are many theories of a general human diet and human feelings are involved in defining what ``health'' means, it is best to spend some time understanding oneself and one's reactions to dietary components. Such a knowledge about oneself is often illusive because humans are very complex multicellular organisms. Whereas macromolecules are absolutely essetial, there are nutrients like vitamins and minerals that have now become equally essential (in much smaller quantities as compared to macromolecules identified above), perhaps not for survival, but to  \emph{manage} health problems and age gracefully.
\subsubsection{Synthesis of Biological Macromolecules}
Biological Macromolecules (BMm's henceforth) are made of atoms of 6 elements: S, P, O, N, C, H. A majority of cell's \emph{dry mass} is made of BMm's. 
\paragraph{Dehydration Synthesis}
BMm's are \emph{polymers} which are chains of \emph{monomers} joined by covalent bonds. A common polymer is the disaccharide \emph{maltose}: \ce{C12H22O11} which is formed when two glucose (\ce{C6H12O6}) molecules combine releasing a water molecule:\\
\ce{2C6H12O11 -> C12H22O11 + H2O}\\

\setchemfig{cram width=2pt}
\schemestart
\chemname
{\chemfig{
([2,0.5]-)([6,0.5]-HO)-[1]([2,0.5]-\ce{CH2OH})([6,0.5]-H)-O-[7]% rightmost C
([2,0.5]-H)([6,0.5,,,red]-OH)<[5] 
([2,0.5]-H)([6,0.5]-OH)-[4,,,,line width=2pt]([2,0.5]-OH)([6,0.5]-)>[3]
}}
{D-Glucose}
\+
\chemname
{\chemfig{
([2,0.5]-)([6,0.5]-HO)-[1]([2,0.5]-\ce{CH2OH})([6,0.5]-H)-O-[7]% rightmost C
([2,0.5]-H)([6,0.5,,,red]-OH)<[5] 
([2,0.5]-H)([6,0.5]-OH)-[4,,,,line width=2pt]([2,0.5]-OH)([6,0.5]-)>[3]
}}
{D-Glucose}
\arrow(.mid east--.mid west)
\schemestop
\vspace{1cm}
\par\medskip
\schemestart
\chemname
{\chemfig{
([2,0.5]-)([6,0.5]-HO)-[1]([2,0.5]-\ce{CH2OH})([6,0.5]-H)-O-[7]% rightmost C left glucose
([2,0.5]-H)([7]-O-[1]
([2,0.5]-H)-[1]([2,0.5]-\ce{CH2OH})([6,0.5]-)-[0]O-[7]
([2,0.5]-H)([6,0.5]-OH)
<[5]
([2,0.5]-H)([6,0.5]-OH)-[4,,,,line width=2pt]([2,0.5]-OH)([6,0.5]-)>[3]
)
<[5]
([2,0.5]-H)([6,0.5]-OH)-[4,,,,line width=2pt]([2,0.5]-OH)([6,0.5]-)>[3]
}}
{Maltose}
\+
\chemname{\chemfig{H_2O}}{Water}
\schemestop

The monomers share electrons as they form covalent bonds. It is called ``dehydration synthesis'' because it literally means formation (of a polymer, in this case) while losing water.

\paragraph{Hydrolysis}
An almost opposite of Dehydration Synthesis is Hydrolysis which literally means separation (-lysis) with water (hydro-). Thus, the energy contained in bonds of polymers is released when they react with water (water is split into hydrogen and hydroxyl ions) and monomers are created.

\ctext{black}{lime}{
    Dehydration Synthesis \emph{requires} energy and releases water to form [polymer] bonds.
}
\par
\ctext{white}{purple}{
    Hydrolysis requires water and \emph{releases} energy stored in [polymer] bonds.
}

\paragraph{Enzymes}
Both these kinds of reaction require enzymes that are \emph{specific} to the BMm's:
\begin{table}[htbp!]
  \begin{center}
    \caption{Enzymes for Hydrolysis of BMm's}
    \label{tab:enzymes-bmms}
    \begin{tabular}{|r|l|} 
      \hline
      \textbf{BMm} & \textbf{Enzymes}\\
      \hline
      Carbohydrates & amylase, sucrase, lactase, or maltase\\
      \hline
      Proteins & pepsin, peptidase, hydrochloric acid\\
      \hline
      Lipids & lipase\\
      \hline
    \end{tabular}
  \end{center}
\end{table}


\ctext{black}{teal}{
    \small{
        We wonder how life formed on Earth. In 1953, Stanley Miller and Harold Urey developed an apparatus to model early conditions on earth. They wanted to test if \emph{organic molecules could form from inorganic precursors believed to exist very early in Earth’s history}. They used boiling water to mimic early Earth’s oceans. Steam from the “ocean” combined with methane, ammonia, and hydrogen gases from the early Earth’s atmosphere and was exposed to electrical sparks to act as lightning. As the gas mixture cooled and condensed, \emph{it was found to contain organic compounds, such as amino acids and nucleotides}. According to the \textbf{abiogenesis theory} (the original evolution of life or living organisms from inorganic or inanimate substances), these organic molecules came together to form the earliest form of life about 3.5 billion years ago.
   }
}
\subsubsection{Carbohydrates}
The \href{https://en.wikipedia.org/wiki/Carbohydrate}{Wikipedia page on Carbohydrates} is comprehensive. A few things to know about carbohydrates:
\begin{itemize}
    \item A carbohydrate consists of carbon, hydrogen, and oxygen atoms, usually (\emph{but not always}) with a 2:1 hydrogen-oxygen atom ratio and the empirical formula \ce{C_m(H2O)_n}.
    \item In biochemistry, the term \emph{carbohydrates} is synonymous with the term \emph{saccharides} which contain sugars, starch, and cellulose. 
    \item Four main groups (of these, the first two are commonly referred to as sugars): 
        \begin{enumerate}
            \item monosaccharides (simple sugars: \ce{(CH2O)_n}): Simplest of carbohydrates. \textbf{They cannot be hydrolyzed into simpler compounds}. Common examples: glyceraldehyde, ribose, glucose, fructose, and galactose.
            \item disaccharides: Formed when two monosaccharide molecules are joined with the (covalent) glycosidic bonds. Common examples: sucrose, lactose, and maltose.
            \item oligosaccharides: Consist of a few to several monosaccharide molecules (as a rule of thumb: 3 to 10). Common examples: glycolipids.
            \item polysaccharides: These long chains of monosaccharide molecules (as a rule of thumb: more than 10) are the most abundant carbohydrates in human food. Two main types:
            \begin{enumerate}
                \item Storage polysaccharides: starch, glycogen, and galactogen.
                \item Structural polysaccharides: cellulose and chitin.
            \end{enumerate}
        \end{enumerate}
    \item Common names -- fructose (a monosaccharide): \emph{fruit sugar}, glucose (a monosaccharide): \emph{starch sugar}, sucrose (a disaccharide): \emph{cane} or \emph{beet sugar}, lactose (a disaccharide): \emph{milk sugar}.
    \item 
\end{itemize}
\subsubsection{Lipids}
\subsubsection{Proteins}
\subsubsection{Nucleic Acids}

\section{Unit 2: The Cell}
\subsection{Cell Structure}
\paragraph{Introduction}
Cells are the fundamental building blocks of all organisms. There are, of course, many kinds of cell and each kind of cell has a \emph{specific} function to perform. There's tremendous variety of functions that cells perform and the organisms that they create are also diverse (e.g. sponges, bacteria, fungi, plants, animals) in nature. Even so, many cells share a few fundamental characteristics.

A cell's development and growth to perform its eventual tasks is called its ``specialization''. In humans, before a cell develops into its specialized type, it is called a \emph{stem cell}. Thus, stem cells do not undergo changes involved in specialization. Stem cells are called \emph{pluripotent} (that means they are capable of giving rise to several different cell types). There is a great potential in stem cell research to address human illnesses.
\subsubsection{Studying Cells}
\ctext{red}{white}{
    Cells of a kind --- form $\rightarrow$ Tissue --- combine to form $\rightarrow$ Organ (e.g. stomach, heart, or brain) --- make up $\rightarrow$ Organ System(s) (e.g. digestive system, endocrine system) --- work together to form $\rightarrow$ an Organism.
}
\subsubsection{Prokaryotic Cells}
\subsubsection{Eukaryotic Cells}
\subsubsection{The Endomembrane System and Proteins}
\subsubsection{Cytoskeleton}
\subsubsection{Connections between Cells and Cellular Activities}


\subsection{Structure and Function of Plasma Membranes}
\paragraph{Introduction}
The plasma membrane (the PM, henceforth), which is also called the cell membrane, has many functions; but, the most basic one is to \emph{define the borders} and \emph{act as gatekeeper} for the cell.

The PM is a selectively permeable membrane that allows only certain molecules to freely enter cells. Other molecules may require assistance. A striking example of this is the (membrane) protein called NPC1 (\href{https://en.wikipedia.org/wiki/NPC1}{Niemann-Pick Type C1}) which mediates the intracellular cholesterol trafficking in mammals. NPC1 is therefore called a \emph{cholesterol transporter}. A mutation in the NPC1 gene (that encodes the NPC1 protein) may prevent the entry of cholesterol into cells that results in cholesterol accumulation in intercellular space. But this problem may be a boon in disguise because if such an organism is infected with the Ebola virus, since the NPC1 protein's cholesterol transportation is impared, its cells are impervious to the virus even when it binds with NPC1.

\subsection{Metabolism}
\subsubsection{ATP: Adenosine Tri-Phospate}
\begin{itemize}
    \item Even exergonic reactions (those that \textit{release} energy) require a small amount of
        \textit{activation energy} to proceed.
    \item Products of endergonic reactions (those that \textit{require} energy input) have
        more energy than that of their reactants. Where does the cell provide this energy from?
    \item ATP is the molecule that provides the energy required for the endergonic reactions 
        as well as the small activation energy required for the exergonic reactions.
    \item ATP is a \textit{relatively} simple molecule: C\textsubscript{10}H\textsubscript{16}N\textsubscript{5}O\textsubscript{13}P\textsubscript{3} %$C_{10}H_{16}N_{5}O_{13}P_{3}$
        \begin{itemize}
            \item It has adenosine bound to three phosphate groups that are named alpha- (closest to ribose), beta-, and gamma-phosphate (farthest from ribose).
            \item Adenosine is a \textit{nucleoside}---a compound commonly found in DNA or RNA, consisting of a purine or pyrimidine base linked to a sugar.
        \end{itemize}
    \item ATP \textit{hydrolyzes} to ADP (Adenosine Di-Phosphate) and inorganic phosphate. Regeneration of ATP requires energy to be supplied. Thus, hydrolysis of ATP is reversible: \\
        \schemestart ATP + H\textsubscript{2}O\arrow{<=>}ADP + P\textsubscript{i} + free energy\schemestop\par
    \item Cellular conditions differ from standard conditions. In cellular conditions, the $\Delta G$ (change in Gibbs free energy) of ATP's hydrolysis is -57 kJ/mol = -14 kcal/mol.
\end{itemize}


\subsection{Cellular Respiration}
\paragraph{Introduction}
Energy enters an organism’s body in one form and is converted into another form that can fuel the organism’s life functions. In the process of photosynthesis, plants and other photosynthetic producers take in energy in the form of light (solar energy) and convert it into chemical energy, glucose, which stores this energy in its chemical bonds. Then, a series of metabolic pathways, collectively called cellular respiration, extract the energy from the carbon–carbon bonds of glucose and convert it into a form that all living things can use—both producers, such as plants, and consumers, such as animals.
\subsubsection{Energy in Living Systems}
\subsubsection{Glycolysis}
\subsubsection{Oxidation of Pyruvate and the Citric Acid Cycle}
\subsubsection{Oxidative Phosphorylation}
\subsubsection{Metabolism without Oxygen}
\subsubsection{Connections of Carbohydrate, Protein, and Lipid Metabolic Pathways}
\subsubsection{Regulation of Cellular Respiration}
\subsection{Photosynthesis}
\emph{Many organisms} access stored energy by \emph{eating} or \emph{ingesting (consuming)} other organisms: \textmarathi{jiivo jiivasya jiivanam|}. All of this energy can be traced back to photosynthesis.
\subsubsection{Overview}
Photosynthesis is the \emph{only} biological process that can capture the energy in sunlight and convert it into the energy stored in the covalent bonds of sugar molecules. Additionally, it acts as a source of oxygen necessary for \emph{many} living organisms.

\emph{Photoautotrophs} (organisms that use light to synthesize their own food) such as plants, algae(\textmarathi{\dev aljii}), and cyanobacteria are the only ones that can perform photosynthesis.

\emph{Chemoautotrophs} (organisms that use energy stored in inorganic molecules, and not sunlight, to synthesize their own food) found in deep sea represent another class of \emph{autotrophs}.

Animals, fungi, and most other bacteria are termed \emph{Heterotrophs} because they have to rely on the sugars produced by Photoautotrophs.

\paragraph{Main Structures and Summary}
Photosynthesis is a multi-step process that requires
\begin{enumerate}
    \item Specific wavelengths of visible sunlight and
    \item Carbon dioxide (low in energy) and water as substrates.
\end{enumerate}
It 
\begin{enumerate}
    \item Releases oxygen and
    \item Produces glyceraldehyde 3-phosphate (G3P) which can be synthesized into different sugar molecules (G3P has three carbon atoms; two G3P molecules form one glucose molecule). G3P is formed when one CH2OH from glycerol is replaced in an aldehyde and then a hydrogen is replaced by phosphate (H2PO4-).
\end{enumerate}
\begin{figure}[ht!]
    \centering
    \includegraphics[width=0.8\linewidth]{g3p.png}
    \caption{Glyceraldehyde-3 phosphate}
    \label{fig: g3p}
\end{figure}

The effective chemical equation of photosynthesis is:

\ce{6CO2 + 6H2O ->[Sunlight] C6H12O6 + 6O2}

But note that there are several steps involved in this transformation.

\paragraph{Basic Photosynthetic Structures}
\begin{enumerate}
    \item In plants, it generally takes place in \emph{leaves} which have several layers of cells. The layer for photosynthesis is called \emph{mesophyll} (Greek: \emph{meso}: in the middle, \emph{-phyll}: leaf or pigment)
    \item The gas exchange takes place through \emph{stomata} (Greek: \emph{stoma}: mouth or opening)
\end{enumerate}

\subsection{Cell Communication}
\subsubsection{Signaling Molecules and Cellular Receptors}
\paragraph{Introduction}
In order to respond properly to external stimuli, cells have developed \emph{sophisticated} mechanisms of communication that can
\begin{enumerate}
    \item receive a \emph{message},
    \item transfer the information across the plasma membrane, and
    \item produce changes in the cell as a response
\end{enumerate}
Both single-celled (yes, single-celled) and multicellular organisms communicate with each other at a cellular level and across the organisms.

Communication within a cell is called the \emph{intracellular} communication, whereas the communication between two or more cells is called the \emph{intercellular} communication.

The small, usually volatile, and soluble signaling molecules that \emph{bind} to another specific molecule (and deliver a signal in the process) are called \emph{ligands}.

Ligands bind or interact with proteins called the \emph{receptor proteins} in the \emph{target cells} (the cells that would be affected by the ``signals''). Specific ligands bind with specific receptors. 

\subsection{Cell Reproduction}
\subsubsection{Introduction}
\textbf{A human, like every sexually reproducing organism, begins life as a fertilized egg (embryo) or zygote}. All multicellular organisms use cell division (even after they are fully grown) for growth and maintenance and repair of cells and tissues. Cell division is \emph{regulated closely}. Unicellular organisms also use cell division for their reproduction. 

Not all cells in the human body can reproduce \emph{to repair tissues}. Most nerve cells, for example, are not capable of regeneration. This means that people who have damaged their nerve cells or nervous system are often left paralyzed\footnote{This may change because of research}.
\subsubsection{Cell Division}
The cell cycle is an orderly sequence of events. It describes the stages of a cell's life from the division of a single parent cell to the production of two new genetically identical daughter cells.
\paragraph{Genomic DNA}
A cell's DNA, packaged as a double-stranded DNA molecule, is called its \emph{genome}. 

\lettrine[lines=2]{I}{n prokaryotes}, the genome is composed of a single DNA molecule. It's found in an area called \emph{nucleoid}. Sometimes, smaller loops of DNA called \emph{plasmids} are also present. Prokaryotes exchange plasmids with other prokaryotes. \emph{Antibiotic resistance} may be developed in a bacteria colony as a result of plasmid exchange between a resistant donor and recipient cells.

\lettrine[lines=2]{I}{n eukaryotes}, the genome is composed of several \emph{double stranded} linear DNA molecules. Each species has a \emph{characteristic number} of chromosomes. Human somatic (body) cells have 46 chromosomes, whereas human gametes (sperm or egg cells) have 23 chromosomes each.

\textbf{Matched pairs of chromosomes in a diploid organism are called \underline{homologous chromosomes}}. Each copy of a homologous chromosome originates in a different parent. The individual homologous chromosome therefore is not identical to its copy. \textbf{Regions of chromosome are identified as genes}. Genes are the \emph{functional units} of chromosome and they determine the phenotypical (externally visible) characteristics of an individual by \emph{encoding specific proteins}. 

A location on a chromosome is called \textbf{locus} and a gene is said to be located at some locus. Since there are population variations, two different humans are more likely to have variations in the genes. Although we do not have the DNA sequence of each human, there are statistical models that determine the form or version of each gene. \textbf{Each form or version of a gene at a given locus is called an allele}.
\subsubsection{The Cell Cycle}
\subsubsection{Control of the Cell Cycle}
\subsubsection{Cancer and the Cell Cycle}
\subsubsection{Prokaryotic Cell Division}

\section{Unit 3: Genetics}
\subsection{Meiosis and Sexual Reproduction}
\paragraph{Introduction}
\lettrine[lines=3]{T}{he ability} to reproduce ``in kind'' (the offspring closely resembles the parent(s)) in a basic characteristics of all living things. The Dr. Seuss story ``Horton hatches the egg'', however lyrical and sentimental, is ascientific (that, of course, does not reduce its literary value).

Many unicellular organisms, such as yeast, and a few multicellular organisms, such as sponges, \textbf{can produce genetically identical clones of themselves through cell division}. However, many single-celled organisms and most multicellular organisms reproduce regularly using ``Sexual Reproduction'', a method requiring two parents. Sexual reproduction occurs through
\begin{enumerate}
    \item the \emph{production by each parent of a \textbf{haploid} cell each containing only half of the required genetic information} (meiosis), and
    \item the \emph{fusion of these two haploid cells to form a single, unique diploid cell with a complete set of genetic information} (fertilization of an egg).
\end{enumerate}
\emph{Variation} or \emph{Variability} is an important component of the \emph{evolutionary success} of species. A vast majority of eukaryotes use \textbf{meiosis} and \textbf{fertilization} to ``reproduce''. But like Yashiro et.al. have argued in \cite{yashiro-matsuura} the prevalence of sexual reproduction remains an enigma. \emph{Queens} of Asian termites can reproduce both sexually and asexually (asexual reproduction may be better termed ``cloning'') by a method known as \emph{parthenogenesis} (The prefix parthen- means virgin in greek; development of embryo from an unfertilized egg).

\subsubsection{The Process of Meiosis}
\paragraph{Introduction}
In \emph{mitosis} cells divide to grow, replace other cells, and \emph{reproduce asexually}. \textbf{Without mutation}, or \textbf{changes in the DNA}, the daughter cells produced by mitosis receive a set of genetic instructions that is identical to that of the parent cell. Because changes in genes drive both the unity and diversity of life, organisms without genetic variation cannot evolve through natural selection. Evolution occurs only because organisms have developed ways to vary their genetic material. This occurs through 
\begin{itemize}
    \item mutations in DNA, 
    \item recombination of genes during meiosis, and 
    \item meiosis followed by fertilization in sexually reproducing organisms.
\end{itemize}
Sexual reproduction requires that diploid (2n) organisms produce haploid (1n) cells through meiosis and that these haploid cells fuse to form new, diploid offspring. The \emph{union of these two haploid cells}, one from each parent, is \emph{fertilization}.

The eukaryotic DNA is contained in chromosomes, and chromosomes occur in \emph{homologous pairs} (homologues). At fertilization, the male parent contributes one member of each homologous pair to the offspring, and the female parent contributes the other. With the exception of the sex chromosomes, homologous chromosomes contain the same genes, but \emph{these genes can have different variations, called alleles}. (For example, you might have inherited an allele for brown eyes from your father and an allele for blue eyes from your mother.) As in mitosis, homologous chromosomes are duplicated during the S-stage (synthesis) of interphase. However, unlike mitosis, in which there is just one nuclear division, meiosis has two complete rounds of nuclear division—meiosis I and meiosis II. These result in four nuclei and (usually) four daughter cells, each with half the number of chromosomes as the parent cell (1n). The first division, meiosis I, separates homologous chromosomes, and the second division, meiosis II, separates chromatids. 


\ctext{white}{purple}{During meiosis, DNA replicates ONCE but divides TWICE, whereas in mitosis, DNA replicates ONCE but divides only ONCE.}

It is important to note that \cite{embryo-project} meiosis is the process by which sexually reproducing organisms generate gametes (sex cells). However, the main function of meiosis is \textbf{the reduction of the ploidy (number of chromosomes) of the gametes from diploid (2n, or two sets of 23 chromosomes) to haploid (1n or one set of 23 chromosomes)}.

Both males and females use meiosis to produce their gametes, although there are some key differences between the sexes at certain stages. In females, the process of meiosis is called \textbf{oogenesis}, since it produces oocytes and ultimately yields mature ova(eggs). The male counterpart is \textbf{spermatogenesis}, the production of sperm. While they occur at different times and different locations depending on the sex, both processes begin meiosis in essentially the same way. \textbf{Meiosis occurs in the primordial germ cells, cells specified for sexual reproduction and separate from the body’s normal somatic cells}  \cite{embryo-project}.

See Figure \ref{fig: meiosis-1} for some terminology.

\begin{figure}[h!]
    \centering
    \includegraphics[scale=0.3]{meiosis-1.png}
    \caption{Meiosis terminology for a germ cell}
    \label{fig: meiosis-1}
\end{figure}

\newpage

The \href{https://upload.wikimedia.org/wikipedia/commons/7/74/Meiosis_Stages.svg}{Wikipedia page} has many good figures and explanation of meiosis. It is important to realize that the \textbf{germ cells that undergo meiosis are different from the sperm and egg cells}. Germ cells are diploid (2n) cells. 

Meiosis proceeds in two distinct stages called Meiosis I and Meiosis II.

Steps in Meiosis I:
\begin{enumerate}
    \item Mitosis preceeds meiosis. A germ cell divides by mitosis. Figure shows the \textbf{diploid} oogonium (egg cell) or spermatogonium (sperm). This is the Prophase I. At this stage, as you can see, there are sister chromatids for each chromosome (including the X- and Y-chromosome). 
\begin{figure}[h!]
    \centering
    \includegraphics[scale=0.4]{prophase-I.png}
    \caption{Prophase-I: chromosomes condense and crossover begings to happen}
    \label{fig: prophase-I}
\end{figure}
        The synapse formation is shown below. Note the difference between homologous chromosomes and sister chromatids. Figure \ref{fig: synapse-formation} shows the replication of the genetic material in the diploid cell. 
\begin{figure}[h!]
    \centering
    \includegraphics[scale=0.4]{synapse-formation.jpeg}
    \caption{Prophase-I: Synaptonemal complex, centromere, and synapse formation}
    \label{fig: synapse-formation}
\end{figure}
        Located at intervals along the synaptonemal complex are large protein assemblies called \textbf{recombination nodules}. These assemblies mark the points of later chiasmata and mediate the multistep process of \emph{crossover}—or \emph{genetic recombination}—between the \emph{non-sister chromatids}. Near the recombination nodule on each chromatid, the double-stranded DNA is cleaved, the cut ends are modified, and \textbf{a new connection is made between the non-sister chromatids}. A chiasma (pl. chiasmata) is the point of contact, the physical link, between two (non-sister) chromatids belonging to homologous chromosomes. At a given chiasma, an exchange of genetic material can occur between both chromatids, what is called a chromosomal crossover, but this is much \href{https://en.wikipedia.org/wiki/Chiasma_(genetics)}{more frequent during meiosis than mitosis}. In meiosis, absence of a chiasma generally results in improper chromosomal segregation and \textbf{aneuploidy} (presence of an abnormal number of chromosomes in a cell, for example a human cell having 45 or 47 chromosomes instead of the usual 46.)


\end{enumerate}


\newpage
\subsection{Mendel's Experiments and Heredity}

\subsubsection{Mendel's Experiments and the Laws of Probability}
Johann Gregor Mendel conducted experiments to demonstrate the existence of genes long before the term was coined. His experiments are an exemplary lesson in the ``design and execution of scientific experiments''. Mendel is regarded as the father of modern genetics.

\subsubsection{Extra: Mendel's Experiments (Refers to Various Resources)}
Mendel wanted to investigate if there was a \emph{general law} for the formation and development of \emph{hybrids}, something he noted had not been previously formulated \cite{mendel-fisher}: 
\epigraph{
    Those who survey the work done in this department will arrive at the conviction that among all the numerous experiments made, \emph{not one has been carried out to such an extent and in such a way as to make it possible to determine the number of different forms under which the offspring of hybrids appear}, or to arrange these forms with certainty according to their separate generations, or definitely to ascertain their statistical relations.
}
{
    ---\textit{Johann Gregor Mendel \cite{mendel-fisher}}
}

Mendel and his colleagues carried out their experiments for nearly a decade (1856-1865). Mendel required \cite{mendel-fisher} that the experimental plant must
\begin{enumerate}
    \item possess constant differentiating factors
    \item be capable, during flowering, of protecting from the influence of \emph{foreign pollen}
\end{enumerate}

After careful analysis of various options, Mendel chose ``garden pea'' (\emph{Pisum Sativum L.}) to experiment with. He decided to experiment with the garden pea and examine following traits:
\begin{enumerate}
    \item Seed shape: Round or Wrinkled (a \emph{seed\footnote{A characteristic that can be ascertained by simply analyzing the seed}} characteristic)
    \item Cotyledon (the first leaves that sprout out of the seed upon germination) color: Yellow or Green (a seed characteristic)
    \item Seed-coat color: Colored or White (a \emph{plant\footnote{A characteristic that can only be ascertained by waiting for a plant to grow from the seed} characteristic})
    \item Pod shape: Inflated or Constricted (a plant characteristic)
    \item Pod color: Green or Yellow (a plant characteristic)
    \item Flower position: Axial (along the stem) or Terminal (at the end of the stem) (a plant characteristic)
    \item Stem length: Long (6--7 feet) or Short (less than a foot) (a plant characteristic)
\end{enumerate}

An interesting and somewhat philosophical question arises: ``Chicken first or egg first?'' Mendel tried to answer it from what he had available: Generations of ``true-bred'' plants which were available in the monastery garden where he conducted the experiments. 

There is a lot of terminology here and we need to understand just what the \emph{key terms} mean before proceeding. We should also put ourselves into Mendel's shoes and try to reason the way he did. Explaining what Mendel did using the modern Genetics terms is likely to be confusing. For example, we shouldn't be using a term like \emph{homozygous} because Mendel didn't even know what that term meant. Of course, at some point we should reconcile all of this in terms of contemporary knowledge. 

Note that a seed belongs to the generation of the plant that comes \emph{after} the [generation of the] plant whose fruit bears the seeds.

Here are the key terms in the order of their dependence (i.e. a term appearing later in the list will depend on the terms appearing before):
\begin{enumerate}
    \item \textbf{Character}: This is a heritable feature, for example, stem length. Typically, an organism simultaneously exhibits several different Characters.
    \item \textbf{Trait}: Each \emph{variant} of a Character is called a Trait, for example, a plant could have ``Tall'' (stem length) Trait.
    \item \textbf{True-bred}: This is an adjective (past participle of ``True-breed''), typically applied to an organism. An organism is said to be True-bred from seed alone with respect to certain Traits when it passes down those Traits to its offspring (more generally, descendants) \emph{of many generations}. Since flowering plants have both female and male reproductive organs, true-bred Traits of many plants can be created by their \emph{self-pollination}. (Note, however, that flowering plants like apple are notorious for not breeding true from seed; apple needs to be \emph{grafted}). Mendel started with true-bred pea plants mainly because they were relatively much easier to true-breed from seed by self-pollination.
    \item \textbf{Cross}: Fertilization of the egg by the pollen from another plant. Since a single flowering plant has, unlike humans, both male and female reproductive organs, Mendel had to prevent their self-pollination in order to carry out the desired crosses. In a way, a cross-bred offspring is opposite of a true-bred offspring.
    \item \textbf{Monohybrid Cross}: Cross between two \emph{true-bred} organisms of a species that differ in a single Trait. Mendel painstakingly carried out such crosses with peas (e.g. by crossing true-bred green-seed plants with true-bred yellow-seed plants).
    \item \textbf{Dominance}: The ability of a certain Trait (of a given Characteristic) to express in 
    \item \textbf{Submissiveness}:
\end{enumerate}

\subsubsection{Characteristics and Traits}
\subsubsection{Laws of Inheritance}

\subsection{Modern Understandings of Inheritance}
\subsection{DNA Structure and Function}
\subsection{Genes and Proteins}
\subsection{Gene Regulation}
\subsection{Biotechnology and Genomics}

\section{Unit 4: Evolution}
\subsection{Evolution and Origin of Species}
\subsection{The Evolution of Populations}
\subsection{Phylogenies and the History of Life}

\section{Unit 5: Viruses}

\begin{thebibliography}{00}
    \bibitem{mendel-fisher} Allan Franklin, A. W. F. Edwards, Daniel J. Fairbanks, Daniel L. Hartl. Ending the Fisher-Mendel Controversy. University of Pittsburgh Press, 2008. Page 2.
    \bibitem{yashiro-matsuura} Yashiro, Toshihisa and Matsuura, Kenji. (2014). Termite queens close the sperm gates of eggs to switch from sexual to asexual reproduction. Proceedings of the National Academy of Sciences. 111. 17212-17217. 10.1073/pnas.1412481111. 
    \bibitem{embryo-project} Maayan, Inbar. ``\href{http://embryo.asu.edu/handle/10776/2084}{Meiosis in Humans}''. Embryo Project Encyclopedia (2011-03-24). ISSN: 1940-5030. 
    \bibitem{meiosis-int-demo} \href{https://www.cellsalive.com/meiosis_js.htm}{Animal Cell Meiosis}.
\end{thebibliography}
\end{document}
