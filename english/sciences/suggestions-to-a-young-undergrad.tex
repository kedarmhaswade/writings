\documentclass[a6paper]{article}
\usepackage{polyglossia}
\usepackage[margin=5mm]{geometry}
\usepackage{fontspec}
\usepackage{lettrine}
\usepackage{epigraph, varwidth}
\setlength{\epigraphwidth}{0.8\textwidth}
\setdefaultlanguage{english}
\setotherlanguages{marathi}
\setmainfont[Scale=1.0]{Source Sans Pro}
\newfontfamily\marathifont[Mapping=velthuis-sanskrit,Script=Devanagari,Language=Marathi]{Noto Serif Devanagari}
\usepackage{xcolor}
\usepackage{hyperref}
\hypersetup{
    backref,
    colorlinks=true,
    filecolor=magenta,
    linkcolor=blue,
    urlcolor=violet,
    citecolor=blue,
}
\urlstyle{same}
\usepackage[skip=\medskipamount]{parskip}
% control hyphenation: https://tex.stackexchange.com/a/177179/64425
\tolerance=1
\emergencystretch=\maxdimen
\hyphenpenalty=10000
\hbadness=10000
% control hyphenation: https://tex.stackexchange.com/a/177179/64425
\begin{document}
\title{Letter to a Young Undergrad\footnote{a university student who has not yet received a first degree.}}
\date{December 2021 {\small (last updated)}}
\author{Kedar Mhaswade}
\maketitle
%TC:ignore
\lettrine[lines=3]{D}{ear Reader}: This\footnote{It has undergone several revisions since it was first written. And although I do not acknowledge, many friends have made valuable suggestions.} is what I \emph{might} have read if someone I trust were to give me suggestions about how I should think about my undergrad education back when I was an undergrad. Please take them with a grain of salt but remember that I have tried to make these suggestions \emph{objective}. 

This is \emph{not} an advice for I truly believe in the maxim, ``No advice is as bad as advice". I have a Master's degree in Computer Science and I have worked in the software industry for a couple of decades, but that alone does not qualify anyone to give suggestions to young people. I will keep this short (about 1200 words) and will stick to my thoughts about how one should approach one's graduation and career. Since this is a personal opinion, it is not generally applicable. Perhaps it is about an education in science and engineering and not arts. But you know, some people say that artistic pursuit of anything can be rewarding in the end. If you need some \emph{masterly advice}, especially from a seasoned scientist (or naturalist), take a look at a book like Edward O. Wilson's ``Letters to a Young Scientist" \cite{letters}. To limit this narrative to actionable items, I have used bullet points:
%TC:endignore

\begin{enumerate}
    \item \textbf{Know thyself}. Although this feels like a cliche, it is useful to have a realistic view of oneself at any point in time. In early career, this can translate to knowing what you \emph{truly} like to do. Nothing else really matters. It is like what Einstein is said to have once said, ``That is the way to learn the most; when you are doing something with such enjoyment that you don't notice that the time passes". This means that you should ask hard questions about what you would like to \emph{do} the most. Is it mechanical engineering with all its physical grandeur of well-designed systems (for example, did you carefully observe the construction of a nearby flyover or the massive metro work?), computer science with its application of discrete mathematical concepts to solve real-world problems (for example, how much do you like to \emph{program your computer}?), or biotechnology with its promise to use technology to understand and enhance human life (for example, would you like to get your DNA sequenced and then would you analyze it yourself?), or something else. I agree, this is not an easy decision. One needs to go into something deeply enough in order to find it and even then it might remain illusive. Psychology plays an important role. Knowing what (for example, winning a competition, artistic pursuit, recognition, awards etc.) \emph{really} motivates you is tricky. A mix of intrinsic motivation (doing something \emph{for the sake of it}) and extrinsic motivation (doing something for the other things like fame (even a fickle fame on social media) or money it may bring) seems to work for most of us. I have a simple trick that may help decide what you like to do. Let us say you are procrastinating while you have some task at hand. Then try to observe what you \emph{actually} do when you procrastinate (well, of course something other than chatting on a smartphone). Maybe you should be doing that. But I warn you, it -- having a \emph{realistic} understanding of our (changing) self -- is difficult. 
    \item \textbf{Our work defines us}. If you \emph{keep falling in love} with what you do, sooner or later, it will pay off. You see, it is a feedback loop. On the one hand, you must have fun doing whatever you do, on the other, you get better at that craft since you simply keep doing it. For an undergrad student this means solving problems and being curious. And these problems come in a broad variety. You should have the zeal to solve every problem that appears worthwhile. Your allegiance is to the problem. You shouldn't care if you are spending an inordinate amount of time on solving problems of interest. There are many anecdotes where students tenaciously solved problems posed to them and the payoff was huge. I know that not all of us will be able to solve unsolved problems, but why do you care? If someone posed a problem not known to us before and we solved it, for all practical purposes, we solved a previously unsolved problem! Problem solving needs curiosity, a sense of wonder. One might also feel a need of \emph{value addition}, as David Aldous \cite{david} and Matt Might \cite{matt} have successfully argued. Such a need arises because of our innate desire to create something of our own and contribute to an ever-expanding body of human knowledge. Be aware, however, that a lasting contribution is extremely hard; one need not sacrifice one's happiness for it. Instead, having a satisfying career in something meaningful to you may be a worthwhile goal. I avoided the word \emph{job} because it may have negative connotations.
    \item \textbf{Research shows the direction; industrial engineering implements it}. People say that to do research, one needs a ``research mindset". That may be true. A famous scientist-mathematician \cite{hamming} once said, ``You are doing research if you \emph{don't} know what you are doing. You are \emph{not} doing engineering if you don't know what you are doing". The essence of this is, of course, that there are a lot of unknowns in research whereas the main challenge in engineering is soundly \emph{building} things with an ad hoc design. Indeed, we say that something is ``production quality" when it is \emph{built} right and keeps working. It is about applying the best practices tenaciously. Of course even well-engineered things may fail, but the reasons of their failure often point to a faulty design or faulty procedures.  It is generally accepted that doing research is finding a problem that you are prepared to solve for a long time. It is possible that you might just keep searching for a long time, but don't let that frustrate you. It is a marathon, not a 100-meter dash. In time, you will choose to do research in an academic setting, research in an industrial setting, or product development in an engineering company\footnote{Although we need good managers to manage projects, I hope you defer that decision for now.}. Third choice is what most of us do as a career and that is okay!
    \item \textbf{Believe in yourself}. This sounds vacuous, but it is true. Many a time we just disregard our abilities. Whereas confidence can be misleading\footnote{When there is a skill-confidence mismatch, we suffer from a cognitive bias called the Dunning-Kruger effect\cite{dke}.}, a lack of it is unfortunate. Everyone has a potential to succeed (and you know, success has a ``personal" definition) and fulfillment of that potential is meaningful. 
    \item \textbf{Skills trump university degrees}. I believe we knew this all along. However, we shouldn't misconstrue this suggestion to mean ``drop out of college" even though Steve Jobs and Bill Gates did that. As a \emph{lifelong learner}, one should spend a majority of time learning new skills and honing existing skills. Remember, luck happens when preparation meets opportunity.
    \item \textbf{Use the Internet wisely and read; read a lot}. The Internet is a treasure trove and unless you use it wisely, you will have a lot to miss. Recent knowledge enhancement and educational experiments like Archive.org, Wikipedia, Coursera, edX, OpenStax, Khan Academy, YouTube, Reddit, Stack Overflow, GitHub etc. have a lot to offer although they are not a substitute for formal education \emph{yet}.
    \item \textbf{Try to seek a balance in a balanced manner!} This is not to say that you must behave in a balanced manner all the time for it may become too mechanical. Balance refers to living a communal, humorous, all-around college life to the extent possible. Many of us make lifelong friends during our college years and they are arguably our best years of life. As normal people, in isolation, we can only do so much. Whereas isolating yourself when doing some \emph{deep work} is necessary, having a reasonably sized and dependable network of people is essential for leading a good professional and social life. Harnessing the power of a personal network is a skill whose seeds are sown during the college education.
\end{enumerate}

These are my suggestions. Thank you for reading. I have quoted other successful people not to intimidate you, but for us to learn from them. Remember that those people have failed too. For a more authoritative (and expensive) account, see \cite{ken-bain}. 

\large{Good luck!}

\begin{thebibliography}{00}
    \bibitem{letters} Edward O. Wilson. Letters to a Young Scientist. Liveright; 1st edition, 2013. \href{https://www.amazon.in/Letters-Young-Scientist-Edward-Wilson-ebook/dp/B00AR3551Y}{Amazon Link}.
    \bibitem{know-thyself} Wikipedia. \href{https://en.wikipedia.org/wiki/Know_thyself}{Know Thyself}.
    \bibitem{hamming} \href{https://en.wikipedia.org/wiki/Richard_Hamming}{Richard Hamming}. \href{https://www.cs.virginia.edu/~robins/YouAndYourResearch.html}{You and Your Research}.
    \bibitem{david} David Aldous. \href{https://medium.com/cantors-paradise/on-four-quotes-from-g-h-hardy-36083e56228e}{On Four Quotes by G. H. Hardy}.
    \bibitem{matt} Matt Might. \href{http://matt.might.net/articles/phd-school-in-pictures/}{The Illustrated Guide to a Ph.D.}
    \bibitem{ken-bain} Ken Bain. What the Best College Students Do. Harvard University Press, July 2012. \href{https://www.amazon.in/What-Best-College-Students-Do-ebook/dp/B008L42UJ6}{Amazon Link}.
    \bibitem{dke} Wikipedia. \href{https://en.wikipedia.org/wiki/Dunning%E2%80%93Kruger_effect}{Dunning-Kruger Effect}.
\end{thebibliography}
\end{document}
