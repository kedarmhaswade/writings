\documentclass{article}
\usepackage[utf8]{inputenc}
\usepackage[english]{babel}
\usepackage{epigraph, varwidth}
\usepackage{longtable}
\usepackage[colorinlistoftodos]{todonotes}
\usepackage{chemfig}
\usepackage{gensymb}
\usepackage{mhchem}
\usepackage{color, colortbl}
\usepackage{xcolor}
\definecolor{rnmgreen}{rgb}{0.549, 0.929, 0}
\definecolor{noblepink}{rgb}{0.898,0.741, 0.898}
\definecolor{alkalibeige}{rgb}{0.867,0.682,0.027}
\definecolor{alkalineyellow}{rgb}{0.945,0.945,0.396}
\definecolor{metalloid}{rgb}{0.6196,0.898,0.831}
\definecolor{lanthorange}{rgb}{0.965,0.831,0.635}
\definecolor{actipink}{rgb}{0.98,0.8,0.859}
\definecolor{transred}{rgb}{0.941,0.737,0.737}
\definecolor{postblue}{rgb}{0.675,0.875,0.926}
\definecolor{unk}{rgb}{0.933,0.933,0.933}
\usepackage{hyperref}
\hypersetup{
    colorlinks=true,
    linkcolor=blue,
    filecolor=magenta,
    urlcolor=cyan,
}
\urlstyle{same}

\begin{document}
\listoftodos
\section{Atoms, Molecules, and Ions}
\label{sec: a-m-i}
\subsection{The Periodic Table}
\label{sec: pt}
The Periodic Table is like the mathematical axioms of chemistry. Everyone interested in chemistry should know these axioms readily. There are so many ways in which you can look at the Periodic Table! The ultimate resource for the periodic table is \url{https://ptable.com}.
\begin{center}
            \begin{longtable}{l}
            \caption{Element's class legend} \label{tab: the-color-legend} \\
            \endfirsthead
            \endhead
            \cellcolor{rnmgreen}Reactive non metal\\ 
            \cellcolor{alkalibeige}Alkali metal\\ 
            \cellcolor{alkalineyellow}Alkaline earth metal\\ 
            \cellcolor{lanthorange}Lanthanoids\\ 
            \cellcolor{actipink}Actinoids\\ 
            \cellcolor{transred}Transition metals\\ 
            \cellcolor{postblue}Post-transition metals\\ 
            \cellcolor{metalloid}Metalloids\\ 
            \cellcolor{unk}Unknown\\ 
            \end{longtable}
\begin{longtable}{|l|l|l|l|}
\caption{The Periodic Table of Elements} \label{tab: the-ptable} \\
\hline
    Z & Symbol & Name & A\textsubscript{r, short}\\
\hline
\endfirsthead
\endhead
    1 & H & \cellcolor{rnmgreen}Hydrogen & 1.008\\
 \hline
    2 & He & \cellcolor{noblepink}Helium & 4.0026\\
 \hline
 \hline
    3 & Li & \cellcolor{alkalibeige}Lithium & 6.94\\
 \hline
    4 & Be & \cellcolor{alkalineyellow}Beryllium & 9.0122\\
 \hline
    5 & B & \cellcolor{metalloid}Boron & 10.81\\
 \hline
    6 & C &\cellcolor{rnmgreen}Carbon & 12.011\\
 \hline
    7 & N &\cellcolor{rnmgreen}Nitrogen & 14.007\\
 \hline
    8 & O &\cellcolor{rnmgreen}Oxygen & 15.999\\
 \hline
    9 & F &\cellcolor{rnmgreen}Fluorine & 18.998\\
 \hline
    10 & Ne &\cellcolor{noblepink}Neon & 20.18\\
 \hline
 \hline
    11 & Na & \cellcolor{alkalibeige}Sodium (Natrium) & 22.99\\
 \hline
    12 & Mg &\cellcolor{alkalineyellow}Magnesium & 24.305\\
 \hline
    13 & Al & \cellcolor{postblue}Aluminum & 26.982\\
 \hline
    14 & Si & \cellcolor{metalloid}Silicon & 28.085\\
 \hline
    15 & P & \cellcolor{rnmgreen}Phosphorus & 30.974\\
 \hline
    16 & S & \cellcolor{rnmgreen}Sulfur & 32.06\\
 \hline
    17 & Cl & \cellcolor{rnmgreen}Chlorine & 35.45\\
 \hline
    18 & Ar &\cellcolor{noblepink}Argon & 39.95\\
 \hline
 \hline
    19 & K & \cellcolor{alkalibeige}Potassium (Kalium) & 39.098\\
 \hline
    20 & Ca & \cellcolor{alkalineyellow}Calcium & 40.078\\
 \hline
    21 & Sc & \cellcolor{transred}Scandium & 44.956\\
 \hline
    22 & Ti & \cellcolor{transred}Titanium & 47.867\\
 \hline
    23 & V & \cellcolor{transred}Vanadium & 50.942\\
 \hline
    24 & Cr & \cellcolor{transred}Chromium & 51.996\\
 \hline
    25 & Mn & \cellcolor{transred}Manganese & 54.938\\
 \hline
    26 & Fe & \cellcolor{transred}Iron (Ferrum) & 55.845\\
 \hline
    27 & Co & \cellcolor{transred}Cobalt & 58.933\\
 \hline
    28 & Ni & \cellcolor{transred}Nickel & 58.693\\
 \hline
    29 & Cu & \cellcolor{transred}Copper & 63.546\\
 \hline
    30 & Zn & \cellcolor{transred}Zinc & 65.38\\
 \hline
    31 & Ga & \cellcolor{postblue}Gallium & 69.723\\
 \hline
    32 & Ge & \cellcolor{metalloid}Germanium & 72.63\\
 \hline
    33 & As & \cellcolor{metalloid}Arsenic & 74.922\\
 \hline
    34 & Se & \cellcolor{rnmgreen}Selenium & 78.971\\
 \hline
    35 & Br & \cellcolor{rnmgreen}Bromine & 79.904\\
 \hline
    36 & Kr &\cellcolor{noblepink}Krypton & 83.798\\
 \hline
 \hline
    37 & Rb & \cellcolor{alkalibeige}Rubidium & 85.468\\
 \hline
    38 & Sr & \cellcolor{alkalineyellow}Strontium & 87.62\\
 \hline
    39 & Y & \cellcolor{transred}Yttrium & 88.906\\
 \hline
    40 & Zr & \cellcolor{transred}Zirconium & 91.224\\
 \hline
    41 & Nb & \cellcolor{transred}Niobium & 92.906\\
 \hline
    42 & Mo & \cellcolor{transred}Molymbdenum & 95.95\\
 \hline
    43 & Tc & \cellcolor{transred}Technetium & (98)\footnote{This and other values in a pair of parentheses indicates the absence of a \emph{stable} isotope.}\\
 \hline
    44 & Ru & \cellcolor{transred}Ruthenium & 101.07\\
 \hline
    45 & Rh & \cellcolor{transred}Rhodium & 102.91\\
 \hline
    46 & Pd & \cellcolor{transred}Palladium & 106.42\\
 \hline
    47 & Ag & \cellcolor{transred}Argentum (Silver) & 107.87\\
 \hline
    48 & Cd & \cellcolor{transred}Cadmium & 112.41\\
 \hline
    49 & In & \cellcolor{postblue}Indium & 114.82\\
 \hline
    50 & Tn & \cellcolor{postblue}Stannum (Tin) & 118.71\\
 \hline
    51 & Sb & \cellcolor{metalloid}Stibium (Antimony) & 121.76\\
 \hline
    52 & Te & \cellcolor{metalloid}Tellurium & 127.60\\
 \hline
    53 & I & \cellcolor{rnmgreen}Iodine & 126.90\\
 \hline
    54 & Xe &\cellcolor{noblepink}Xenon & 131.29\\
 \hline
 \hline
    55 & Cs & \cellcolor{alkalibeige}Caesium & 132.91\\
 \hline
    56 & Ba & \cellcolor{alkalineyellow}Barium & 137.33\\
 \hline
    57 & La & \cellcolor{lanthorange}Lanthanum & 138.91\\
 \hline
    58 & Ce & \cellcolor{lanthorange}Cerium & 140.12\\
 \hline
    59 & Pr & \cellcolor{lanthorange}Praseodymium & 140.91\\
 \hline
    60 & Nd & \cellcolor{lanthorange}Neodymium & 144.24\\
 \hline
    61 & Pm & \cellcolor{lanthorange}Promethium & (145)\\
 \hline
    62 & Sm & \cellcolor{lanthorange}Samarium & 150.36\\
 \hline
    63 & Eu & \cellcolor{lanthorange}Europium & 151.96\\
 \hline
    64 & Gd & \cellcolor{lanthorange}Gadolinium & 157.25\\
 \hline
    65 & Tb & \cellcolor{lanthorange}Terbium & 158.93\\
 \hline
    66 & Dy & \cellcolor{lanthorange}Dysprosium & 162.50\\
 \hline
    67 & Ho & \cellcolor{lanthorange}Holmium & 164.93\\
 \hline
    68 & Er & \cellcolor{lanthorange}Erbium & 167.26\\
 \hline
    69 & Tm & \cellcolor{lanthorange}Thulium & 168.93\\
 \hline
    70 & Yb & \cellcolor{lanthorange}Ytterbium & 173.05\\
 \hline
    71 & Lu & \cellcolor{lanthorange}Lutetium & 174.97\\
 \hline
    72 & Hf & \cellcolor{transred}Hafnium & 178.49\\
 \hline
    73 & Ta & \cellcolor{transred}Tantalum & 180.95\\
 \hline
    74 & W & \cellcolor{transred}Wolfram (Tungsten) & 183.84\\
 \hline
    75 & Re & \cellcolor{transred}Rhenium & 186.21\\
 \hline
    76 & Os & \cellcolor{transred}Osmium & 190.23\\
 \hline
    77 & Ir & \cellcolor{transred}Iridium & 192.22\\
 \hline
    78 & Pt & \cellcolor{transred}Platinum & 195.08\\
 \hline
    79 & Au & \cellcolor{transred}Aurum (Gold) & 196.97\\
 \hline
    80 & Hg & \cellcolor{transred}Hydrargyrum (Mercury) & 200.59\\
 \hline
    81 & Tl & \cellcolor{postblue}Thallium & 204.38\\
 \hline
    82 & Pb & \cellcolor{postblue}Plumbum (Lead) & 207.2\\
 \hline
    83 & Bi & \cellcolor{postblue}Bismuth & 208.98\\
 \hline
    84 & Po & \cellcolor{postblue}Polonium & (209)\\
 \hline
    85 & At & \cellcolor{metalloid}Astatine & (210)\\
 \hline
    86 & Rn & \cellcolor{noblepink}Radon & (222)\\
 \hline
 \hline
    87 & Fr & \cellcolor{alkalibeige}Francium & (223)\\
 \hline
    88 & Fr & \cellcolor{alkalineyellow}Radium & (226)\\
 \hline
    89 & Ac & \cellcolor{actipink}Actinium & (227)\\
 \hline
    90 & Th & \cellcolor{actipink}Thorium & 232.04\\
 \hline
    91 & Pa & \cellcolor{actipink}Protactinium & 231.04\\
 \hline
    92 & U & \cellcolor{actipink}Uranium & 238.03\\
 \hline
    93 & Np & \cellcolor{actipink}Neptunium & (237)\\
 \hline
    94 & Pu & \cellcolor{actipink}Plutonium & (244)\\
 \hline
    95 & Am & \cellcolor{actipink}Americium & (243)\\
 \hline
    96 & Cm & \cellcolor{actipink}Curium & (247)\\
 \hline
    97 & Bk & \cellcolor{actipink}Berkelium & (247)\\
 \hline
    98 & Cf & \cellcolor{actipink}Californium & (251)\\
 \hline
    99 & Es & \cellcolor{actipink}Einsteinium & (252)\\
 \hline
    100 & Fm & \cellcolor{actipink}Fermium & (257)\\
 \hline
    101 & Md & \cellcolor{actipink}Mendelevium & (258)\\
 \hline
    102 & No & \cellcolor{actipink}Nobelium & (259)\\
 \hline
    103 & Lr & \cellcolor{actipink}Lawrencium & (266)\\
 \hline
\end{longtable}
\end{center}

Elements are classified as:
\begin{itemize}
    \item Metals
        \begin{itemize}
            \item Alkali metals (Li, Na, K, Rb, Cs, Fr)
            \item Alkaline earth metals (Be, Mg, Ca, Sr, Ba, Ra)
            \item Lanthanoids (La, Ce, Pr, Nd, Pm, Sm, Eu, Gd, Tb, Dy, Ho, Er, Tm, Yb, Lu)
            \item Actinoids (Ac, Th, Pa, U, Np, Pu, Am, Cm, Bk, Cf, Es, Fm, Md, No, Lr)
            \item Transition metals
            \item Post-transition metals
        \end{itemize}
    \item Metalloids
    \item Nonmetals
        \begin{itemize}
            \item Reactive nonmetals(H, ) 
            \item Noble gases
        \end{itemize}
\end{itemize}
\subsection{Molecular and Ionic Compounds}
\label{sec: molecular-ionic}

In ordinary chemical reactions, the nucleus of each atom (and thus the identity of the element) remains unchanged. Electrons, however, can be added to atoms, removed from atoms, or shared with other atoms. 

During the formation of some compounds, atoms gain or lose electrons, and form electrically charged particles called \emph{ions}.
\begin{enumerate}
    \item \emph{Cations} have a positive charge. Cations go to \emph{Cathode} in an electrochemical cell.
    \item \emph{Anions} have a negative charge. Anions go to \emph{Anode} in an electrochemical cell.
\end{enumerate}

Metals readily lose the electrons when, for example, they dissolve in an aqueous solution to release ions: \emph{Na} forms the $Na^+$ ion, \emph{Ca} forms the $Ca^{2+}$ ion.

One can predict if an atom will create cation or anion by looking at its position in the periodic table.

\section{Stoichiometry of Chemical Reactions}
\renewcommand{\epigraphsize}{\small}
\setlength{\epigraphwidth}{0.8\textwidth}
\epigraph
{
    Meeting of two personalities is like the contact of two chemical substances; if there is a reaction, both are transformed.
}
{
    \textit{Carl Jung}
}

\label{sec: stoichiometry}
Stoichiometry helps us determine the \textit{quantitative relations} between \textit{reactants} and \textit{products}. This section also describes various common reactions and their types. Conditions also need to be right for the reactions to take place. Balancing reactions in a \textit{chemical equation} (for matter and charge) is also given sufficient treatment.

\section{Todo's} 
\todo[inline]{Complete this section later}
\section{Entropy}
Entropy is a confusing term. It pays to understand what it means.
At his webpage, \url{https://energyandentropy.com/page/index.html}, Professor Leff asks several questions that challenge the notion that entropy measures the amount of \textit{disorder} in the system. Unfortunately, the disorder metaphor is usually applied \textit{inconsistently} when explaining entropy. Some counterexamples are also ignored:
\begin{itemize}
    \item It \textit{appears} that entropy increases with volume. But water provides a counterexample. Water contracts when heated from 0\degree C to 4\degree C, however, its numerical entropy increases.
    \item It is known that the numerical entropy of 2 kg ice is more than that of 1 kg ice. Since it is still ice, how is more \textit{quantity} of ice more \textit{disordered}?
\end{itemize}

\subsection{Energy-entropy Connection}

\section{Reactions}
\begin{itemize}
    \item \chemfig{R~C=0-R}
    \item \chemfig{A*5(-B=C-D)}
    \item \chemfig{H-C(-[2]H)(-[6]H)-C(=[1]O)-[7]H}
\end{itemize}

\end{document}
