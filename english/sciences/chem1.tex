\documentclass{article}
\usepackage[utf8]{inputenc}
\usepackage[english]{babel}
\usepackage{epigraph, varwidth}
\usepackage{longtable}
\usepackage[colorinlistoftodos]{todonotes}
\usepackage{chemfig}
\usepackage{gensymb}
\usepackage{mhchem}
\usepackage{color, colortbl}
\usepackage{xcolor}
\usepackage{lettrine}
\definecolor{rnmgreen}{rgb}{0.549, 0.929, 0}
\definecolor{noblepink}{rgb}{0.898,0.741, 0.898}
\definecolor{alkalibeige}{rgb}{0.867,0.682,0.027}
\definecolor{alkalineyellow}{rgb}{0.945,0.945,0.396}
\definecolor{metalloid}{rgb}{0.6196,0.898,0.831}
\definecolor{lanthorange}{rgb}{0.965,0.831,0.635}
\definecolor{actipink}{rgb}{0.98,0.8,0.859}
\definecolor{transred}{rgb}{0.941,0.737,0.737}
\definecolor{postblue}{rgb}{0.675,0.875,0.926}
\definecolor{unk}{rgb}{0.933,0.933,0.933}
\usepackage{hyperref}
\hypersetup{
    colorlinks=true,
    linkcolor=blue,
    filecolor=magenta,
    urlcolor=cyan,
}
\urlstyle{same}

\begin{document}
\title{Self-study Notes on OpenStax AP Chemistry Books and Other Resources}
\author{Kedar Mhaswade}
\date{2020-21}
\maketitle
\listoftodos
\section{Atoms, Molecules, and Ions}
\label{sec: a-m-i}
\subsection{Chemical Formulas}
\label{sec: chem-form}
\subsection{The Periodic Table}
\label{sec: pt}
The Periodic Table is like the mathematical axioms of chemistry. Everyone interested in chemistry should know these axioms readily. There are so many ways in which you can look at the Periodic Table! The ultimate resource for the periodic table is \url{https://ptable.com}.
\begin{center}
            \begin{longtable}{l}
            \caption{Element's class legend} \label{tab: the-color-legend} \\
            \endfirsthead
            \endhead
            \cellcolor{rnmgreen}Reactive nonmetal\\ 
            \cellcolor{alkalibeige}Alkali metal\\ 
            \cellcolor{alkalineyellow}Alkaline earth metal\\ 
            \cellcolor{lanthorange}Lanthanoids\\ 
            \cellcolor{actipink}Actinoids\\ 
            \cellcolor{transred}Transition metals\\ 
            \cellcolor{postblue}Post-transition metals\\ 
            \cellcolor{metalloid}Metalloids\\ 
            \cellcolor{unk}Unknown\\ 
            \end{longtable}
\begin{longtable}{|l|l|l|l|}
\caption{The Periodic Table of Elements} \label{tab: the-ptable} \\
\hline
    Z & Symbol & Name & A\textsubscript{r, short}\\
\hline
\endfirsthead
\endhead
    1 & H & \cellcolor{rnmgreen}Hydrogen & 1.008\\
 \hline
    2 & He & \cellcolor{noblepink}Helium & 4.0026\\
 \hline
 \hline
    3 & Li & \cellcolor{alkalibeige}Lithium & 6.94\\
 \hline
    4 & Be & \cellcolor{alkalineyellow}Beryllium & 9.0122\\
 \hline
    5 & B & \cellcolor{metalloid}Boron & 10.81\\
 \hline
    6 & C &\cellcolor{rnmgreen}Carbon & 12.011\\
 \hline
    7 & N &\cellcolor{rnmgreen}Nitrogen & 14.007\\
 \hline
    8 & O &\cellcolor{rnmgreen}Oxygen & 15.999\\
 \hline
    9 & F &\cellcolor{rnmgreen}Fluorine & 18.998\\
 \hline
    10 & Ne &\cellcolor{noblepink}Neon & 20.18\\
 \hline
 \hline
    11 & Na & \cellcolor{alkalibeige}Sodium (Natrium) & 22.99\\
 \hline
    12 & Mg &\cellcolor{alkalineyellow}Magnesium & 24.305\\
 \hline
    13 & Al & \cellcolor{postblue}Aluminum & 26.982\\
 \hline
    14 & Si & \cellcolor{metalloid}Silicon & 28.085\\
 \hline
    15 & P & \cellcolor{rnmgreen}Phosphorus & 30.974\\
 \hline
    16 & S & \cellcolor{rnmgreen}Sulfur & 32.06\\
 \hline
    17 & Cl & \cellcolor{rnmgreen}Chlorine & 35.45\\
 \hline
    18 & Ar &\cellcolor{noblepink}Argon & 39.95\\
 \hline
 \hline
    19 & K & \cellcolor{alkalibeige}Potassium (Kalium) & 39.098\\
 \hline
    20 & Ca & \cellcolor{alkalineyellow}Calcium & 40.078\\
 \hline
    21 & Sc & \cellcolor{transred}Scandium & 44.956\\
 \hline
    22 & Ti & \cellcolor{transred}Titanium & 47.867\\
 \hline
    23 & V & \cellcolor{transred}Vanadium & 50.942\\
 \hline
    24 & Cr & \cellcolor{transred}Chromium & 51.996\\
 \hline
    25 & Mn & \cellcolor{transred}Manganese & 54.938\\
 \hline
    26 & Fe & \cellcolor{transred}Iron (Ferrum) & 55.845\\
 \hline
    27 & Co & \cellcolor{transred}Cobalt & 58.933\\
 \hline
    28 & Ni & \cellcolor{transred}Nickel & 58.693\\
 \hline
    29 & Cu & \cellcolor{transred}Copper & 63.546\\
 \hline
    30 & Zn & \cellcolor{transred}Zinc & 65.38\\
 \hline
    31 & Ga & \cellcolor{postblue}Gallium & 69.723\\
 \hline
    32 & Ge & \cellcolor{metalloid}Germanium & 72.63\\
 \hline
    33 & As & \cellcolor{metalloid}Arsenic & 74.922\\
 \hline
    34 & Se & \cellcolor{rnmgreen}Selenium & 78.971\\
 \hline
    35 & Br & \cellcolor{rnmgreen}Bromine & 79.904\\
 \hline
    36 & Kr &\cellcolor{noblepink}Krypton & 83.798\\
 \hline
 \hline
    37 & Rb & \cellcolor{alkalibeige}Rubidium & 85.468\\
 \hline
    38 & Sr & \cellcolor{alkalineyellow}Strontium & 87.62\\
 \hline
    39 & Y & \cellcolor{transred}Yttrium & 88.906\\
 \hline
    40 & Zr & \cellcolor{transred}Zirconium & 91.224\\
 \hline
    41 & Nb & \cellcolor{transred}Niobium & 92.906\\
 \hline
    42 & Mo & \cellcolor{transred}Molymbdenum & 95.95\\
 \hline
    43 & Tc & \cellcolor{transred}Technetium & (98)\footnote{This and other values in a pair of parentheses indicates the absence of a \emph{stable} isotope.}\\
 \hline
    44 & Ru & \cellcolor{transred}Ruthenium & 101.07\\
 \hline
    45 & Rh & \cellcolor{transred}Rhodium & 102.91\\
 \hline
    46 & Pd & \cellcolor{transred}Palladium & 106.42\\
 \hline
    47 & Ag & \cellcolor{transred}Argentum (Silver) & 107.87\\
 \hline
    48 & Cd & \cellcolor{transred}Cadmium & 112.41\\
 \hline
    49 & In & \cellcolor{postblue}Indium & 114.82\\
 \hline
    50 & Sn & \cellcolor{postblue}Stannum (Tin) & 118.71\\
 \hline
    51 & Sb & \cellcolor{metalloid}Stibium (Antimony) & 121.76\\
 \hline
    52 & Te & \cellcolor{metalloid}Tellurium & 127.60\\
 \hline
    53 & I & \cellcolor{rnmgreen}Iodine & 126.90\\
 \hline
    54 & Xe &\cellcolor{noblepink}Xenon & 131.29\\
 \hline
 \hline
    55 & Cs & \cellcolor{alkalibeige}Caesium & 132.91\\
 \hline
    56 & Ba & \cellcolor{alkalineyellow}Barium & 137.33\\
 \hline
    57 & La & \cellcolor{lanthorange}Lanthanum & 138.91\\
 \hline
    58 & Ce & \cellcolor{lanthorange}Cerium & 140.12\\
 \hline
    59 & Pr & \cellcolor{lanthorange}Praseodymium & 140.91\\
 \hline
    60 & Nd & \cellcolor{lanthorange}Neodymium & 144.24\\
 \hline
    61 & Pm & \cellcolor{lanthorange}Promethium & (145)\\
 \hline
    62 & Sm & \cellcolor{lanthorange}Samarium & 150.36\\
 \hline
    63 & Eu & \cellcolor{lanthorange}Europium & 151.96\\
 \hline
    64 & Gd & \cellcolor{lanthorange}Gadolinium & 157.25\\
 \hline
    65 & Tb & \cellcolor{lanthorange}Terbium & 158.93\\
 \hline
    66 & Dy & \cellcolor{lanthorange}Dysprosium & 162.50\\
 \hline
    67 & Ho & \cellcolor{lanthorange}Holmium & 164.93\\
 \hline
    68 & Er & \cellcolor{lanthorange}Erbium & 167.26\\
 \hline
    69 & Tm & \cellcolor{lanthorange}Thulium & 168.93\\
 \hline
    70 & Yb & \cellcolor{lanthorange}Ytterbium & 173.05\\
 \hline
    71 & Lu & \cellcolor{lanthorange}Lutetium & 174.97\\
 \hline
    72 & Hf & \cellcolor{transred}Hafnium & 178.49\\
 \hline
    73 & Ta & \cellcolor{transred}Tantalum & 180.95\\
 \hline
    74 & W & \cellcolor{transred}Wolfram (Tungsten) & 183.84\\
 \hline
    75 & Re & \cellcolor{transred}Rhenium & 186.21\\
 \hline
    76 & Os & \cellcolor{transred}Osmium & 190.23\\
 \hline
    77 & Ir & \cellcolor{transred}Iridium & 192.22\\
 \hline
    78 & Pt & \cellcolor{transred}Platinum & 195.08\\
 \hline
    79 & Au & \cellcolor{transred}Aurum (Gold) & 196.97\\
 \hline
    80 & Hg & \cellcolor{transred}Hydrargyrum (Mercury) & 200.59\\
 \hline
    81 & Tl & \cellcolor{postblue}Thallium & 204.38\\
 \hline
    82 & Pb & \cellcolor{postblue}Plumbum (Lead) & 207.2\\
 \hline
    83 & Bi & \cellcolor{postblue}Bismuth & 208.98\\
 \hline
    84 & Po & \cellcolor{postblue}Polonium & (209)\\
 \hline
    85 & At & \cellcolor{metalloid}Astatine & (210)\\
 \hline
    86 & Rn & \cellcolor{noblepink}Radon & (222)\\
 \hline
 \hline
    87 & Fr & \cellcolor{alkalibeige}Francium & (223)\\
 \hline
    88 & Ra & \cellcolor{alkalineyellow}Radium & (226)\\
 \hline
    89 & Ac & \cellcolor{actipink}Actinium & (227)\\
 \hline
    90 & Th & \cellcolor{actipink}Thorium & 232.04\\
 \hline
    91 & Pa & \cellcolor{actipink}Protactinium & 231.04\\
 \hline
    92 & U & \cellcolor{actipink}Uranium & 238.03\\
 \hline
    93 & Np & \cellcolor{actipink}Neptunium & (237)\\
 \hline
    94 & Pu & \cellcolor{actipink}Plutonium & (244)\\
 \hline
    95 & Am & \cellcolor{actipink}Americium & (243)\\
 \hline
    96 & Cm & \cellcolor{actipink}Curium & (247)\\
 \hline
    97 & Bk & \cellcolor{actipink}Berkelium & (247)\\
 \hline
    98 & Cf & \cellcolor{actipink}Californium & (251)\\
 \hline
    99 & Es & \cellcolor{actipink}Einsteinium & (252)\\
 \hline
    100 & Fm & \cellcolor{actipink}Fermium & (257)\\
 \hline
    101 & Md & \cellcolor{actipink}Mendelevium & (258)\\
 \hline
    102 & No & \cellcolor{actipink}Nobelium & (259)\\
 \hline
    103 & Lr & \cellcolor{actipink}Lawrencium & (266)\\
 \hline
    104 & Rf & \cellcolor{transred}Rutherfordium & (267)\\
 \hline
    105 & Db & \cellcolor{transred}Dubnium & (268)\\
 \hline
    106 & Sg & \cellcolor{transred}Seaborgium & (269)\\
 \hline
    107 & Bh & \cellcolor{transred}Bohrium & (270)\\
 \hline
    108 & Hs & \cellcolor{transred}Hassium & (277)\\
 \hline
    109 & Mt & Meitnerium & (278)\\
 \hline
    110 & Ds & Damstadtium & (281)\\
 \hline
    111 & Rg & Roentgenium & (282)\\
 \hline
    112 & Cn & Copernicium & (285)\\
 \hline
    113 & Nh & Nihonium & (286)\\
 \hline
    114 & Fl & Flerovium & (289)\\
 \hline
    115 & Mc & Moscovium & (290)\\
 \hline
    116 & Lv & Livermorium & (293)\\
 \hline
    117 & Ts & Tennessine & (294)\\
 \hline
    118 & Og & Oganesson & (294)\\
 \hline
\end{longtable}
\end{center}

Elements are classified as:
\begin{itemize}
    \item Metals
        \begin{itemize}
            \item Alkali metals (Li, Na, K, Rb, Cs, Fr)
            \item Alkaline earth metals (Be, Mg, Ca, Sr, Ba, Ra)
            \item Lanthanoids (La, Ce, Pr, Nd, Pm, Sm, Eu, Gd, Tb, Dy, Ho, Er, Tm, Yb, Lu)
            \item Actinoids (Ac, Th, Pa, U, Np, Pu, Am, Cm, Bk, Cf, Es, Fm, Md, No, Lr)
            \item Transition metals (Sc, Ti, V, Cr, Mn, Fe, Co, Ni, Cu, Zn,\\
                Y, Zr, Nb, Mo, Tc, Ru, Rh, Pd, Ag, Cd,\\
                Hf, Ta, W, Re, Os, Ir, Pt, Au, Hg,\\
                Rf, Db, Sg, Bh, Hs)
            \item Post-transition metals (Al, Ga, In, Sn, Tl, Pb, Bi, Po) 
        \end{itemize}
    \item Metalloids (B, Si, Ge, As, Sb, Te, At)
    \item Nonmetals
        \begin{itemize}
            \item Reactive nonmetals (H, C, N, O, F, P, S, Cl, Se, Br, I) 
            \item Noble gases (He, Ne, Ar, Kr, Xe, Rn)
        \end{itemize}
\end{itemize}
\subsection{Molecular and Ionic Compounds}
\label{sec: molecular-ionic}

In ordinary chemical reactions, the nucleus of each atom (and thus the identity of the element) remains unchanged. Electrons, however, can be added to atoms, removed from atoms, or shared with other atoms. \textbf{The transfer and sharing of electrons govern the chemistry of the elements}.

During the formation of some compounds, atoms gain or lose electrons, and form \emph{electrically charged} particles called \emph{ions}. Main-group (or main-block) elements are the elements in groups 1-2 and 13-18. Many people have argued\footnote{See ``The Place of Zn, Cd, and Hg in the Periodic Table, William Jensen, Journal of Chemical Education. 80(8).} that group-12 elements: Zn(30), Cd(48), and Hg(80) should also be called the main-group elements. They tend to lose electrons to form Cations. The trend is predictable, however, there are several exceptions in that many transition metals (e.g. Cu, Mn) tend to form Cations. Most nonmetals (e.g. halogens) tend to accept electrons and form Anions. 
\begin{enumerate}
    \item \emph{Cations} have a positive charge. Cations go to \emph{Cathode} in an electrochemical cell.
    \item \emph{Anions} have a negative charge. Anions go to \emph{Anode} in an electrochemical cell.
\end{enumerate}

Metals readily lose the electrons when, for example, they dissolve in an aqueous solution to release ions: \emph{Na} forms the \ce{Na+} ion, \emph{Ca} forms the \ce{Ca^2+} ion.

One can predict if an atom will create cation or anion by looking at its position in the periodic table.

\textbf{Monatomic ions} are formed by a single atom (e.g. an atom of \ce{Na} loses an electron to form an \ce{Na+} ion, an atom of \ce{Al} loses three electrons to form an \ce{Al^3+} ion etc.) We also have \textbf{polyatomic ions} which act as discrete units and are a group of \emph{different} bonded atoms with an overall electrical charge. Common polyatomic anions (listed below) react with hydrogen cations (\ce{H+}) to form acidsr:
\begin{itemize}
    \item acetate: \ce{CH3COO-}, forms: acetic acid (\ce{CH3COOH})
    \item cyanide: \ce{CN-}, forms: hydrocyanic acid (\ce{HCN})
    \item azide: \ce{N3-}, forms: hydrazoic acid (\ce{HN3})
    \item carbonate: \ce{CO3^2-}, forms: carbonic acid (\ce{H2CO3})
    \item nitrate: \ce{NO3-}, forms: nitric acid (\ce{HNO3})
    \item nitrite: \ce{NO2-}, forms: nitrous acid (\ce{HNO2})
    \item sulfate: \ce{SO4^2-}, forms: sulfuric acid (\ce{H2SO4})
    \item sulfite: \ce{SO3^2-}, forms: sulfurous acid (\ce{H2SO3})
    \item phosphate: \ce{PO4^3-}, forms: phosphoric acid (\ce{H3PO4})
    \item perchlorate: \ce{ClO4-}, forms: perchloric acid (\ce{HClO4})
    \item chlorate: \ce{ClO3-}, forms: chloric acid (\ce{HClO3})
    \item chlorite: \ce{ClO2-}, forms: chlorous acid (\ce{HClO2})
    \item hypochlorite: \ce{ClO-}, forms: hypochlorous acid (\ce{HClO})
    \item chromate: \ce{CrO4^2-}, forms: chromic acid (\ce{H2CrO4})
    \item dichromate: \ce{Cr2O7^2-}, forms: dichromic acid (\ce{H2Cr2O7})
    \item permanganate: \ce{MnO4-}, forms: permanganic acid (\ce{HMnO4})
\end{itemize}

When electrons are \underline{transferred} and \underline{ions form}, ionic bonds result. \textbf{Ionic bonds are electrostatic forces of attraction between particles (ions) bearing opposite charges}.

When electrons are \underline{shared} and \underline{molecules form}, molecular or covalent bonds result. \textbf{Covalent bonds are forces of attraction between positively charged nuclei of bonded atoms and pair(s) of electrons shared or located between them}.

\section{Stoichiometry of Chemical Reactions}
\renewcommand{\epigraphsize}{\small}
\setlength{\epigraphwidth}{0.8\textwidth}
\epigraph
{
    Meeting of two personalities is like the contact of two chemical substances; if there is a reaction, both are transformed.
}
{
    \textit{Carl Jung}
}

\label{sec: stoichiometry}
Stoichiometry helps us determine the \textit{quantitative relations} between \textit{reactants} and \textit{products}. This section also describes various common reactions and their types. Conditions also need to be right for the reactions to take place. Balancing reactions in a \textit{chemical equation} (for matter and charge) is also given sufficient treatment.

\section{Todo's} 
\todo[inline]{Complete this section later}
\section{Entropy}
Entropy is a confusing term. It pays to understand what it means.
At his webpage, \url{https://energyandentropy.com/page/index.html}, Professor Leff asks several questions that challenge the notion that entropy measures the amount of \textit{disorder} in the system. Unfortunately, the disorder metaphor is usually applied \textit{inconsistently} when explaining entropy. Some counterexamples are also ignored:
\begin{itemize}
    \item It \textit{appears} that entropy increases with volume. But water provides a counterexample. Water contracts when heated from 0\degree C to 4\degree C, however, its numerical entropy increases.
    \item It is known that the numerical entropy of 2 kg ice is more than that of 1 kg ice. Since it is still ice, how is more \textit{quantity} of ice more \textit{disordered}?
\end{itemize}

\subsection{Energy-entropy Connection}

\section{Reactions}
\begin{itemize}
    \item \chemfig{R~C=0-R}
    \item \chemfig{A*5(-B=C-D)}
    \item \chemfig{H-C(-[2]H)(-[6]H)-C(=[1]O)-[7]H}
\end{itemize}

\section{Electronic Structure and Periodic Properties of Elements}
\paragraph{Introduction}
This chapter discusses the \emph{dual nature of light}, the so-called \emph{wave-particle duality}, the electromagnetic radiation (of which visible light is just a short portion) and its relation to the \emph{electronic structure of atoms}. We will also see how electromagnetic radiation can be used to identify elements from thousands of light years away.

\subsection{Electromagnetic Energy}
Newton, of course, was a towering figure in the development of physics and mathematics. He conducted experiments using a glass prism. Glass prism \emph{separates} the visible (white) light into bands of color. He went on to propose a corpuscular theory of light that asserts that light is made of tiny particles moving according to Newton's laws of motion.

Imagine yourself in late 1700's or early 1800's. To ask questions, carry out careful experiments, and objectively conclude, in the spirit of science, some findings that contradict Newton's reasoning was considered heresy. Still, Francesco Grimaldi, Robert Hooke, Christiaan Huygens, Thomas Young, and Augustin-Jean Fresnel (along with François Arago) helped develop the \href{https://en.wikipedia.org/wiki/Light\#Wave_theory}{wave theory of light}. One limitation of wave theory of light was that it proposed the need of a medium for the light to travel. This was later superseded by Maxwell in 1860 when he showed how electromagnetic radiation traveled through space that is devoid of any medium. Later, several scientists developed a quantum theory of light which pictures light as comprised of both waves and particles. Wikipedia mentions the following about a progression from particle theory, to wave theory, to electromagnetic theory, and finally (as of 2020) to quantum theory:

\setlength{\epigraphwidth}{0.9\textwidth}
\epigraph
{
    \lettrine[lines=3]{M}{odern physics} sees light as something that can be described sometimes with mathematics appropriate to one type of macroscopic metaphor (particles), and sometimes another macroscopic metaphor (water waves), but is actually something that cannot be fully imagined. As in the case for radio waves and the X-rays involved in Compton scattering, physicists have noted that electromagnetic radiation tends to behave more like a classical wave at lower frequencies, but more like a classical particle at higher frequencies, but never completely loses all qualities of one or the other. Visible light, which occupies a middle ground in frequency, can easily be shown in experiments to be describable using either a wave or particle model, or sometimes both.
}
{
    \textit{\href{https://en.wikipedia.org/wiki/Light\#Quantum_theory}{Wikipedia 2020}}
}

Visible light and other forms of electromagnetic radiation play important roles in chemistry, since they can be used to infer the energies of electrons within atoms and molecules. Much of modern technology is based on electromagnetic radiation. For example, radio waves from a mobile phone, X-rays used by dentists, the energy used to cook food in your microwave, the radiant heat from red-hot objects, and the light from your television screen are forms of electromagnetic radiation that all exhibit wavelike behavior.

\todo[inline]{Add Appendix for \href{http://www.cavendishscience.org/phys/p_index2.htm}{Young's Experiment}}
\todo[inline]{Take a look at \href{https://archive.org/details/wavetheoryofligh00crewrich/page/n11/mode/2up}{the Wave Theory of Light}}
\todo[inline]{Add Appendixes}
\paragraph{Waves}

\subsection{The Bohr Model}
\subsection{Development of Quantum Theory}
\subsection{Electronic Structure of Atoms (Electron Configurations)}
\subsection{Periodic Variations in Element Properties}
\end{document}
