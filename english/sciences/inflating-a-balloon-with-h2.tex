\documentclass[12pt]{article}
\usepackage[version=4]{mhchem}
\usepackage{amsmath,amssymb,amsthm,amsfonts}   %% AMS mathematics macros
\usepackage{siunitx}
\usepackage{hyperref}
\hypersetup{
    colorlinks=true,
    linkcolor=blue,
    filecolor=magenta,
    urlcolor=cyan,
}

\begin{document}
\title{Using Stochiometry in Simple Chemistry Experiments}
\author{Kedar Mhaswade}
\date{November 2020}
\maketitle
\textbf{Practical Problem}: 
It was decided to inflate a common latex balloon to 5cm radius by the hydrogen gas produced during a single-displacement reaction: \ce{2HCl(aq) + Zn(s) -> ZnCl2(s) + H2(g)} that uses an $N/10$ i.e. 0.1M HCl solution (available in many drugstores, typically in 200ml bottles). How much reactants are needed?

\textbf{Solution}:
We need to inflate the balloon to about 5cm radius. Assuming the balloon to be a sphere gives us the volume of balloon, $V_b$, as:
$ V_b = {\frac{4}{3}\cdot \pi r^3}$\si{\cubic\centi\metre} = 523.599\si{\cubic\centi\metre}.

Assuming the density of hydrogen\footnote{See \url{https://ptable.com/\#Properties/Weight}} at STP as 0.0000899\si{\gram\per\cubic\centi\metre} , the mass of hydrogen is $523.599\times 0.0000899$\si{\g} = $0.0471$\si{\gram}. Since 1\si{\mol} of \ce{H2(g)} has a mass of $2\times 1.008$\si{\gram} = $2.016$\si{\gram}, we need 
$\frac{0.0471}{2.016}$\si{\mol} = 0.0233\si{\mol} \ce{H2(g)}.

Balancing the reaction, 
\ce{2HCl(aq) + Zn(s) -> ZnCl2(s) + H2(g)}, 
we see that to produce 1\si{\mol} \ce{H2(g)} we need 2\si{\mol} \ce{HCl(aq)} and 1\si{\mol} \ce{Zn(s)}.

This means that to produce 0.0233\si{\mol} \ce{H2(g)} we need $2\times0.0233$\si{\mol}=0.0466\si{\mol} \ce{HCl(aq)}. We have a 0.1M HCl (aqueous) solution which means that $\frac{\text{mol of solute}}{\text{litres of solution}} = 0.1$. Clearly, we need $\frac{0.0466}{0.1} = 0.466$\si{L} or \textbf{466\si{ml} HCl}.

The amount of zinc needed = $0.0233\times\text{molar mass of zinc}$=$0.0233\times 65.38$\si{g}=\textbf{1.523\si{g}}.

\let\thefootnote\relax\footnotetext{Author wishes to thank Apoorv Mhaswade and Rujuta Mhaswade for verifying the correctness of the calculations}
\let\thefootnote\relax\footnotetext{See \href{https://chemistry.stackexchange.com/questions/142429/is-this-calculation-of-the-amount-of-required-hydrogen-gas-correct}{Chemistry Stack Exchange} for a discussion}
\end{document}
