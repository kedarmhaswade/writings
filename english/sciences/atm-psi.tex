\documentclass[a6paper]{article}
\usepackage[margin=5mm]{geometry}
\usepackage[version=4]{mhchem}
\usepackage{amsmath,amssymb,amsthm,amsfonts}   %% AMS mathematics macros
\usepackage{siunitx}
\usepackage{xcolor}
\usepackage{hyperref}
\hypersetup{
    colorlinks=true,
    linkcolor=blue,
    filecolor=magenta,
    citecolor=blue,
    urlcolor=purple,
}

\begin{document}
\title{Atmospheric Pressure and Its American Unit}
\author{Kedar Mhaswade}
\date{December 2020}
\maketitle
\textbf{Practical Problem}: 
How do you specify 1 atmospheric pressure in the USA?

\textbf{Solution}:
We won't go into why the USA keeps the FPS system of measurement, but instead, focus on the problem as it is faced by a young and hapless student of science.

In the USA, pressure, which is force per unit area, is expressed in \textbf{pound-force/in\textsuperscript{2}}, \textbf{lbf/in\textsuperscript{2}}, or \textbf{psi} for short. Note that the `p' in psi stands for \emph{pound-force} and not pound, although informally it is taken that way (i.e. pound/in\textsuperscript{2}). If you ask a random American, they might say that the unit of pressure there is ``psi, \emph{pound} per square inch". Historical aspects of the mess resulting from such perceived equivalence are described in \cite{esu}.

As if this were not confusing enough, the definition of \textbf{1 pound-force} or \textbf{1 lbf} was defined as the \emph{weight} of 1 pound \emph{mass}\footnote{We feel sorry for those who were taught in a manner to treat mass and weight interchangeably.} on the surface of earth! This was really short-sighted. Anyway, the weight $W$ in \textbf{lbf} of \textbf{1 pound mass} (or 454 \si{\gram} or 0.454 \si{\kg}) on the surface of earth, is 
\begin{align*}
    W 
    &= mg \dots \footnotemark\\
    &= 1\times 9.8 \; \text{lb m/s\textsuperscript{2}} \\
    &= 0.454\times 9.8\; \text{kg m/s\textsuperscript{2}}\\
    &= 4.4492\; \text{N} \dots \footnotemark
\end{align*}
Thus, \textbf{1 lbf} force is equivalent to \textbf{4.448 \si{\newton}}.

Then, 
\begin{align*}
    1 \; \text{psi} 
    &= 1 \; \text{lbf/in\textsuperscript{2}} \\
    &= 4.4492 \; \text{N/in\textsuperscript{2}} \\
    &= 4.4492 \times 39.37 \times 39.37\; \text{N/m\textsuperscript{2}} \\
    &= 6896.246 \; \si{\pascal}
\end{align*}
Pascal: \si{\pascal} is the SI unit of pressure. Since 1 atomospheric pressure is 101325 \si{\pascal}, from the above we get:
\begin{align*}
    1 \; \text{atm} 
    &= \frac{101325}{6896.246} \;\text{psi} \\
    &= 14.693 \; \text{psi}
\end{align*}

That's it. 1 atmospheric pressure is 14.693 psi. It is enormous pressure that we have evolved to bear easily.
\addtocounter{footnote}{-1}
\footnotetext{$g$ on eath is 9.8 m/s\textsuperscript{2} = 32.2 ft/s\textsuperscript{2}}
\stepcounter{footnote}
\footnotetext{SI unit of force is \si{\newton} (Newton) which is cleanly defined as the force required to accelerate 1 kg \emph{mass} at 1 m/s\textsuperscript{2}}

\begin{thebibliography}{00}
    \bibitem{esu} Fran{\c c}ois Cardarelli. Encyclopaedia of Scientific Units, Weights, and Measures: Their SI Equivalences and Origins. Springer. 2003.
\end{thebibliography}
\end{document}
