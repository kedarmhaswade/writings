\chapter{Arithmetic.}
\label{cha:arithmetic}

\section*{Introduction}
I wanted to write an article about Fixed Point Arithmetic aimed at high-school students. I had some doubts about the early development of computers and the use of fixed and floating point arithmetic. I thought that perhaps we should have strived for using integer arithmetic to get exact answers. I am sure there are limitations to using only integer arithmetic that I want to more fully understand.

I asked ChatGPT about it and it led me to read Volume 2 of Knuth's classics. From time to time, I have had an urge to read his books and solve at least some of the exercises. Like many others, however, I am an eager starter and slow (but deep) doer, reader, and problem-solver. Finally, I feel like I can devote some time to embark on my arduous-looking Knuth journey again. 

As a result, I have postponed my plan to complete the article I was writing. Hopefully, I will return to it someday and it might then contain fewer mistakes.

I record my notes and answers to his exercises beginning with the fourth chapter. The section numbers have been removed but section names match those from the book. I am reading the book on Safari; thanks to O'Reilly and Addison-Wesley Professional. I believe the ebooks published by MSP are at least as faultless as the printed books which are painstakingly proofread and corrected by Knuth and many others.

Arithmetic is fun and it is crucial. In this chapter we analyze algorithms to perform arithmetic operations (addition, subtraction, multiplication, division, and, I guess, exponentiation) on integers, ``floating point'' numbers, extremely large numbers, rational numbers (Knuth identifies rationals as different from floating point numbers), polynomials, and power series. In addition, we discuss radix (base) conversion, factorization, polynomial evaluation.

\section*{Positional Number Systems}
\label{sec:positional}

Numbers are abstract ideas; their \textit{textual representations} are concrete. Textual representations of a number may differ depending upon the notation used.

In the beginning, there were only positive integers. Then came the rational numbers (fractions). The positional notation using base $b$ (or \textit{radix} $b$) is defined for the rational numbers by the rule
\begin{equation} \label{eqn:positional}
%\begin{multlined}
\mathcolor{red}
{
(\dots a_{3}a_{2}a_{1}a_{0}.a_{-1}a_{-2}a_{-3}\dots)_{b} = \\
\dots +a_{2}b^{2}+a_{1}b+a_{0}+a_{-1}b^{-1}+a_{-2}b^{-2}+a_{-3}b^{-3}\dots
}
%\end{multlined}
\end{equation}

In equation (\ref{eqn:positional}), if $b \in \mathbb{N}, b > 1$, then there are $b$ ``digits'': $0\le a_k < b$ that express a non negative rational number \textit{positionally}. The `.' in the notation is variously called the `point', `decimal point', `radix point', `binary point', etc. The number to its left is the \emph{integer part} and that to its right is the \emph{fractional part}. The value (magnitude) of the fractional part is less than 1.

Babylonians used the base- or radix-60 (Sexagesimal) system as early as 1750 {\tiny B.C.} In fact, they had a form of floating point representation! The exponent of the radix (60) was to be \textit{understood from context}. The numbers $2,120,7200,\frac{1}{30}$ can all be written as $2$ because $2=2\cdot 60^0$, $120=2\cdot 60^1$, $7200=2\cdot 60^2$, and $\frac{1}{30}=2\cdot 60^{-1}$ respectively. Here are some interesting features of positional number systems:
\begin{itemize}
\item In Babylonian system, the ``square of $30$ is $15$'' is same as saying ``The square of $\frac{1}{2}$ is $\frac{1}{4}$'' because $30$ means $30\cdot60^{-1}$ which is $\frac{1}{2}$ and $15$ means $15\cdot60^{-1}$ which is $\frac{1}{4}$ in this context.

The reciprocal\footnote{I had to write a computer program using only integer arithmetic to find an accurate representation; I am amazed at the ability of the ancients who did it without any help from machines!} of $81=(1\quad 21)_{60}$ (i.e. $\frac{1}{81}$) is $(44\quad 26\quad 40)_{60}$ and its square (i.e. $\frac{1}{6561}$) is $(32\quad 55\quad 18\quad 31\quad 6\quad 40)_{60}$ (See \ref{app:baby-long-mult} for details). Here too one can see that \textit{the point floats}: It is placed after the first leading zero in the former, whereas after the first two leading zeros in the latter; in both the cases, however, the leading zeros do \emph{not} appear in the notation and we deduce the exponents of 60 from context. 

\item Knuth mentions in \cite{Knuth1972} on historical study of \textit{algorithms} that 60 is a large enough integer for a Babylonian mathematician to estimate (and not requiring an explicit mention of) the correct exponent of 60 relatively easily. Babylonians had developed detailed tables and algorithmic procedures for some fairly involved computational tasks as demonstrated in their clay tablets. Among the tables routinely used by the Babylonians was a table of reciprocals\footnote{I wrote a computer program to generate this table.}. 
\begin{table}[h!]
\caption{A sample Babylonian table of reciprocals.}
\label{tab:reciprocals}
\centering
\setlength{\tabcolsep}{8pt} % separator between columns (standard = 6pt)
\renewcommand{\arraystretch}{1.25} % vertical stretch factor (standard = 1.0)
\begin{tabular}{@{}llllll@{}}
\toprule
Decimal
& Babylonian
& Decimal
& Babylonian
& Decimal
& Babylonian
\\
\midrule
$2$ 
& $30$ 
& $11$
& -- 
& $20$
& $3$ \\
$3$
& $20$ 
& $12$
& $5$ 
& $21$
& -- \\
$4$
& $15$ 
& $13$
& -- 
& $22$
& -- \\
$5$
& $12$ 
& $14$
& -- 
& $23$
& -- \\
$6$
& $10$ 
& $15$
& $4$ 
& $24$
& $2,30$\\
$7$
& -- 
& $16$
& $3,45$ 
& $25$
& $2,24$ \\
$8$
& $7,30$ 
& $17$
& -- 
& $26$
& -- \\
$9$
& $6,40$ 
& $18$
& $3,20$ 
& $27$
& $2, 13, 20$ \\
$10$
& $6$ 
& $19$
& -- 
& $28$
& -- \\
\bottomrule
\end{tabular}
\end{table}


Mayans in Central America apparently were the first ones to develop the fixed point positional notation. Their system was based on radix 20.
\end{itemize}

Knuth then discusses the long history of the positional notation.

A \textit{Decimal Computer} is the one whose basic unit of data was the decimal digit, encoded in one of several schemes like binary-coded decimal (BCD). The Wikipedia article \cite{DecComp} on Decimal Computer says, ``The rapid improvements in general performance of binary machines eroded the value of decimal operations.'' The use of the binary notation was a radical departure from the past then and it was spurred by John von Neumann's design of the computer to which several unsung heroes contributed. W. Buchholz mentions the reasons for using binary instead of decimal notation for an IBM computer in \cite{Buchholz1959}.

The \colorbox{lightgray}{MIX} (and its successor, \colorbox{lightgray}{MMIX}) computer designed by Knuth is oblivious to the base in most cases. Although there are exceptions, most algorithms are unaffected by the choice of the base. 

\subsection*{Representing Negative Numbers}
We are only considering consistent notations for numbers. How numbers are actually realized in a physical computer is an important but separate topic. For now, our concern is about a \textit{representation}. We need to \textit{abstract} out a machine and concentrate on one thing--representation of numbers, all kinds of numbers. Presently, we focus on negative rational numbers. 

\vspace{5pt}
\hrule
\vspace{5pt}

Knuth urges his readers to study the \textit{machine} seriously. He designed his \colorbox{lightgray}{MIX} (these Roman literals represent the number $1009$) computer in the Volume 1 of his books, specified its complete instruction set, implemented most algorithms (that he discussed in his books) in MIX's instruction set, and then upgraded it in 2009 to \colorbox{lightgray}{MMIX} (which is pronounced \textit{Em-micks} and which represents the number $2009$) that uses the RISC (diminished Instruction Set Computer) instruction format. He described \colorbox{lightgray}{MMIX} in what he called the Volume 1 - Fascicle 1. If you read his \textit{The Art of Computer Programming Volume 1: Fundamental Algorithms}, you should skip \S 1.3 MIX and read \S 1.3$'$ MMIX in Fascicle 1 instead. Martin Ruckert ported all the MIX programs to \colorbox{lightgray}{MMIX} in the MMIX Supplement to Knuth's books (see also \cite{mmix}). We study \colorbox{lightgray}{MMIX} in \ref{cha:mmix}. 

\vspace{5pt}
\hrule
\vspace{5pt}

\noindent There are several ways to represent negative numbers in a computer. We consider a decimal computer (\cite{DecComp}) in this section. A digit can represent $10$ distinct values. The `sign' applies to the integer part of the positional notation (\ref{eqn:positional}).

\begin{itemize}
\item Signed magnitude representation. We reserve one \textit{position} for the minus sign. If we have ten positions for the digits, then we need another position to hold just the $-$ sign:
$$
-1234567890
$$
One disadvantage is $0000000000$ and $-0000000000$ are distinct representations for the same number, zero. This may seem like a minor limitation, but an interesting challenge is to represent negative numbers without the provision of a special ``sign'' position (also called the ``sign digit.''

\item Radix complement representation. This eliminates the need for a sign digit. We first introduce the idea of the ``complement'' of a number and then employ it to represent negative integers. The Wikipedia page on the Method of Complements \cite{Complements} gives an excellent overview of how mechanical calculators handled negative numbers using this method in the decimal notation and its generalization, including the arithmetic involved. The generalizations are called radix complement and diminished radix complement. When the radix is $10$, the two complements are called \textit{ten's} complement and \textit{nines'} complement, respectively (Knuth prefers the placement of apostrophes as shown).
\end{itemize}

\begin{definition}\label{def:complements}
The ten's complement of an $n$-digit number $y$ is defined as $y_c=10^n-y$ and the nines' complement is defined as $y_{dc}=(10^n-1)-y$.
\end{definition}

\begin{equation} \label{eqn:complements}
y_c=y_{dc}+1
\end{equation}

follows from Definition(\ref{def:complements}).

In the decimal\footnote{The arithmetic applies to all integral bases greater than 1; there is nothing special about $10$} arithmetic system with $n$ integral digits, the number $10^n$ is called the \emph{modulus}. It is the smallest positive integer that cannot be represented in the system. Calculations are said to be done \emph{modulo $10^n$}. This entails modular arithmetic when manipulating integers. An \textit{overflow} occurs when arithmetic operations result in numbers greater than or equal to the modulus. You can ignore an overflow, but acknowledging it is usually helpful. A general-purpose computer is a finite-resource-machine. Since we are not used to such a limitation (when we need, we can always use another digit in our paper-and-pencil calculations), we cannot always appreciate the issues (e.g. overflow) that it has to address. 

Consider two $n$-digit integers, $x$ and $y$. Let's first see how we might calculate $x+y$. Why, just add like in grade school. And that is the way to do it if no overflow occurs. For instance, if $n=1$, $x=5$, and $x=6$, then, if the overflow is ignored, a computer will report $x+y$ as $11\mod 10^1=11\mod 10=1$.


