%9 Thanks to: https://tex.stackexchange.com/q/565173/64425
\documentclass[a4paper, 12pt]{article}
\usepackage{geometry}
\geometry{
    bottom=20mm,
    left=20mm,
    right=20mm,
}

\usepackage{polyglossia}
\setdefaultlanguage{english}
\setotherlanguages{sanskrit}
\setmainfont[Language=English]{IBM Plex Sans}
% Other font (for Sanskrit): Shobhika
\newfontfamily\sanskritfont[Mapping=velthuis-sanskrit,Script=Devanagari,Language=Default]{Shobhika}
\newcommand \sansletter[1]{
    \fontsize{2cm}{2.4cm}\selectfont
    \sans{#1}
}
\usepackage{xcolor}
\usepackage{hyperref}
\hypersetup{
    colorlinks=true,
    linkcolor=blue,
    anchorcolor=brown
}
\newcommand \eng[1]{
    \textenglish{#1}
}
\newcommand \sans[1]{
    \textsanskrit{#1}
}
\newcommand \shortsansletter[1]{
    \textcolor{red}{\sansletter{#1}}
}
\newcommand \longsansletter[1]{
    \textcolor{blue}{\sansletter{#1}}
}
\newcommand \sansconsonant[1]{
    %\textcolor{purple}{\sansletter{#1}}
    \textcolor{purple} {
    \fontsize{1cm}{1.2cm}\selectfont
    \sans{#1}
    }
}
\begin{document}
\def\dev{\edef~{\string~}\textsanskrit}
\edef~{\string~}
\section {
    Introduction
}
\eng{
    Sanskrit (\sans{sa.msk.rtam}), like Greek and Latin, is an \emph{inflected} language. This means it shows \emph{alteration in form especially by adding affixes}. The bulk of grammatical information is carried by morphology (i.e. the rules for forming admissible words).

    A \emph{morpheme} which is the minimal meaningful language unit, is of one of these types:
    \begin{enumerate}
        \item nominal stem (adjectives, pronouns, and indeclinables(\sans{avyayam}))
            \begin{itemize}
                \item primary 
                \item secondary 
                    \begin{itemize}
                        \item derived from other nominals via affixation (e.g. \sans{kuru} + \sans{a = kaurava, nara} + \sans{tva = naratva})
                        \item derived from verbal roots via affixation (e.g. \sans{gam} + \sans{ana = gamana, k.r} + \sans{t.r = kart.r})
                        \item compounds (e.g. \sans{nara} + \sans{pati = narapati, cakra} + \sans{paa.ni = cakrapaa.ni})
                    \end{itemize}
            \end{itemize}
        \item verbal root (\sans{dhaatu.h})
            \begin{itemize}
                \item primary
                \item secondary
            \end{itemize}
        \item indeclinables (\sans{avyayam})
            \begin{itemize}
                \item particles (e.g. \sans{upari})
                \item pre-positions (e.g. \sans{adhi, pari, anu})
                \item post-positions
                \item adverbs (e.g. \sans{satatam})
                \item connectives (e.g. \sans{ca, vaa})
                \item (occasionally) nouns
            \end{itemize}
    \end{enumerate}

    The nominal stem is characterized by gender as an \emph{intrinsic property} and it is grammatical, usually unrelated to semantics (though the living beings are usually masculine or feminine). There are three genders:
    \begin{itemize}
        \item masculine,
        \item feminine, and
        \item neuter
    \end{itemize}
    Between masculine and feminine, the former is \emph{generic}, meaning it takes precedence. For pronouns, neuter is the most generic.

    Declension of nouns (as we shall later see, declension serves the same purpose that prepositions serve in English) is affected by several factors such as 
    their 
    \begin{itemize}
        \item gender (masculine, feminine, neuter), 
        \item final sound or sounds of the stem (e.g. \sans{akaaraanta, n-kaaraanta}), 
        \item number (singular, dual, and plural), and 
        \item case (\sans{prathamaa}: nominative -- I, \sans{dvitiiyaa}: accusative -- II, \sans{t.rtiiyaa}: instrumental -- III, \sans{caturthii}: dative -- IV, \sans{pa~ncamii}: ablative -- V, \sans{.sa.s.Tii}: genitive -- VI, \sans{saptamii}: locative -- VII, \sans{sa.mbodhanam}: vocative -- VIII). The following list may help describe the usual purpose of cases:
            \begin{enumerate}
                \item nominative -- serving as or indicating the \emph{subject} of the verb (\sans{kartaa})
                \item accusative -- serving as or indicating the \emph{(direct) object} of the verb (\sans{karma})
                \item instrumental -- serving or acting as a \emph{means} or aid (\sans{saadhana, kara.na})
                \item dative -- serving as the \emph{(indirect) object} or the recipient (beneficiary) of the action of the verb (\sans{sampradaanam})
                \item ablative -- indicating the \emph{source or separation} of the agent, instrument, or location (\sans{apaadaanam})
                \item genitive -- expressing \emph{ownership} (--)
                \item locative -- designating the \emph{place or state or action} denoted by the verb (\sans{adhikara.nam})
                \item vocative -- identifying the person being \emph{addressed} (\sans{sambodhanam})
            \end{enumerate}
    \end{itemize}

    Here is the declension of a masculine \sans{akaaraanta} word \sans{deva}:

}
\begin{table}[h]
    \centering
\begin{tabular}{|c|c|c|c|}
    \hline 
    Singular (\sans{ekavacanam}) & Dual (\sans{dvivacanam}) & Plural (\sans{bahuvacanam}) &\\
\hline
    \sans{deva.h} & \sans{devau} & \sans{devaa.h} & \sans{prathamaa}\\
\hline
    \sans{deva.m (devam)} & \sans{devau} & \sans{devaan} & \sans{dvitiiyaa}\\
\hline
    \sans{devena} & \sans{devaabhyaa.m (devaabhyaam)} & \sans{devai.h} & \sans{t.rtiiyaa}\\
\hline
    \sans{devaaya} & \sans{devaabhyaa.m (devaabhyaam)} & \sans{devebhya.h} & \sans{caturthii}\\
\hline
    \sans{devaat} & \sans{devaabhyaa.m (devaabhyaam)} & \sans{devebhya.h} & \sans{pa~ncamii}\\
\hline
    \sans{devasya} & \sans{devayo.h} & \sans{devaanaa.m (devaanaam)} & \sans{.sa.s.Tii}\\
\hline
    \sans{deve} & \sans{devayo.h} & \sans{deve.su} & \sans{saptamii}\\
\hline
    \sans{he deva} & \sans{he devau} & \sans{he devaa.h} & \sans{sambodhanam}\\
\hline
\end{tabular}
\end{table}


    The verbal system is more complex and in the vedic system it is even more so [than the classical system]. The book describes complexities of the vedic verbal system and mentions that classical verbal system gradually got rid of a lot of constructs from the former. The language evolved to favor nominal sentences over verbal sentences. \textbf{However, it seems imperative to me to know at least a few constructs like, for example, \sans{lakaaraa.h}}.


    The most remarkable feature of the classical language is the \emph{compounds} (especially their phenomenal length). Here is an example from Jayadeva's \sans{giitagovinda: \\
    candanacarcitaniilakalevarapiitavasanavanamaalii|\\
    kelicalanma.niku.n.dalama.n.dita ga.n.dayuga.h smita"saalii||\\
    }

  The author believes that several changes occurred to the vedic Sanskrit that Panini grammarized. There was also the influence of local languages. In spite of that, because of Panini's efforts, the language established itself as an ``elite language''. In such evolution, the language's \emph{surface form}s\footnote{Surface form of a word is the form of a word as it appears in the text (e.g. ``goes'' is a surface form of the verb ``go''). Contrast it with the \href{https://wiki.apertium.org/wiki/Lexical_form}{lexical form} which consists of things such as the root, the part of speech etc.} were retained.

    \section {
        The \sans{sa.msk.rta} Alphabet
}
\subsection{Basics}
\subsubsection{Vowels (when not combined with consonants)}
\eng{
    There are 13 vowels of which 5 are \textcolor{red}{short} (\textsanskrit{r+hasva}) and 8 are \textcolor{blue}{long} (\textsanskrit{diirgha}). Not combined with consonants, here are they:
}
\begin{center}
\begin{tabular}{|c|c|}
\hline
    \shortsansletter{a} &
    \longsansletter{aa}\\ 
    \hline
    \shortsansletter{i} &
    \longsansletter{ii} \\
    \hline
    \shortsansletter{u} &
    \longsansletter{uu} \\
    \hline
    \shortsansletter{.r} &
    \longsansletter{.R}\\
    \hline
    \shortsansletter{.l} &
    \sansletter{ }\\
    \hline
    \longsansletter{e} &
    \longsansletter{ai}\\
    \hline
    \longsansletter{o} &
    \longsansletter{au}\\
\hline
\end{tabular}
\end{center}
There are 33 consonants and 2 special consonant clusters. The arrangement is according to the location and mechaism of sound production:

\begin{center}
\begin{tabular}{|c|c|c|c|c|c|}
\hline
    Velar (Guttral) (\sans{ka.n.thya}) &
    \sansconsonant{ka} &
    \sansconsonant{kha} &
    \sansconsonant{ga} &
    \sansconsonant{gha} &
    \sansconsonant{"na} \\
    \hline
    Palatal (\sans{muurdhanya}) &
    \sansconsonant{ca} &
    \sansconsonant{cha} &
    \sansconsonant{ja} &
    \sansconsonant{jha} &
    \sansconsonant{~na} \\
    \hline
    Cerebral (\sans{taalavya}) &
    \sansconsonant{.ta} &
    \sansconsonant{.tha} &
    \sansconsonant{.da} &
    \sansconsonant{.dha} &
    \sansconsonant{.na} \\
    \hline
    Dental (\sans{dantya}) &
    \sansconsonant{ta} &
    \sansconsonant{tha} &
    \sansconsonant{da} &
    \sansconsonant{dha} &
    \sansconsonant{na} \\
    \hline
    Labial (\sans{o.s.thya}) &
    \sansconsonant{pa} &
    \sansconsonant{pha} &
    \sansconsonant{ba} &
    \sansconsonant{bha} &
    \sansconsonant{ma} \\
    \hline
    Semivowels &
    \sansconsonant{ya} &
    \sansconsonant{ra} &
    \sansconsonant{la} &
    \sansconsonant{va} \\
    \hline
    Sibilants (hissing sound) &
    \sansconsonant{"sa} &
    \sansconsonant{.sa} &
    \sansconsonant{sa} \\
    \hline
    Aspirate (rush of air) &
    \sansconsonant{ha} \\
    \hline
    Special consonant clusters &
    \sansconsonant{k.sa} &
    \sansconsonant{j~na}  \\
    \hline
\hline
\end{tabular}
\end{center}

\end{document}
